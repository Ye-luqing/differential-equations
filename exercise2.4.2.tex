\documentclass[a4paper, 12pt]{article} % Font size (can be 10pt, 11pt or 12pt) and paper size (remove a4paper for US letter paper)
\usepackage{amsmath,amsfonts,bm}
\usepackage{hyperref}
\usepackage{amsthm,epigraph} 
\usepackage{amssymb}
\usepackage{framed,mdframed}
\usepackage{graphicx,color} 
\usepackage{mathrsfs,xcolor} 
\usepackage[all]{xy}
\usepackage{fancybox} 
% \usepackage{xeCJK}
\usepackage{CJKutf8}
\newtheorem*{adtheorem}{定理}
% \setCJKmainfont[BoldFont=FZYaoTi,ItalicFont=FZYaoTi]{FZYaoTi}
\definecolor{shadecolor}{rgb}{1.0,0.9,0.9} %背景色为浅红色
\newenvironment{theorem}
{\bigskip\begin{mdframed}[backgroundcolor=gray!40,rightline=false,leftline=false,topline=false,bottomline=false]\begin{adtheorem}}
    {\end{adtheorem}\end{mdframed}\bigskip}
\newtheorem*{bdtheorem}{定义}
\newenvironment{definition}
{\bigskip\begin{mdframed}[backgroundcolor=gray!40,rightline=false,leftline=false,topline=false,bottomline=false]\begin{bdtheorem}}
    {\end{bdtheorem}\end{mdframed}\bigskip}
\newtheorem*{cdtheorem}{习题}
\newenvironment{exercise}
{\bigskip\begin{mdframed}[backgroundcolor=gray!40,rightline=false,leftline=false,topline=false,bottomline=false]\begin{cdtheorem}}
    {\end{cdtheorem}\end{mdframed}\bigskip}
\newtheorem*{ddtheorem}{注}
\newenvironment{remark}
{\bigskip\begin{mdframed}[backgroundcolor=gray!40,rightline=false,leftline=false,topline=false,bottomline=false]\begin{ddtheorem}}
    {\end{ddtheorem}\end{mdframed}\bigskip}
\newtheorem*{edtheorem}{引理}
\newenvironment{lemma}
{\bigskip\begin{mdframed}[backgroundcolor=gray!40,rightline=false,leftline=false,topline=false,bottomline=false]\begin{edtheorem}}
    {\end{edtheorem}\end{mdframed}\bigskip}
\newtheorem*{pdtheorem}{例}
\newenvironment{example}
{\bigskip\begin{mdframed}[backgroundcolor=gray!40,rightline=false,leftline=false,topline=false,bottomline=false]\begin{pdtheorem}}
    {\end{pdtheorem}\end{mdframed}\bigskip}

\usepackage[protrusion=true,expansion=true]{microtype} % Better typography
\usepackage{wrapfig} % Allows in-line images
\usepackage{mathpazo} % Use the Palatino font
\usepackage[T1]{fontenc} % Required for accented characters
\linespread{1.05} % Change line spacing here, Palatino benefits from a slight increase by default

\makeatletter
\renewcommand\@biblabel[1]{\textbf{#1.}} % Change the square brackets for each bibliography item from '[1]' to '1.'
\renewcommand{\@listI}{\itemsep=0pt} % Reduce the space between items in the itemize and enumerate environments and the bibliography

\renewcommand{\maketitle}{ % Customize the title - do not edit title
  % and author name here, see the TITLE block
  % below
  \renewcommand\refname{参考文献}
  \newcommand{\D}{\displaystyle}\newcommand{\ri}{\Rightarrow}
  \newcommand{\ds}{\displaystyle} \renewcommand{\ni}{\noindent}
  \newcommand{\pa}{\partial} \newcommand{\Om}{\Omega}
  \newcommand{\om}{\omega} \newcommand{\sik}{\sum_{i=1}^k}
  \newcommand{\vov}{\Vert\omega\Vert} \newcommand{\Umy}{U_{\mu_i,y^i}}
  \newcommand{\lamns}{\lambda_n^{^{\scriptstyle\sigma}}}
  \newcommand{\chiomn}{\chi_{_{\Omega_n}}}
  \newcommand{\ullim}{\underline{\lim}} \newcommand{\bsy}{\boldsymbol}
  \newcommand{\mvb}{\mathversion{bold}} \newcommand{\la}{\lambda}
  \newcommand{\La}{\Lambda} \newcommand{\va}{\varepsilon}
  \newcommand{\be}{\beta} \newcommand{\al}{\alpha}
  \newcommand{\dis}{\displaystyle} \newcommand{\R}{{\mathbb R}}
  \newcommand{\N}{{\mathbb N}} \newcommand{\cF}{{\mathcal F}}
  \newcommand{\gB}{{\mathfrak B}} \newcommand{\eps}{\epsilon}
  \begin{flushright} % Right align
    {\LARGE\@title} % Increase the font size of the title
    
    \vspace{50pt} % Some vertical space between the title and author name
    
    {\large\@author} % Author name
    \\\@date % Date
    
    \vspace{40pt} % Some vertical space between the author block and abstract
  \end{flushright}
}

% ----------------------------------------------------------------------------------------
%	TITLE
% ----------------------------------------------------------------------------------------
\begin{document}
\begin{CJK}{UTF8}{gkai}
  \title{\textbf{习题2.4.2}} 
  % \setlength\epigraphwidth{0.7\linewidth}
  \author{\small{叶卢庆}\\{\small{杭州师范大学理学院,学号:1002011005}}\\{\small{Email:h5411167@gmail.com}}} % Institution
  \renewcommand{\today}{\number\year. \number\month. \number\day}
  \date{\today} % Date
  
  % ----------------------------------------------------------------------------------------
  
  
  \maketitle % Print the title section
  
  % ----------------------------------------------------------------------------------------
  %	ABSTRACT AND KEYWORDS
  % ----------------------------------------------------------------------------------------
  
  % \renewcommand{\abstractname}{摘要} % Uncomment to change the name of the abstract to something else
  
  % \begin{abstract}
  
  % \end{abstract}
  
  % \hspace*{3,6mm}\textit{关键词:}  % Keywords
  
  % \vspace{30pt} % Some vertical space between the abstract and first section
  
  % ----------------------------------------------------------------------------------------
  %	ESSAY BODY
  % ----------------------------------------------------------------------------------------
  利用适当的变换,求解下列方程
  \begin{exercise}[2.4.2.1]
$$
y'=\cos (x-y).
$$    
  \end{exercise}
  \begin{proof}[解]
    令 $u=x-y$,可得
$$
\frac{du}{dx}=1-\frac{dy}{dx}.
$$
因此
$$
1-\cos u=\frac{du}{dx}.
$$
当 $\cos u\neq 1$ 时,
$$
dx-\frac{1}{1-\cos u}du=0,
$$
设存在二元函数 $\phi(x,u)$,使得
$$
\frac{\pa\phi}{\pa x}=1\ri \phi=x+f(u).
$$
可见,
$$
f'(u)=-\frac{1}{1-\cos u}=-\frac{1}{1-(2\cos^2
  \frac{u}{2}-1)}=-\frac{1}{2\sin^2
  \frac{u}{2}}=\frac{-1}{2}(1+\frac{1}{\tan^2 \frac{u}{2}}).
$$
我们知道,
$$
(\cot u)'=(\frac{1}{\tan u})'=-\frac{1}{\tan^2u}(1+\tan^2u)=-(1+\frac{1}{\tan^2u}).
$$
因此,
$$
f(u)=\cot \frac{u}{2}+C.
$$
于是我们得到通积分
$$
\phi\equiv x+\cot \frac{u}{2}+C=0.
$$
可见,
$$
x+\cot \frac{x-y}{2}+C=0.
$$
当 $\cos u=1$ 时,$x-y=u=2\pi k,k=\cdots,-1,0,1,\cdots$
  \end{proof}
  \begin{exercise}[2.4.2.2]
$$
(3uv+v^2)du+(u^2+uv)dv=0.
$$
  \end{exercise}
  \begin{proof}[解]
    这是关于 $u,v$ 的齐次方程,$v=0$ 的情形是简单的,因此不予讨论.当 $v\neq 0$ 时,设 $u=kv$,其中 $k$ 是 $v$ 的函数.因
    此可得
$$
(3v^2k+v^2)du+(k^2v^2+kv^2)dv=0.
$$
因此
$$
(3k+1)du+(k^2+k)dv=0.
$$
$k=\frac{-1}{3}$ 的情形是简单的,因此不予讨论.因此当 $k\neq \frac{-1}{3}$ 时,
$$
\frac{du}{dv}=-\frac{k^2+k}{3k+1}.
$$
我们知道,
$$
\frac{du}{dv}=\frac{dk}{dv}v+k.
$$
因此
$$
\frac{dk}{dv}v=\frac{-4k^2-2k}{3k+1}.
$$
$k=0$ 或 $\frac{-1}{2}$ 的情形是简单的,因此不予讨论.因此 当 $k\neq \frac{-1}{3}$ 且 $k\neq 0,\frac{-1}{2}$ 时,我们有
$$
\frac{1}{v}dv+\frac{3k+1}{4k^2+2k}dk=0.
$$
设存在二元函数 $\phi(x,y)$,使得
$$
\frac{\pa\phi}{\pa v}=\frac{1}{v}\ri \phi=\ln |v|+f(k).
$$
因此,
$$
f'(k)=\frac{3k+1}{4k^2+2k}=\frac{3k+1}{2k(2k+1)}=\frac{2k+1+k}{2k(2k+1)}=\frac{1}{2k}+\frac{1}{4k+2}.
$$
可见,
$$
f(k)=\frac{1}{2}\ln |2k|+\frac{1}{4}\ln |4k+2|+C.
$$
于是我们得到通积分
$$
\ln |v|+\frac{1}{2}\ln |2k|+\frac{1}{4}|4k+2|+C=0.
$$
也即
$$
\ln |v|+\frac{1}{2}\ln |2 \frac{u}{v}|+\frac{1}{4}|4 \frac{u}{v}+2|+C=0.
$$  \end{proof}
\begin{exercise}[2.4.2.3]
$$
(x^2+y^2+3)\frac{dy}{dx}=2x(2y-\frac{x^2}{y}).
$$
\end{exercise}
\begin{proof}[解]
令 $x=\rho\cos\theta,y=\rho\sin\theta$,则可得
$$
dy=\sin\theta d\rho+\rho d\theta,
$$
$$
dx=\cos\theta d\rho+\rho d\theta.
$$
因此,
$$
\frac{dy}{dx}=\frac{\sin\theta \frac{d\rho}{d\theta}+\rho}{\cos\theta \frac{d\rho}{d\theta}+\rho}.
$$
但是这样之后,我们就不知道该怎样做了.事实上,从如下微分方程的向量场可以
看出,积分曲线并没有绕着原点旋转之后重合的迹象.

\includegraphics[width=0.8\textwidth]{/home/luqing/exercise.png}

因此,通过极坐标代换的方法希望将原微分方程化为变量分离方程似乎是行不通
的.我们尝试别的方法.\\

微分方程两边同时乘以 $y$,可得
$$
(x^2y+y^3+3y)\frac{dy}{dx}=2x(2y^2-x^2).
$$
也即,
$$
\frac{dy}{dx}=\frac{4xy^2-2x^3}{x^2y+y^3+3y}.
$$
我表示我不知道该怎么做了.看了书上答案提示后,我再来做.\\

令 $u=x^2,v=y^2$,则
$$
dv=2ydy,du=2xdx.
$$
因此,
$$
\frac{dy}{dx}=\frac{dv}{du}\frac{x}{y}.
$$
因此
$$
(u+v+3)\frac{dv}{du}=2(2v-u).
$$
然后就解决了.不忍直视……
\end{proof}
\begin{exercise}[2.4.4]
$$
\frac{dy}{dx}=\frac{2x^3+3xy^2-7x}{3x^2y+2y^3-8y}.
$$  
\end{exercise}
\begin{proof}[解]
  可得
$$
\frac{dy}{dx}=\frac{x(2x^2+3y^2-7)}{y(3x^2+2y^2-8)}.
$$
当 $x\neq 0$ 时,也即
$$
\frac{dy}{dx}\frac{y}{x}=\frac{2x^2+3y^2-7}{3x^2+2y^2-8}.
$$
然后令 $u=y^2,v=x^2$,可得
$$
du=2ydy,dv=2vdx.
$$
因此
$$
\frac{dy}{dx}=\frac{du}{dv}\frac{x}{y}.
$$
……然后就解决了.这次没有不忍直视,因为同样的问题我已经见过了.
\end{proof}
  % ----------------------------------------------------------------------------------------
\end{CJK}
\end{document}