\documentclass[a4paper, 12pt]{article} % Font size (can be 10pt, 11pt or 12pt) and paper size (remove a4paper for US letter paper)
\usepackage{amsmath,amsfonts,bm}
\usepackage{hyperref}
\usepackage{amsthm,epigraph} 
\usepackage{amssymb}
\usepackage{framed,mdframed}
\usepackage{graphicx,color} 
\usepackage{mathrsfs,xcolor} 
\usepackage[all]{xy}
\usepackage{fancybox} 
% \usepackage{xeCJK}
\usepackage{CJKutf8}
\newtheorem*{adtheorem}{定理}
% \setCJKmainfont[BoldFont=FZYaoTi,ItalicFont=FZYaoTi]{FZYaoTi}
\definecolor{shadecolor}{rgb}{1.0,0.9,0.9} %背景色为浅红色
\newenvironment{theorem}
{\bigskip\begin{mdframed}[backgroundcolor=gray!40,rightline=false,leftline=false,topline=false,bottomline=false]\begin{adtheorem}}
    {\end{adtheorem}\end{mdframed}\bigskip}
\newtheorem*{bdtheorem}{定义}
\newenvironment{definition}
{\bigskip\begin{mdframed}[backgroundcolor=gray!40,rightline=false,leftline=false,topline=false,bottomline=false]\begin{bdtheorem}}
    {\end{bdtheorem}\end{mdframed}\bigskip}
\newtheorem*{cdtheorem}{习题}
\newenvironment{exercise}
{\bigskip\begin{mdframed}[backgroundcolor=gray!40,rightline=false,leftline=false,topline=false,bottomline=false]\begin{cdtheorem}}
    {\end{cdtheorem}\end{mdframed}\bigskip}
\newtheorem*{ddtheorem}{注}
\newenvironment{remark}
{\bigskip\begin{mdframed}[backgroundcolor=gray!40,rightline=false,leftline=false,topline=false,bottomline=false]\begin{ddtheorem}}
    {\end{ddtheorem}\end{mdframed}\bigskip}
\newtheorem*{edtheorem}{引理}
\newenvironment{lemma}
{\bigskip\begin{mdframed}[backgroundcolor=gray!40,rightline=false,leftline=false,topline=false,bottomline=false]\begin{edtheorem}}
    {\end{edtheorem}\end{mdframed}\bigskip}
\newtheorem*{pdtheorem}{例}
\newenvironment{example}
{\bigskip\begin{mdframed}[backgroundcolor=gray!40,rightline=false,leftline=false,topline=false,bottomline=false]\begin{pdtheorem}}
    {\end{pdtheorem}\end{mdframed}\bigskip}

\usepackage[protrusion=true,expansion=true]{microtype} % Better typography
\usepackage{wrapfig} % Allows in-line images
\usepackage{mathpazo} % Use the Palatino font
\usepackage[T1]{fontenc} % Required for accented characters
\linespread{1.05} % Change line spacing here, Palatino benefits from a slight increase by default

\makeatletter
\renewcommand\@biblabel[1]{\textbf{#1.}} % Change the square brackets for each bibliography item from '[1]' to '1.'
\renewcommand{\@listI}{\itemsep=0pt} % Reduce the space between items in the itemize and enumerate environments and the bibliography

\renewcommand{\maketitle}{ % Customize the title - do not edit title
  % and author name here, see the TITLE block
  % below
  \renewcommand\refname{参考文献}
  \newcommand{\D}{\displaystyle}\newcommand{\ri}{\Rightarrow}
  \newcommand{\ds}{\displaystyle} \renewcommand{\ni}{\noindent}
  \newcommand{\pa}{\partial} \newcommand{\Om}{\Omega}
  \newcommand{\om}{\omega} \newcommand{\sik}{\sum_{i=1}^k}
  \newcommand{\vov}{\Vert\omega\Vert} \newcommand{\Umy}{U_{\mu_i,y^i}}
  \newcommand{\lamns}{\lambda_n^{^{\scriptstyle\sigma}}}
  \newcommand{\chiomn}{\chi_{_{\Omega_n}}}
  \newcommand{\ullim}{\underline{\lim}} \newcommand{\bsy}{\boldsymbol}
  \newcommand{\mvb}{\mathversion{bold}} \newcommand{\la}{\lambda}
  \newcommand{\La}{\Lambda} \newcommand{\va}{\varepsilon}
  \newcommand{\be}{\beta} \newcommand{\al}{\alpha}
  \newcommand{\dis}{\displaystyle} \newcommand{\R}{{\mathbb R}}
  \newcommand{\N}{{\mathbb N}} \newcommand{\cF}{{\mathcal F}}
  \newcommand{\gB}{{\mathfrak B}} \newcommand{\eps}{\epsilon}
  \begin{flushright} % Right align
    {\LARGE\@title} % Increase the font size of the title
    
    \vspace{50pt} % Some vertical space between the title and author name
    
    {\large\@author} % Author name
    \\\@date % Date
    
    \vspace{40pt} % Some vertical space between the author block and abstract
  \end{flushright}
}

% ----------------------------------------------------------------------------------------
%	TITLE
% ----------------------------------------------------------------------------------------
\begin{document}
\begin{CJK}{UTF8}{gkai}
  \title{\textbf{《常微分方程教程》\cite{dinglichang}定理2.3}}
  \setlength\epigraphwidth{0.7\linewidth}

  \author{\small{叶卢庆}\\{\small{杭州师范大学理学院,学
        号:1002011005}}\\{\small{Email:h5411167@gmail.com}}} % Institution
  \renewcommand{\today}{\number\year. \number\month. \number\day}
  \date{\today} % Date
  
  % ----------------------------------------------------------------------------------------
  
  
  \maketitle % Print the title section
  
  % ----------------------------------------------------------------------------------------
  %	ABSTRACT AND KEYWORDS
  % ----------------------------------------------------------------------------------------
  
  % \renewcommand{\abstractname}{摘要} % Uncomment to change the name of the abstract to something else
  
  % \begin{abstract}
  
  % \end{abstract}
  
  % \hspace*{3,6mm}\textit{关键词:} % Keywords
  
  % \vspace{30pt} % Some vertical space between the abstract and first section
  
  % ----------------------------------------------------------------------------------------
  %	ESSAY BODY
  % ----------------------------------------------------------------------------------------
  \begin{theorem}
    设 \textbf{Riccati} 方程
    \begin{equation}
      \label{eq:1}
      \frac{dy}{dx}+ay^2=bx^m,
    \end{equation}
    其中 $a\neq 0$,$b,m$ 都是常数.又设 $x\neq 0$ 和 $y\neq 0$.则当
    \begin{equation}
      \label{eq:2}
      m=0,-2,\frac{-4k}{2k+1},\frac{-4k}{2k-1}(k=1,2,\cdots)
    \end{equation}
    时,方程 \eqref{eq:1} 可通过适当的变换化为变量分离方程.
  \end{theorem}
  \begin{proof}[证明]
    \begin{itemize}
    \item 当 $m=0$ 时,方程 \eqref{eq:1} 为
      \begin{equation}
        \label{eq:3}
        \frac{dy}{dx}+ay^2=b.
      \end{equation}
      即为
      \begin{equation}
        \label{eq:4}
        dy+(ay^2-b)dx=0.
      \end{equation}
      当 $ay^2-b\neq 0$ 时,两边同时除以 $ay^2-b$,可得
      \begin{equation}
        \label{eq:5}
        \frac{1}{ay^2-b}dy+dx=0.
      \end{equation}
      这是一个变量分离方程.
    \item 当 $m=-2$ 时,方程 \eqref{eq:1} 为
      \begin{equation}
        \label{eq:6}
          \frac{dy}{dx}+ay^2=bx^{-2}.
      \end{equation}
令 $y(x)=\frac{u(x)}{x}$.则
$$
\frac{dy}{dx}=\frac{\frac{du}{dx}x-u}{x^2}.
$$
因此 \eqref{eq:6} 化为
\begin{equation}
  \label{eq:7}
  x\frac{du}{dx}-u+au^2=b.
\end{equation}
\eqref{eq:7} 即为
\begin{equation}
  \label{eq:8}
  xdu+(au^2-u-b)dx=0.
\end{equation}
当 $au^2-u-b\neq 0$ 时,\eqref{eq:8} 化为
\begin{equation}
  \label{eq:9}
  \frac{1}{au^2-u-b}du+\frac{1}{x}dx=0.
\end{equation}
这是个变量分离方程.
\item 当 $m=\frac{-4k}{2k+1}$ 时,方程 \eqref{eq:1} 为
  \begin{equation}
    \label{eq:10}
    \frac{dy}{dx}+ay^2=bx^{\frac{-4k}{2k+1}}=bx^{\frac{-2(2k+1)+2}{2k+1}}=bx^{-2}x^{\frac{2}{2k+1}}.
  \end{equation}
\footnote{我们发现,如果 $k\to \infty$,则 \eqref{eq:10} 化为了 \eqref{eq:6},可见
\eqref{eq:6} 是 \eqref{eq:10} 的极限情形.}.下面我们来解
\eqref{eq:10}.令
$$
y=u(x)x^{-1}x^{\frac{1}{2k+1}}=u(x)x^{\frac{-2k}{2k+1}},
$$
则
\begin{equation}
  \label{eq:11}
  \frac{1}{x^{\frac{-4k}{2k+1}}}\frac{dy}{dx}+au^2(x)=b.
\end{equation}
且
$$
\frac{dy}{dx}=\frac{du}{dx}x^{\frac{-2k}{2k+1}}+\frac{-2k}{2k+1}u(x)x^{\frac{-4k-1}{2k+1}}.
$$
因此
\begin{equation}
  \label{eq:12}
  \frac{du}{dx}x^{\frac{2k}{2k+1}}-\frac{2k}{2k+1}u(x)x^{\frac{-1}{2k+1}}+au^2(x)=b.
\end{equation}
令 $t=x^{\frac{2k}{2k+1}}$,则
\begin{equation}
  \label{eq:13}
  \frac{du}{dx}t-u \frac{dt}{dx}+au^2=b.
\end{equation}
饶恕我吧,我真的不知道该怎么做了!
    \end{itemize}
  \end{proof}
  % ----------------------------------------------------------------------------------------
  %	BIBLIOGRAPHY
  % ----------------------------------------------------------------------------------------
  
  \bibliographystyle{unsrt}
  
  \bibliography{sample}
  
  % ----------------------------------------------------------------------------------------
\end{CJK}
\end{document}