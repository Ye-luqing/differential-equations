\documentclass[a4paper, 12pt]{article} % Font size (can be 10pt, 11pt or 12pt) and paper size (remove a4paper for US letter paper)
\usepackage{amsmath,amsfonts,bm}
\usepackage{hyperref,enumerate}
\usepackage{amsthm,epigraph} 
\usepackage{amssymb}
\usepackage{framed,mdframed}
\usepackage{graphicx,color} 
\usepackage{mathrsfs,xcolor} 
\usepackage[all]{xy}
\usepackage{fancybox} 
% \usepackage{xeCJK}
\usepackage{CJKutf8}
\newtheorem*{adtheorem}{定理}
% \setCJKmainfont[BoldFont=FZYaoTi,ItalicFont=FZYaoTi]{FZYaoTi}
\definecolor{shadecolor}{rgb}{1.0,0.9,0.9} %背景色为浅红色
\newenvironment{theorem}
{\bigskip\begin{mdframed}[backgroundcolor=gray!40,rightline=false,leftline=false,topline=false,bottomline=false]\begin{adtheorem}}
    {\end{adtheorem}\end{mdframed}\bigskip}
\newtheorem*{bdtheorem}{定义}
\newenvironment{definition}
{\bigskip\begin{mdframed}[backgroundcolor=gray!40,rightline=false,leftline=false,topline=false,bottomline=false]\begin{bdtheorem}}
    {\end{bdtheorem}\end{mdframed}\bigskip}
\newtheorem*{cdtheorem}{题目}
\newenvironment{exercise}
{\bigskip\begin{mdframed}[backgroundcolor=gray!40,rightline=false,leftline=false,topline=false,bottomline=false]\begin{cdtheorem}}
    {\end{cdtheorem}\end{mdframed}\bigskip}
\newtheorem*{ddtheorem}{注}
\newenvironment{remark}
{\bigskip\begin{mdframed}[backgroundcolor=gray!40,rightline=false,leftline=false,topline=false,bottomline=false]\begin{ddtheorem}}
    {\end{ddtheorem}\end{mdframed}\bigskip}
\newtheorem*{edtheorem}{引理}
\newenvironment{lemma}
{\bigskip\begin{mdframed}[backgroundcolor=gray!40,rightline=false,leftline=false,topline=false,bottomline=false]\begin{edtheorem}}
    {\end{edtheorem}\end{mdframed}\bigskip}
\newtheorem*{pdtheorem}{例}
\newenvironment{example}
{\bigskip\begin{mdframed}[backgroundcolor=gray!40,rightline=false,leftline=false,topline=false,bottomline=false]\begin{pdtheorem}}
    {\end{pdtheorem}\end{mdframed}\bigskip}

\usepackage[protrusion=true,expansion=true]{microtype} % Better typography
\usepackage{wrapfig} % Allows in-line images
\usepackage{mathpazo} % Use the Palatino font
\usepackage[T1]{fontenc} % Required for accented characters
\linespread{1.05} % Change line spacing here, Palatino benefits from a slight increase by default

\makeatletter
\renewcommand\@biblabel[1]{\textbf{#1.}} % Change the square brackets for each bibliography item from '[1]' to '1.'
\renewcommand{\@listI}{\itemsep=0pt} % Reduce the space between items in the itemize and enumerate environments and the bibliography

\renewcommand{\maketitle}{ % Customize the title - do not edit title
  % and author name here, see the TITLE block
  % below
  \renewcommand\refname{参考文献}
  \newcommand{\D}{\displaystyle}\newcommand{\ri}{\Rightarrow}
  \newcommand{\ds}{\displaystyle} \renewcommand{\ni}{\noindent}
  \newcommand{\pa}{\partial} \newcommand{\Om}{\Omega}
  \newcommand{\om}{\omega} \newcommand{\sik}{\sum_{i=1}^k}
  \newcommand{\vov}{\Vert\omega\Vert} \newcommand{\Umy}{U_{\mu_i,y^i}}
  \newcommand{\lamns}{\lambda_n^{^{\scriptstyle\sigma}}}
  \newcommand{\chiomn}{\chi_{_{\Omega_n}}}
  \newcommand{\ullim}{\underline{\lim}} \newcommand{\bsy}{\boldsymbol}
  \newcommand{\mvb}{\mathversion{bold}} \newcommand{\la}{\lambda}
  \newcommand{\La}{\Lambda} \newcommand{\va}{\varepsilon}
  \newcommand{\be}{\beta} \newcommand{\al}{\alpha}
  \newcommand{\dis}{\displaystyle} \newcommand{\R}{{\mathbb R}}
  \newcommand{\N}{{\mathbb N}} \newcommand{\cF}{{\mathcal F}}
  \newcommand{\gB}{{\mathfrak B}} \newcommand{\eps}{\epsilon}
  \begin{flushright} % Right align
    {\LARGE\@title} % Increase the font size of the title
    
    \vspace{50pt} % Some vertical space between the title and author name
    
    {\large\@author} % Author name
    \\\@date % Date
    
    \vspace{40pt} % Some vertical space between the author block and abstract
  \end{flushright}
}

% ----------------------------------------------------------------------------------------
%	TITLE
% ----------------------------------------------------------------------------------------
\begin{document}
\begin{CJK}{UTF8}{gkai}
  \title{\textbf{2013年辽宁高考理科数学第12题解析}} 
  % \setlength\epigraphwidth{0.7\linewidth}
  \author{\small{叶卢庆}\\{\small{杭州师范大学理学院,学号:1002011005}}\\{\small{Email:h5411167@gmail.com}}} % Institution
  \renewcommand{\today}{\number\year. \number\month. \number\day}
  \date{\today} % Date
  
  % ----------------------------------------------------------------------------------------
  
  
  \maketitle % Print the title section
  
  % ----------------------------------------------------------------------------------------
  %	ABSTRACT AND KEYWORDS
  % ----------------------------------------------------------------------------------------
  
  % \renewcommand{\abstractname}{摘要} % Uncomment to change the name of the abstract to something else
  
  % \begin{abstract}
  
  % \end{abstract}
  
  % \hspace*{3,6mm}\textit{关键词:}  % Keywords
  
  % \vspace{30pt} % Some vertical space between the abstract and first section
  
  % ----------------------------------------------------------------------------------------
  %	ESSAY BODY
  % ----------------------------------------------------------------------------------------
\ni在齐建民老师的
\href{http://user.qzone.qq.com/350601384/blog/1384324917}{QQ空间里},我
看到了2013年辽宁高考理科数学第12道选择题.我眼睛一亮,竟然是个一阶线性微
分方程的题目,真不错.因此就写下此篇解析短文.\\\\
\textbf{2013年辽宁高考理科数学第12题}
\begin{shaded}
设函数 $f(x)$ 满足
$x^2f'(x)+2xf(x)=\frac{e^x}{x}$,$f(2)=\frac{e^2}{8}$,则 $x>0$ 时,$f(x)$
\begin{enumerate}[A]
\item 有极大值,无极小值.
\item 有极小值,无极大值.
\item 既有极大值,又有极小值.
\item 既无极大值也无极小值.
\end{enumerate}
\end{shaded}
\begin{proof}[解]
我们直接来解微分方程
\begin{equation}
  \label{eq:1}
 \frac{dy}{dx}x^2+2xy=\frac{e^x}{x}.
\end{equation}
将 \eqref{eq:1} 化为
\begin{equation}
  \label{eq:2}
  x^2dy+(2xy-\frac{e^x}{x})dx=0.
\end{equation}
我们发现 \eqref{eq:2} 是一个恰当微分方程.设二元函数 $\phi(x,y)$ 满足
\begin{equation}
  \label{eq:3}
    \frac{\pa\phi}{\pa y}=x^2
\end{equation}
以及
\begin{equation}
  \label{eq:4}
  \frac{\pa\phi}{\pa x}=2xy-\frac{e^x}{x}.
\end{equation}
由 \eqref{eq:3} 可得,
\begin{equation}
  \label{eq:5}
  \phi(x,y)=yx^2+g(x).
\end{equation}
其中 $g(x)$ 是关于 $x$ 的函数.将 \eqref{eq:5} 代入 \eqref{eq:4},可得
\begin{equation}
  \label{eq:6}
  2xy+g'(x)=2xy-\frac{e^x}{x}\ri g'(x)=-\frac{e^x}{x}.
\end{equation}
因此 $g(x)=-\int \frac{e^x}{x}dx+C$,其中 $C$ 是一个常数.因此我们可得通积
分为
$$
\phi(x,y)\equiv yx^2-\int \frac{e^x}{x}dx+C=0.
$$
令 $H(x)=\int \frac{e^x}{x}dx$,则由题目条件可知,
$$
\frac{e^2}{2}-H(2)+C=0.
$$
可见,
$$
y=\frac{H(x)+\frac{e^2}{2}-H(2)}{x^2}=\frac{\int_2^x \frac{e^x}{x}dx+\frac{e^2}{2}}{x^2}.
$$
因此
$$
y'=\frac{e^{x}-2(\int_2^{x} \frac{e^{x}}{x}dx+\frac{e^2}{2})}{x^3}.
$$
下面我们来看函数
$$
p(x)=e^x-2(\int_2^x \frac{e^x}{x}dx+\frac{e^2}{2}).
$$
易得
$$
p'(x)=e^x-\frac{2e^x}{x}.
$$
可见,当 $0<x<2$ 时,$p(x)$ 递减,当 $x\geq 2$ 时,$p(x)$ 递增,且
$p(2)=0$.可见,$y'$ 恒不小于0,且只有在 $x=2$ 处等于0.可见,$y$ 在 $x>0$
时没有极值点.于是选D.
\end{proof}
  % ----------------------------------------------------------------------------------------
  %	BIBLIOGRAPHY
  % ----------------------------------------------------------------------------------------
  
  \bibliographystyle{unsrt}
  
  \bibliography{sample}
  
  % ----------------------------------------------------------------------------------------
\end{CJK}
\end{document}