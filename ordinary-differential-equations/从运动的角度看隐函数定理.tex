\documentclass[a4paper, 12pt]{article} % Font size (can be 10pt, 11pt or 12pt) and paper size (remove a4paper for US letter paper)
\usepackage{amsmath,amsfonts,bm}
\usepackage{hyperref,verbatim}
\usepackage{amsthm,epigraph} 
\usepackage{amssymb}
\usepackage{framed,mdframed}
\usepackage{graphicx,color} 
\usepackage{mathrsfs,xcolor} 
\usepackage[all]{xy}
\usepackage{fancybox} 
\usepackage{xeCJK}
\newtheorem*{adtheorem}{定理}
\setCJKmainfont[BoldFont=FZYaoTi,ItalicFont=FZYaoTi]{FZYaoTi}
\definecolor{shadecolor}{rgb}{1.0,0.9,0.9} %背景色为浅红色
\newenvironment{theorem}
{\bigskip\begin{mdframed}[backgroundcolor=gray!40,rightline=false,leftline=false,topline=false,bottomline=false]\begin{adtheorem}}
    {\end{adtheorem}\end{mdframed}\bigskip}
\newtheorem*{bdtheorem}{定义}
\newenvironment{definition}
{\bigskip\begin{mdframed}[backgroundcolor=gray!40,rightline=false,leftline=false,topline=false,bottomline=false]\begin{bdtheorem}}
    {\end{bdtheorem}\end{mdframed}\bigskip}
\newtheorem*{cdtheorem}{习题}
\newenvironment{exercise}
{\bigskip\begin{mdframed}[backgroundcolor=gray!40,rightline=false,leftline=false,topline=false,bottomline=false]\begin{cdtheorem}}
    {\end{cdtheorem}\end{mdframed}\bigskip}
\newtheorem*{ddtheorem}{注}
\newenvironment{remark}
{\bigskip\begin{mdframed}[backgroundcolor=gray!40,rightline=false,leftline=false,topline=false,bottomline=false]\begin{ddtheorem}}
    {\end{ddtheorem}\end{mdframed}\bigskip}
\newtheorem*{edtheorem}{引理}
\newenvironment{lemma}
{\bigskip\begin{mdframed}[backgroundcolor=gray!40,rightline=false,leftline=false,topline=false,bottomline=false]\begin{edtheorem}}
    {\end{edtheorem}\end{mdframed}\bigskip}
\newtheorem*{pdtheorem}{例}
\newenvironment{example}
{\bigskip\begin{mdframed}[backgroundcolor=gray!40,rightline=false,leftline=false,topline=false,bottomline=false]\begin{pdtheorem}}
    {\end{pdtheorem}\end{mdframed}\bigskip}

\usepackage[protrusion=true,expansion=true]{microtype} % Better typography
\usepackage{wrapfig} % Allows in-line images
\usepackage{mathpazo} % Use the Palatino font
\usepackage[T1]{fontenc} % Required for accented characters
\linespread{1.05} % Change line spacing here, Palatino benefits from a slight increase by default

\makeatletter
\renewcommand\@biblabel[1]{\textbf{#1.}} % Change the square brackets for each bibliography item from '[1]' to '1.'
\renewcommand{\@listI}{\itemsep=0pt} % Reduce the space between items in the itemize and enumerate environments and the bibliography

\renewcommand{\maketitle}{ % Customize the title - do not edit title
  % and author name here, see the TITLE block
  % below
  \renewcommand\refname{参考文献}
  \newcommand{\D}{\displaystyle}\newcommand{\ri}{\Rightarrow}
  \newcommand{\ds}{\displaystyle} \renewcommand{\ni}{\noindent}
  \newcommand{\pa}{\partial} \newcommand{\Om}{\Omega}
  \newcommand{\om}{\omega} \newcommand{\sik}{\sum_{i=1}^k}
  \newcommand{\vov}{\Vert\omega\Vert} \newcommand{\Umy}{U_{\mu_i,y^i}}
  \newcommand{\lamns}{\lambda_n^{^{\scriptstyle\sigma}}}
  \newcommand{\chiomn}{\chi_{_{\Omega_n}}}
  \newcommand{\ullim}{\underline{\lim}} \newcommand{\bsy}{\boldsymbol}
  \newcommand{\mvb}{\mathversion{bold}} \newcommand{\la}{\lambda}
  \newcommand{\La}{\Lambda} \newcommand{\va}{\varepsilon}
  \newcommand{\be}{\beta} \newcommand{\al}{\alpha}
  \newcommand{\dis}{\displaystyle} \newcommand{\R}{{\mathbb R}}
  \newcommand{\N}{{\mathbb N}} \newcommand{\cF}{{\mathcal F}}
  \newcommand{\gB}{{\mathfrak B}} \newcommand{\eps}{\epsilon}
  \begin{flushright} % Right align
    {\LARGE\@title} % Increase the font size of the title
    
    \vspace{50pt} % Some vertical space between the title and author name
    
    {\large\@author} % Author name
    \\\@date % Date
    
    \vspace{40pt} % Some vertical space between the author block and abstract
  \end{flushright}
}

% ----------------------------------------------------------------------------------------
%	TITLE
% ----------------------------------------------------------------------------------------
\begin{document}
\title{\textbf{从运动的角度看隐函数定理}}
% \setlength\epigraphwidth{0.7\linewidth}
\author{\small{叶卢庆}\\{\small{杭州师范大学理学院,学
      号:1002011005}}\\{\small{Email:h5411167@gmail.com}}} % Institution
\renewcommand{\today}{\number\year. \number\month. \number\day}
\date{\today} % Date
  
% ----------------------------------------------------------------------------------------
  
  
\maketitle % Print the title section
  
% ----------------------------------------------------------------------------------------
% ABSTRACT AND KEYWORDS
% ----------------------------------------------------------------------------------------
  
% \renewcommand{\abstractname}{摘要} % Uncomment to change the name of the abstract to something else
  
% \begin{abstract}
  
% \end{abstract}
  
% \hspace*{3,6mm}\textit{关键词:} % Keywords
  
% \vspace{30pt} % Some vertical space between the abstract and first section
  
% ----------------------------------------------------------------------------------------
% ESSAY BODY
% ----------------------------------------------------------------------------------------
隐函数存在定理叙述如下:
\begin{theorem}[隐函数存在定理]
  设 $ f:\mathbf{R}^{n+m}\rightarrow\mathbf{R}^m$ 为连续可微函数, $
  \mathbf{R}^{n+m}$ 中的元素写 成$
  \mathbf{(x,y)}=(x_1,\cdots,x_n,y_1,\cdots,y_m)$ 的形式.对于任
  意 一 点$ (\mathbf{a,b}) = (a_{1},\cdots, a_{n}, b_{1},\cdots,b_m)$
  使 得$ f(\mathbf{a,b}) = 0$,隐函数存在定理给出了一个充分 条件,用来判
  断 能否在$ (\mathbf{a,b})$附近定义一 个$ \mathbf{y}$关于$
  \mathbf{x}$的函数$ g$,使得只 要$ f(\mathbf{x,y})=0$,就有 $
  \mathbf{y}=g(\mathbf{x})$.严格地说,就是存 在$ \mathbf{a}$和$
  \mathbf{b}$的邻域$ U$ 和 $ V$,使得$ g$是 从 $ U$ 到 $ V$ 的函
  数,并 且$ g$的函数图像满足
$$
\{(\mathbf{x},g(\mathbf{x}))\}=\{ (\mathbf{x},\mathbf{y}) |
f(\mathbf{x},\mathbf{y}) =0 \}\cap(U\times V).
$$

要使的这样的函数$ g$存在,函数$ f$ 的雅可比矩阵一定要满足一定的性
质.对于 给 定的一点 $ (a,b)$,$ f$ 的雅可比矩阵写做

\begin{align*}
  (Df)(\mathbf{a},\mathbf{b})=\left[\begin{matrix}\frac{\partial
        f_1}{\partial x_1}(\mathbf{a},\mathbf{b}) &
      \cdots&\frac{\partial f_1}{\partial
        x_n}(\mathbf{a},\mathbf{b})\\ \vdots&\ddots&\vdots\\
      \frac{\partial f_m}{\partial
        x_1}(\mathbf{a},\mathbf{b})&\cdots&\frac{\partial
        f_m}{\partial
        x_n}(\mathbf{a},\mathbf{b}) \end{matrix}\right|\left. \begin{matrix}
      \frac{\partial f_1}{\partial y_1}(\mathbf{a},\mathbf{b}) &
      \cdots & \frac{\partial f_1}{\partial
        y_m}(\mathbf{a},\mathbf{b})\\ \vdots & \ddots & \vdots\\
      \frac{\partial f_m}{\partial y_1}(\mathbf{a},\mathbf{b}) &
      \cdots & \frac{\partial f_m}{\partial
        y_m}(\mathbf{a},\mathbf{b})\\ \end{matrix}\right]=[X|Y]
\end{align*}

隐函数存在定理说明了:如果 $ Y$ 是一个可逆的矩阵,那么满足前面性质的$
U,V$ 和函数 $ g$ 就会存在.
\end{theorem}
隐函数存在定理是反函数定理的简单推理.因此为了从运动的角度解释隐函数存在
定理,我们需要先从运动的角度解释反函数定理.反函数定理叙述如下:
\begin{theorem}[反函数定理]
  设 $ E$ 是 $ \mathbf{R}^n$ 的开集 合,并设 $
  f:E\rightarrow\mathbf{R}^n$ 是 在 $ E$上连续可微的函数.假设$ x_0\in
  E$ 使得线性映射 $ f'(x_0):\mathbf{R}^n\rightarrow \mathbf{R}^n$ 是可
  逆的,那么 存在含有 $ x_0$ 的开集 $ U\subset E$ 以及含有$ f(x_0)$ 的开
  集 $ V\subset \mathbf{R}^n$,使得函数 $ f$ 是从 $ U$ 到 $ V$的双
  射\footnote{严格地来说,此 时函数 $ f$ 的定义域已经改变,从 $ E$ 变成
    了 $ U$,已经不再是同一个函数,但 是这影响不大,我们忽略这种差别.}.而
  且 $ f$ 的逆映射 $ f^{-1}:V\rightarrow U$ 在点 $ f(x_0)$处可微,满足$
  {\displaystyle (f^{-1})'(f(x_0))=(f'(x_0))^{-1}.}$
\end{theorem}
设空间 $\mathbf{R}^n$ 中有流体.随着时间的流逝,
% ----------------------------------------------------------------------------------------
% BIBLIOGRAPHY
% ----------------------------------------------------------------------------------------
  
\bibliographystyle{unsrt}
  
\bibliography{sample}
  
% ----------------------------------------------------------------------------------------
\end{document}