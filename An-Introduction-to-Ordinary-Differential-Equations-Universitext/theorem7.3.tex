\documentclass[a4paper, 12pt]{article} % Font size (can be 10pt, 11pt or 12pt) and paper size (remove a4paper for US letter paper)
\usepackage{amsmath,amsfonts,bm}
\usepackage{hyperref,verbatim}
\usepackage{amsthm,epigraph} 
\usepackage{amssymb}
\usepackage{framed,mdframed}
\usepackage{graphicx,color} 
\usepackage{mathrsfs,xcolor} 
\usepackage[all]{xy}
\usepackage{fancybox} 
\usepackage{xeCJK}
\newtheorem*{adtheorem}{Theorem}
\setCJKmainfont[BoldFont=FZYaoTi,ItalicFont=FZYaoTi]{FZYaoTi}
\definecolor{shadecolor}{rgb}{1.0,0.9,0.9} %背景色为浅红色
\newenvironment{theorem}
{\bigskip\begin{mdframed}[backgroundcolor=gray!40,rightline=false,leftline=false,topline=false,bottomline=false]\begin{adtheorem}}
    {\end{adtheorem}\end{mdframed}\bigskip}
\newtheorem*{bdtheorem}{定义}
\newenvironment{definition}
{\bigskip\begin{mdframed}[backgroundcolor=gray!40,rightline=false,leftline=false,topline=false,bottomline=false]\begin{bdtheorem}}
    {\end{bdtheorem}\end{mdframed}\bigskip}
\newtheorem*{cdtheorem}{习题}
\newenvironment{exercise}
{\bigskip\begin{mdframed}[backgroundcolor=gray!40,rightline=false,leftline=false,topline=false,bottomline=false]\begin{cdtheorem}}
    {\end{cdtheorem}\end{mdframed}\bigskip}
\newtheorem*{ddtheorem}{注}
\newenvironment{remark}
{\bigskip\begin{mdframed}[backgroundcolor=gray!40,rightline=false,leftline=false,topline=false,bottomline=false]\begin{ddtheorem}}
    {\end{ddtheorem}\end{mdframed}\bigskip}
\newtheorem*{edtheorem}{引理}
\newenvironment{lemma}
{\bigskip\begin{mdframed}[backgroundcolor=gray!40,rightline=false,leftline=false,topline=false,bottomline=false]\begin{edtheorem}}
    {\end{edtheorem}\end{mdframed}\bigskip}
\newtheorem*{pdtheorem}{例}
\newenvironment{example}
{\bigskip\begin{mdframed}[backgroundcolor=gray!40,rightline=false,leftline=false,topline=false,bottomline=false]\begin{pdtheorem}}
    {\end{pdtheorem}\end{mdframed}\bigskip}

\usepackage[protrusion=true,expansion=true]{microtype} % Better typography
\usepackage{wrapfig} % Allows in-line images
\usepackage{mathpazo} % Use the Palatino font
\usepackage[T1]{fontenc} % Required for accented characters
\linespread{1.05} % Change line spacing here, Palatino benefits from a slight increase by default

\makeatletter
\renewcommand\@biblabel[1]{\textbf{#1.}} % Change the square brackets for each bibliography item from '[1]' to '1.'
\renewcommand{\@listI}{\itemsep=0pt} % Reduce the space between items in the itemize and enumerate environments and the bibliography

\renewcommand{\maketitle}{ % Customize the title - do not edit title
  % and author name here, see the TITLE block
  % below
  \renewcommand\refname{参考文献}
  \newcommand{\D}{\displaystyle}\newcommand{\ri}{\Rightarrow}
  \newcommand{\ds}{\displaystyle} \renewcommand{\ni}{\noindent}
  \newcommand{\pa}{\partial} \newcommand{\Om}{\Omega}
  \newcommand{\om}{\omega} \newcommand{\sik}{\sum_{i=1}^k}
  \newcommand{\vov}{\Vert\omega\Vert} \newcommand{\Umy}{U_{\mu_i,y^i}}
  \newcommand{\lamns}{\lambda_n^{^{\scriptstyle\sigma}}}
  \newcommand{\chiomn}{\chi_{_{\Omega_n}}}
  \newcommand{\ullim}{\underline{\lim}} \newcommand{\bsy}{\boldsymbol}
  \newcommand{\mvb}{\mathversion{bold}} \newcommand{\la}{\lambda}
  \newcommand{\La}{\Lambda} \newcommand{\va}{\varepsilon}
  \newcommand{\be}{\beta} \newcommand{\al}{\alpha}
  \newcommand{\dis}{\displaystyle} \newcommand{\R}{{\mathbb R}}
  \newcommand{\N}{{\mathbb N}} \newcommand{\cF}{{\mathcal F}}
  \newcommand{\gB}{{\mathfrak B}} \newcommand{\eps}{\epsilon}
  \begin{flushright} % Right align
    {\LARGE\@title} % Increase the font size of the title
    
    \vspace{50pt} % Some vertical space between the title and author name
    
    {\large\@author} % Author name
    \\\@date % Date
    
    \vspace{40pt} % Some vertical space between the author block and abstract
  \end{flushright}
}

% ----------------------------------------------------------------------------------------
%	TITLE
% ----------------------------------------------------------------------------------------
\begin{document}
\title{\textbf{Theorem7.3}} 
% \setlength\epigraphwidth{0.7\linewidth}
\author{\small{叶卢庆}\\{\small{杭州师范大学理学院,学号:1002011005}}\\{\small{Email:h5411167@gmail.com}}} % Institution
\renewcommand{\today}{\number\year. \number\month. \number\day}
\date{\today} % Date

% ----------------------------------------------------------------------------------------


\maketitle % Print the title section

% ----------------------------------------------------------------------------------------
%	ABSTRACT AND KEYWORDS
% ----------------------------------------------------------------------------------------

% \renewcommand{\abstractname}{摘要} % Uncomment to change the name of the abstract to something else

% \begin{abstract}

% \end{abstract}

% \hspace*{3,6mm}\textit{关键词:}  % Keywords

% \vspace{30pt} % Some vertical space between the abstract and first section

% ----------------------------------------------------------------------------------------
%	ESSAY BODY
% ----------------------------------------------------------------------------------------
\begin{theorem}[7.3]
Let $u(x),p(x),q(x)$ be nonnegative continuous functions in the
interval $|x-x_0|\leq a$ and
\begin{equation}
  \label{eq:1}
  u(x)\leq p(x)+\left|\int_{x_0}^xq(t)u(t)dt\right| for |x-x_0|\leq a.
\end{equation}
Then the following inequality holds:
\begin{equation}
  \label{eq:2}
  u(x)\leq p(x)+\left|\int_{x_0}^xp(t)q(t)\exp\left(\left
        |\int_t^xq(s)ds\right|\right)dt\right| for |x-x_0|\leq a.
\end{equation}
\end{theorem}
\begin{proof}
According to symmetry,let us assume that
$x\geq x_0$.Then \eqref{eq:1} turned  to 
\begin{equation}
  \label{eq:3}
  u(x)\leq p(x)+\int_{x_0}^xq(t)u(t)dt \ for\  0\leq x-x_{0}\leq a.
\end{equation}
So there exists a non-negative continuous function $g(x)$,such that
\begin{equation}
  \label{eq:4}
  u(x)+g(x)-p(x)=\int_{x_0}^xq(t)u(t)dt\ for\ 0\leq x-x_0\leq a.
\end{equation}
Let $H(x)=u(x)+g(x)-p(x)$,then we have $H(x_0)=0$ and 
\begin{equation}
  \label{eq:5}
  H'(x)=q(x)u(x)\ for \ 0\leq x-x_0\leq a.
\end{equation}
\eqref{eq:5} is equivalent to
\begin{equation}
  \label{eq:6}
  H'(x)=q(x)(H(x)+p(x)-g(x)).
\end{equation}
\eqref{eq:6} is equivalent to
\begin{equation}\label{eq:7}
  H'(x)-q(x)H(x)+(-q(x)p(x)+q(x)g(x))=0.
\end{equation}
\eqref{eq:7} is an ordinary differential equation for $x$ and
$H(x)$,solving it gives
\begin{align*}
  H(x)e^{\int -q(x)dx}=C+\int \left((q(x)p(x)-q(x)g(x))e^{\int -q(x)dx}\right)dx.
\end{align*}
Let 
$$
T(x)=\int -q(x)dx,K(x)=\int \left((q(x)p(x)-q(x)g(x))e^{T(x)}\right)dx
$$
Then we have
$$
H(x)e^{T(x)}=C+K(x).
$$
Because $H(x_0)=0$,we have
$$
C+K(x_0)=H(x_0)e^{T(x_0)}=0,
$$
which means that $C=-K(x_0)$.So
$$
H(t)e^{T(t)}=K(t)-K(x_0)=\int_{x_0}^t \left((q(x)p(x)-q(x)g(x))e^{T(x)}\right)dx
$$
So
$$
(u(t)-(p(t)-g(t)))e^{T(t)}=\int_{x_0}^t\left(q(x)(p(x)-g(x))e^{T(x)}\right)dx
$$
So
\begin{align*}
  u(t)-(p(t)-g(t))&=\int_{x_0}^tq(x)(p(x)-g(x))e^{\int_t^x
    -q(s)ds}dx\\&=\int_{x_0}^tq(x)(p(x)-g(x))e^{\int_x^t
    q(s)ds}dx\\&=\int_{x_0}^tq(x)p(x)e^{\int_x^t
    q(s)ds}dx-\int_{x_0}^tq(x)g(x)e^{\int_x^tq(s)ds}dx.
\end{align*}
So
\begin{align*}
  u(t)-p(t)\leq \int_{x_0}^tq(x)p(x)e^{\int_x^tq(s)ds}.
\end{align*}
\end{proof}
% ----------------------------------------------------------------------------------------
%	BIBLIOGRAPHY
% ----------------------------------------------------------------------------------------

\bibliographystyle{unsrt}

\bibliography{sample}

% ----------------------------------------------------------------------------------------
\end{document}
So
$$
H(x)e^{T(x)}-H(x_0)e^{T(x_0)}=K(x)-K(x_0),
$$
which means that
\begin{align*}
  H(x)e^{T(x)}-H(x_0)e^{T(x_0)}&=\int_{x_0}^x\left((q(x)p(x)-q(x)g(x))e^{\int
    -q(x)dx}\right)dx\\&\leq \int_{x_0}^xq(x)p(x)e^{\int-q(x)dx}dx.
\end{align*}
