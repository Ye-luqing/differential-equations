\documentclass[a4paper, 12pt]{article} % Font size (can be 10pt, 11pt or 12pt) and paper size (remove a4paper for US letter paper)
\usepackage{amsmath,amsfonts,bm}
\usepackage{hyperref,verbatim}
\usepackage{amsthm,epigraph} 
\usepackage{amssymb}
\usepackage{framed,mdframed}
\usepackage{graphicx,color} 
\usepackage{mathrsfs,xcolor} 
\usepackage[all]{xy}
\usepackage{fancybox} 
% \usepackage{xeCJK}
\usepackage{CJKutf8}
\newtheorem*{adtheorem}{Theorem}
% \setCJKmainfont[BoldFont=FZYaoTi,ItalicFont=FZYaoTi]{FZYaoTi}
\definecolor{shadecolor}{rgb}{1.0,0.9,0.9} %背景色为浅红色
\newenvironment{theorem}
{\bigskip\begin{mdframed}[backgroundcolor=gray!40,rightline=false,leftline=false,topline=false,bottomline=false]\begin{adtheorem}}
    {\end{adtheorem}\end{mdframed}\bigskip}
\newtheorem*{bdtheorem}{Definition}
\newenvironment{definition}
{\bigskip\begin{mdframed}[backgroundcolor=gray!40,rightline=false,leftline=false,topline=false,bottomline=false]\begin{bdtheorem}}
    {\end{bdtheorem}\end{mdframed}\bigskip}
\newtheorem*{cdtheorem}{Exercise}
\newenvironment{exercise}
{\bigskip\begin{mdframed}[backgroundcolor=gray!40,rightline=false,leftline=false,topline=false,bottomline=false]\begin{cdtheorem}}
    {\end{cdtheorem}\end{mdframed}\bigskip}
\newtheorem*{ddtheorem}{Remark}
\newenvironment{remark}
{\bigskip\begin{mdframed}[backgroundcolor=gray!40,rightline=false,leftline=false,topline=false,bottomline=false]\begin{ddtheorem}}
    {\end{ddtheorem}\end{mdframed}\bigskip}
\newtheorem*{edtheorem}{Lemma}
\newenvironment{lemma}
{\bigskip\begin{mdframed}[backgroundcolor=gray!40,rightline=false,leftline=false,topline=false,bottomline=false]\begin{edtheorem}}
    {\end{edtheorem}\end{mdframed}\bigskip}
\newtheorem*{pdtheorem}{Corollary}
\newenvironment{corollary}
{\bigskip\begin{mdframed}[backgroundcolor=gray!40,rightline=false,leftline=false,topline=false,bottomline=false]\begin{pdtheorem}}
    {\end{pdtheorem}\end{mdframed}\bigskip}

\usepackage[protrusion=true,expansion=true]{microtype} % Better typography
\usepackage{wrapfig} % Allows in-line images
\usepackage{mathpazo} % Use the Palatino font
\usepackage[T1]{fontenc} % Required for accented characters
\linespread{1.05} % Change line spacing here, Palatino benefits from a slight increase by default

\makeatletter
\renewcommand\@biblabel[1]{\textbf{#1.}} % Change the square brackets for each bibliography item from '[1]' to '1.'
\renewcommand{\@listI}{\itemsep=0pt} % Reduce the space between items in the itemize and enumerate environments and the bibliography

\renewcommand{\maketitle}{ % Customize the title - do not edit title
  % and author name here, see the TITLE block
  % below
  \newcommand{\D}{\displaystyle}\newcommand{\ri}{\Rightarrow}
  \newcommand{\ds}{\displaystyle} \renewcommand{\ni}{\noindent}
  \newcommand{\pa}{\partial} \newcommand{\Om}{\Omega}
  \newcommand{\om}{\omega} \newcommand{\sik}{\sum_{i=1}^k}
  \newcommand{\vov}{\Vert\omega\Vert} \newcommand{\Umy}{U_{\mu_i,y^i}}
  \newcommand{\lamns}{\lambda_n^{^{\scriptstyle\sigma}}}
  \newcommand{\chiomn}{\chi_{_{\Omega_n}}}
  \newcommand{\ullim}{\underline{\lim}} \newcommand{\bsy}{\boldsymbol}
  \newcommand{\mvb}{\mathversion{bold}} \newcommand{\la}{\lambda}
  \newcommand{\La}{\Lambda} \newcommand{\va}{\varepsilon}
  \newcommand{\be}{\beta} \newcommand{\al}{\alpha}
  \newcommand{\dis}{\displaystyle} \newcommand{\R}{{\mathbb R}}
  \newcommand{\N}{{\mathbb N}} \newcommand{\cF}{{\mathcal F}}
  \newcommand{\gB}{{\mathfrak B}} \newcommand{\eps}{\epsilon}
  \begin{flushright} % Right align
    {\LARGE\@title} % Increase the font size of the title
    
    \vspace{50pt} % Some vertical space between the title and author name
    
    {\large\@author} % Author name
    \\\@date % Date
    
    \vspace{40pt} % Some vertical space between the author block and abstract
  \end{flushright}
}

% ----------------------------------------------------------------------------------------
%	TITLE
% ----------------------------------------------------------------------------------------
\begin{document}
\begin{CJK}{UTF8}{gkai}
  \title{\textbf{Corollary7.6}}
  % \setlength\epigraphwidth{0.7\linewidth}
  \author{\small{Luqing Ye}\\{\small{Hangzhou Normal
        University,Student
        ID:1002011005}}\\{\small{Email:h5411167@gmail.com}}} % Institution
  \renewcommand{\today}{\number\year. \number\month. \number\day}
  \date{\today} % Date
  
  
  
  \maketitle
  \begin{corollary}
    Let $u(x),p(x),q(x)$ be nonnegative continuous functions in the
    interval $|x-x_0|\leq a$ and
    \begin{equation}
      \label{eq:1}
      u(x)\leq p(x)+\left|\int_{x_0}^xq(t)u(t)dt\right| for |x-x_0|\leq a.
    \end{equation}
    And $p(x)=c_0+c_1|x-x_0|$,$q(x)=c_2$,where $c_0,c_1,c_2$ are
    non-negative constants, Then the following inequality holds:
    \begin{equation}
      \label{eq:2}
      u(x)\leq \left(c_0+\frac{c_1}{c_2}\right)\exp(c_2|x-x_0|)-\frac{c_1}{c_2}.
    \end{equation}
  \end{corollary}
  \begin{proof}
By symmetry,assume that $x\geq x_0$.And from Theorem7.3,we know that
    \begin{equation}
      \label{eq:3}
      u(x)\leq p(x)+\int_{x_0}^xp(t)q(t)\exp\left(\int_t^xq(s)ds\right)dt
    \end{equation}
We know that
$$
\int_{x_0}^xe^{c_2(x-t)}dt=-\frac{1}{c_2}e^{0}+\frac{1}{c_2}e^{c_2(x-x_{0})}.
$$
Now we calculate 
$$
\int te^{c_2(x-t)}dt=\int t\frac{e^{c_2x}}{e^{c_2t}}dt=e^{c_{2}x}\int
te^{-c_2t}dt=e^{c_2x}(-\frac{1}{c_{2}^{2}}e^{-c_{2}t}(c_{2}t+1))+Ce^{c_2x}
$$
So
\begin{align*}
  \int_{x_0}^xte^{c_2(x-t)}dt&=e^{c_2x}(-\frac{1}{c_2^2}e^{-c_{2}x}(c_2x+1))-e^{c_2x}(-\frac{1}{c_2^2}e^{-c_2x_0}(c_2x_0+1))\\&=\frac{-1}{c_2^2}(c_2x+1)+e^{c_2x}(\frac{1}{c_2^2}e^{-c_2x_0}(c_2x_0+1)).
\end{align*}
So \eqref{eq:3} is equivalent to
\begin{align*}
  u(x)&\leq c_0+c_1(x-x_0)+\int_{x_0}^x(c_0+c_1(t-x_0))c_2\exp\left(\int_t^{x}c_{2}ds\right)dt\\&=c_0+c_1(x-x_0)+\int_{x_0}^xc_2(c_0+c_1(t-x_0))e^{(x-t)c_2}dt\\&=c_0+c_1(x-x_0)+c_0c_{2}\int_{x_0}^xe^{c_2(x-t)}dt+c_1c_2\int_{x_0}^x(t-x_0)e^{c_2(x-t)}dt\\&=c_0+c_1(x-x_{0})+c_{0}c_{2}(-\frac{1}{c_2}e^0+\frac{1}{c_2}e^{c_2(x-x_{0})})-c_1c_2x_0(-\frac{1}{c_2}+\frac{1}{c_2}e^{c_2(x-x_{0})})\\&+c_1c_2(\frac{-1}{c_2^2}(c_2x+1)+e^{c_2x}(\frac{1}{c_2^2}e^{-c_2x_0}(c_2x_0+1)))\\&=e^{c_2(x-x_0)}(c_0+\frac{c_1}{c_2})-\frac{c_1}{c_2}.
\end{align*}
Done.
  \end{proof}
  % ----------------------------------------------------------------------------------------
  %	BIBLIOGRAPHY
  % ----------------------------------------------------------------------------------------
  
  \bibliographystyle{unsrt}
  
  \bibliography{sample}
  
  % ----------------------------------------------------------------------------------------
\end{CJK}
\end{document}