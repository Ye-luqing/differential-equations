\documentclass[a4paper, 12pt]{article} % Font size (can be 10pt, 11pt or 12pt) and paper size (remove a4paper for US letter paper)
\usepackage{amsmath,amsfonts,bm}
\usepackage{hyperref}
\usepackage{amsthm} 
\usepackage{amssymb}
\usepackage{framed,mdframed}
\usepackage{graphicx,color} 
\usepackage{mathrsfs,xcolor} 
\usepackage[all]{xy}
\usepackage{fancybox} 
\usepackage{xeCJK}
\newtheorem*{adtheorem}{定理}
\setCJKmainfont[BoldFont=FZYaoTi,ItalicFont=FZYaoTi]{FZYaoTi}
\definecolor{shadecolor}{rgb}{1.0,0.9,0.9} %背景色为浅红色
\newenvironment{theorem}
{\bigskip\begin{mdframed}[backgroundcolor=gray!40,rightline=false,leftline=false,topline=false,bottomline=false]\begin{adtheorem}}
    {\end{adtheorem}\end{mdframed}\bigskip}
\newtheorem*{bdtheorem}{定义}
\newenvironment{definition}
{\bigskip\begin{mdframed}[backgroundcolor=gray!40,rightline=false,leftline=false,topline=false,bottomline=false]\begin{bdtheorem}}
    {\end{bdtheorem}\end{mdframed}\bigskip}
\newtheorem*{cdtheorem}{习题}
\newenvironment{exercise}
{\bigskip\begin{mdframed}[backgroundcolor=gray!40,rightline=false,leftline=false,topline=false,bottomline=false]\begin{cdtheorem}}
    {\end{cdtheorem}\end{mdframed}\bigskip}
\newtheorem*{ddtheorem}{注}
\newenvironment{remark}
{\bigskip\begin{mdframed}[backgroundcolor=gray!40,rightline=false,leftline=false,topline=false,bottomline=false]\begin{ddtheorem}}
    {\end{ddtheorem}\end{mdframed}\bigskip}
\newtheorem*{edtheorem}{引理}
\newenvironment{lemma}
{\bigskip\begin{mdframed}[backgroundcolor=gray!40,rightline=false,leftline=false,topline=false,bottomline=false]\begin{edtheorem}}
    {\end{edtheorem}\end{mdframed}\bigskip}
\newtheorem*{pdtheorem}{例}
\newenvironment{example}
{\bigskip\begin{mdframed}[backgroundcolor=gray!40,rightline=false,leftline=false,topline=false,bottomline=false]\begin{pdtheorem}}
    {\end{pdtheorem}\end{mdframed}\bigskip}

\usepackage[protrusion=true,expansion=true]{microtype} % Better typography
\usepackage{wrapfig} % Allows in-line images
\usepackage{mathpazo} % Use the Palatino font
\usepackage[T1]{fontenc} % Required for accented characters
\linespread{1.05} % Change line spacing here, Palatino benefits from a slight increase by default

\makeatletter
\renewcommand\@biblabel[1]{\textbf{#1.}} % Change the square brackets for each bibliography item from '[1]' to '1.'
\renewcommand{\@listI}{\itemsep=0pt} % Reduce the space between items in the itemize and enumerate environments and the bibliography

\renewcommand{\maketitle}{ % Customize the title - do not edit title
  % and author name here, see the TITLE block
  % below
  \renewcommand\refname{参考文献}
  \newcommand{\D}{\displaystyle}\newcommand{\ri}{\Rightarrow}
  \newcommand{\ds}{\displaystyle} \renewcommand{\ni}{\noindent}
  \newcommand{\pa}{\partial} \newcommand{\Om}{\Omega}
  \newcommand{\om}{\omega} \newcommand{\sik}{\sum_{i=1}^k}
  \newcommand{\vov}{\Vert\omega\Vert} \newcommand{\Umy}{U_{\mu_i,y^i}}
  \newcommand{\lamns}{\lambda_n^{^{\scriptstyle\sigma}}}
  \newcommand{\chiomn}{\chi_{_{\Omega_n}}}
  \newcommand{\ullim}{\underline{\lim}} \newcommand{\bsy}{\boldsymbol}
  \newcommand{\mvb}{\mathversion{bold}} \newcommand{\la}{\lambda}
  \newcommand{\La}{\Lambda} \newcommand{\va}{\varepsilon}
  \newcommand{\be}{\beta} \newcommand{\al}{\alpha}
  \newcommand{\dis}{\displaystyle} \newcommand{\R}{{\mathbb R}}
  \newcommand{\N}{{\mathbb N}} \newcommand{\cF}{{\mathcal F}}
  \newcommand{\gB}{{\mathfrak B}} \newcommand{\eps}{\epsilon}
  \begin{flushright} % Right align
    {\LARGE\@title} % Increase the font size of the title
    
    \vspace{50pt} % Some vertical space between the title and author name
    
    {\large\@author} % Author name
    \\\@date % Date
    
    \vspace{40pt} % Some vertical space between the author block and abstract
  \end{flushright}
}

% ----------------------------------------------------------------------------------------
%	TITLE
% ----------------------------------------------------------------------------------------

\title{\textbf{《常微分方程教程》\footnote{丁同仁,李承治编著,高等教育
      出版社第二版}性质2.3.1,2.3.3,2.3.4证明}} 

\author{\small{叶卢庆}\\{\small{杭州师范大学理学院,学号:1002011005}}\\{\small{Email:h5411167@gmail.com}}} % Institution
\renewcommand{\today}{\number\year. \number\month. \number\day}
\date{\today} % Date

% ----------------------------------------------------------------------------------------

\begin{document}
\maketitle
丁同仁,李承治编著,高等教育出版社第二版的《常微分方程教程》第34页性质1声
称\begin{shaded}
\ni齐次线性方程
$$
\frac{dy}{dx}+p(x)y=0
$$
的解或者恒等于0,或者恒不等于0.其中函数 $p(x),q(x)$ 在区间 $I=(a,b)$ 上
连续.
\end{shaded}
\begin{proof}[证明]
显然,$y=0$ 是化为
$$
dy+p(x)ydx=0.
$$
方程两边同时乘以非零函数 $u(x)$,得到
$$
u(x)dy+u(x)p(x)ydx=0.
$$
设如上方程是恰当的,则
$$
\frac{du(x)}{dx}=u(x)p(x).
$$
于是 $u(x)=ae^{\int p(x)dx}$,不妨设 $a=1$.则我们得到恰当方程
$$
e^{\int p(x)dx}dy+e^{\int p(x)dx}p(x)ydx=0.
$$
设二元函数 $\phi(x,y)$ 满足
$$
\frac{\pa \phi}{\pa y}=e^{\int p(x)dx}\ri \phi=ye^{\int p(x)dx}+f(x).
$$
因此,
$$
f'(x)+ye^{\int p(x)dx}p(x)=yp(x)e^{\int p(x)dx}.
$$
得到 $f(x)=C$,于是通积分为
$$
ye^{\int p(x)dx}+C=0.
$$
可得 $C=0,y=0$ 是一个解,当 $C\neq 0$ 时,
$$
y=\frac{-C}{e^{\int p(x)dx}}
$$
永远不为0.
\end{proof}\bigskip\bigskip
性质2.3.3声称
\begin{shaded}
\ni齐次线性方程
\begin{equation}\label{eq:8.41}
\frac{dy}{dx}+p(x)y=0
\end{equation}
的任何解的线性组合仍是它的解;齐次线性方程 \eqref{eq:8.41}的任一解与非
齐次线性方程
\begin{equation}
  \label{eq:8.58}
  \frac{dy}{dx}+p(x)y=q(x)
\end{equation}
的任一解之和是非齐次线性方程组 \eqref{eq:8.58} 的解;非齐次线性方程
\eqref{eq:8.58} 的任意两解之差必定是相应齐次线性方程 \eqref{eq:8.41}的
解.其中函数 $p(x),q(x)$ 在区间 $I=(a,b)$ 上连续.
\end{shaded}
\begin{proof}[解]
设 $y=g_1(x),y=g_2(x)$ 是方程 \eqref{eq:8.41} 的两个解,即
$$
g_1'(x)+p(x)g_1(x)=0,g_2'(x)+p(x)g_2(x)=0.
$$
因此,
$$
(g_1(x)+g_2(x))'+p(x)(g_1(x)+g_2(x))=0.
$$
完毕.其余的证明完全类似.这里主要用到了导数运算的线性性质.奇怪吧?导数运
算竟然有线性性质.呵呵,因为其实导数的本质就是线性映射.
\end{proof}
下面来看性质2.3.4
\begin{shaded}
\ni 非齐次线性方程 \eqref{eq:8.58} 的任一解与相应的齐次线性方程 \eqref{eq:8.41}
 的通解之和构成非齐次线性方程 \eqref{eq:8.58} 的通解.
\end{shaded}
\begin{proof}[解]
  :-),具体的证明省略了,从解的公式可以很容易看出来.其实不看解的公式也完
  全可以证出来,齐次线性方程 \eqref{eq:8.41} 的通解的作用是
  带来一个独立的任意给定的常数.
\end{proof}
% BIBLIOGRAPHY
% ----------------------------------------------------------------------------------------

\bibliographystyle{unsrt} \bibliography{sample}

% ----------------------------------------------------------------------------------------
\end{document}