\documentclass[a4paper, 12pt]{article} % Font size (can be 10pt, 11pt or 12pt) and paper size (remove a4paper for US letter paper)
\usepackage{amsmath,amsfonts,bm}
\usepackage{hyperref}
\usepackage{amsthm} 
\usepackage{amssymb}
\usepackage{framed,mdframed}
\usepackage{graphicx,color} 
\usepackage{mathrsfs,xcolor} 
\usepackage[all]{xy}
\usepackage{fancybox} 
\usepackage{xeCJK}
\newtheorem{adtheorem}{定理}
\setCJKmainfont[BoldFont=FZYaoTi,ItalicFont=FZYaoTi]{FZYaoTi}
\definecolor{shadecolor}{rgb}{1.0,0.9,0.9} %背景色为浅红色
\newenvironment{theorem}
{\begin{mdframed}[backgroundcolor=gray!40,rightline=false,leftline=false,topline=false,bottomline=false]\begin{adtheorem}}
    {\end{adtheorem}\end{mdframed}\bigskip}
\newtheorem*{bdtheorem}{定义}
\newenvironment{definition}
{\begin{mdframed}[backgroundcolor=gray!40,rightline=false,leftline=false,topline=false,bottomline=false]\begin{bdtheorem}}
    {\end{bdtheorem}\end{mdframed}\bigskip}
\newtheorem*{cdtheorem}{习题}
\newenvironment{exercise}
{\begin{mdframed}[backgroundcolor=gray!40,rightline=false,leftline=false,topline=false,bottomline=false]\begin{cdtheorem}}
    {\end{cdtheorem}\end{mdframed}\bigskip}
\newtheorem{ddtheorem}{注}
\newenvironment{remark}
{\begin{mdframed}[backgroundcolor=gray!40,rightline=false,leftline=false,topline=false,bottomline=false]\begin{ddtheorem}}
    {\end{ddtheorem}\end{mdframed}\bigskip}
\newtheorem{edtheorem}{引理}
\newenvironment{lemma}
{\begin{mdframed}[backgroundcolor=gray!40,rightline=false,leftline=false,topline=false,bottomline=false]\begin{edtheorem}}
    {\end{edtheorem}\end{mdframed}\bigskip}
\usepackage[protrusion=true,expansion=true]{microtype} % Better typography
\usepackage{wrapfig} % Allows in-line images
\usepackage{mathpazo} % Use the Palatino font
\usepackage[T1]{fontenc} % Required for accented characters
\linespread{1.05} % Change line spacing here, Palatino benefits from a slight increase by default

\makeatletter
\renewcommand\@biblabel[1]{\textbf{#1.}} % Change the square brackets for each bibliography item from '[1]' to '1.'
\renewcommand{\@listI}{\itemsep=0pt} % Reduce the space between items in the itemize and enumerate environments and the bibliography

\renewcommand{\maketitle}{ % Customize the title - do not edit title
  % and author name here, see the TITLE block
  % below
  \renewcommand\refname{参考文献}
  \newcommand{\D}{\displaystyle}\newcommand{\ri}{\Rightarrow}
  \newcommand{\ds}{\displaystyle} \renewcommand{\ni}{\noindent}
  \newcommand{\pa}{\partial} \newcommand{\Om}{\Omega}
  \newcommand{\om}{\omega} \newcommand{\sik}{\sum_{i=1}^k}
  \newcommand{\vov}{\Vert\omega\Vert} \newcommand{\Umy}{U_{\mu_i,y^i}}
  \newcommand{\lamns}{\lambda_n^{^{\scriptstyle\sigma}}}
  \newcommand{\chiomn}{\chi_{_{\Omega_n}}}
  \newcommand{\ullim}{\underline{\lim}} \newcommand{\bsy}{\boldsymbol}
  \newcommand{\mvb}{\mathversion{bold}} \newcommand{\la}{\lambda}
  \newcommand{\La}{\Lambda} \newcommand{\va}{\varepsilon}
  \newcommand{\be}{\beta} \newcommand{\al}{\alpha}
  \newcommand{\dis}{\displaystyle} \newcommand{\R}{{\mathbb R}}
  \newcommand{\N}{{\mathbb N}} \newcommand{\cF}{{\mathcal F}}
  \newcommand{\gB}{{\mathfrak B}} \newcommand{\eps}{\epsilon}
  \begin{flushright} % Right align
    {\LARGE\@title} % Increase the font size of the title
    
    \vspace{50pt} % Some vertical space between the title and author name
    
    {\large\@author} % Author name
    \\\@date % Date
    
    \vspace{40pt} % Some vertical space between the author block and abstract
  \end{flushright}
}

% ----------------------------------------------------------------------------------------
%	TITLE
% ----------------------------------------------------------------------------------------

\title{\textbf{《常微分方程教程》习题2-2,4\\[2em]一个跟踪问题}} 

\author{\small{叶卢庆}\\{\small{杭州师范大学理学院,学号:1002011005}}\\{\small{Email:h5411167@gmail.com}}} % Institution
\renewcommand{\today}{\number\year. \number\month. \number\day}
\date{\today} % Date

% ----------------------------------------------------------------------------------------

\begin{document}
\maketitle % Print the title section

% ----------------------------------------------------------------------------------------
%	ABSTRACT AND KEYWORDS
% ----------------------------------------------------------------------------------------

% \renewcommand{\abstractname}{摘要} % Uncomment to change the name of the abstract to something else

% \begin{abstract}

% \end{abstract}

% \hspace*{3,6mm}\textit{关键词:}  % Keywords

% \vspace{30pt} % Some vertical space between the abstract and first section

% ----------------------------------------------------------------------------------------
%	ESSAY BODY
% ----------------------------------------------------------------------------------------
\begin{exercise}[2-2,4]
跟踪:设某 $A$ 从 $Oxy$  平面的原点出发,沿 $x$ 轴正方向前进;同时某 $B$
从点 $(0,b)$ 开始跟踪 $A$,即 $B$ 的运动方向永远指向 $A$ 并与 $A$ 保持
等距 $b$.试求 $B$ 的光滑运动轨迹.
\end{exercise}
\begin{proof}[解]
设在时刻 $t$ 的时候 $A$ 位于 $(f(t),0)$.其中 $f(0)=0$,且 $f(t)$ 是关于
$t$ 的严格单调增函数.设在时刻 $t$ 的
$B$ 位于 $(P(t),Q(t))$,其中 $P(0)=0,Q(0)=b$.不妨设 $b\neq 0$,否则 $B$
的运动将与 $A$ 重合,这是没什么意思的,再根据对称性不妨设 $b>0$.且由于 $B$ 的路径光滑,因此关于
$t$ 的函数 $P,Q$ 都是连续可微的.由于 $B$ 的方向一直指向 $A$,因此
\begin{equation}
  \label{eq:10.51}
  (P'(t),Q'(t))=k(f(t)-P(t),-Q(t)).
\end{equation}
其中 $k>0$.由于 $A,B$ 间距始终为 $b$,因此
\begin{equation}
  \label{eq:10.52}
  [P(t)-f(t)]^2+Q(t)^2=b^2.
\end{equation}
当 $Q(t)\neq 0$ 时,$Q'(t)$ 也不为0.此时 将(1) 代入 (2) 可得
\begin{equation}
  \label{eq:11.02}
  (P'(t))^2+(Q'(t))^2=b^2k^2=b^2\frac{Q'(t)^{2}}{Q(t)^{2}}.
\end{equation}
于是我们就得到了微分方程
\begin{equation}
  \label{eq:11.54}
  (\frac{P'(t)}{Q'(t)})^2+1=\frac{b^2}{Q(t)^2}.
\end{equation}
也就是
$$
(\frac{dP(t)}{dQ(t)})^2+1=\frac{b^2}{Q(t)^2}.
$$
也即
$$
\frac{dx}{dy}=-\sqrt{(\frac{b}{y})^2-1}.
$$
令 $\frac{b}{y}=\cosh a$.其中 $a\in \mathbf{R}^{+}$,于是,
$$
\frac{dy}{da}=\frac{-b\tanh a}{\cosh a}.
$$
且
$$
\frac{dx}{dy}=-\sinh a.
$$
因此,
$$
\frac{dx}{da}=b(\tanh a)^2=b-b\tanh'a.
$$
因此,
$$
x=ba-b\tanh a+C.
$$
因此,
$$
x=b\cosh^{-1}\frac{b}{y}-b\tanh(\cosh^{-1}\frac{b}{y})+C.
$$
将初始条件 $x=0,y=b$ 代入,解得 $C=0$.于是 $B$ 的光滑轨迹为
$$
x=b\cosh^{-1}\frac{b}{y}-b\tanh(\cosh^{-1}\frac{b}{y}).
$$
通过这个方程,我们发现 $B$ 的运动轨迹和 $A$ 的运动无关!\\

当 $Q(t)=0$ 时,易得 $B$ 已经和 $A$ 同在 $x$ 轴上运动.
\end{proof}
% ----------------------------------------------------------------------------------------
%	BIBLIOGRAPHY
% ----------------------------------------------------------------------------------------

\bibliographystyle{unsrt}

\bibliography{sample}

% ----------------------------------------------------------------------------------------
\end{document}
