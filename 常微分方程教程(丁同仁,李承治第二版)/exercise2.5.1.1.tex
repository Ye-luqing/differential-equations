\documentclass[a4paper, 12pt]{article} % Font size (can be 10pt, 11pt or 12pt) and paper size (remove a4paper for US letter paper)
\usepackage{amsmath,amsfonts,bm}
\usepackage{hyperref,verbatim}
\usepackage{amsthm,epigraph} 
\usepackage{amssymb}
\usepackage{framed,mdframed}
\usepackage{graphicx,color} 
\usepackage{mathrsfs,xcolor} 
\usepackage[all]{xy}
\usepackage{fancybox} 
\usepackage{xeCJK}
\newtheorem*{adtheorem}{定理}
\setCJKmainfont[BoldFont=FZYaoTi,ItalicFont=FZYaoTi]{FZYaoTi}
\definecolor{shadecolor}{rgb}{1.0,0.9,0.9} %背景色为浅红色
\newenvironment{theorem}
{\bigskip\begin{mdframed}[backgroundcolor=gray!40,rightline=false,leftline=false,topline=false,bottomline=false]\begin{adtheorem}}
    {\end{adtheorem}\end{mdframed}\bigskip}
\newtheorem*{bdtheorem}{定义}
\newenvironment{definition}
{\bigskip\begin{mdframed}[backgroundcolor=gray!40,rightline=false,leftline=false,topline=false,bottomline=false]\begin{bdtheorem}}
    {\end{bdtheorem}\end{mdframed}\bigskip}
\newtheorem*{cdtheorem}{习题}
\newenvironment{exercise}
{\bigskip\begin{mdframed}[backgroundcolor=gray!40,rightline=false,leftline=false,topline=false,bottomline=false]\begin{cdtheorem}}
    {\end{cdtheorem}\end{mdframed}\bigskip}
\newtheorem*{ddtheorem}{注}
\newenvironment{remark}
{\bigskip\begin{mdframed}[backgroundcolor=gray!40,rightline=false,leftline=false,topline=false,bottomline=false]\begin{ddtheorem}}
    {\end{ddtheorem}\end{mdframed}\bigskip}
\newtheorem*{edtheorem}{引理}
\newenvironment{lemma}
{\bigskip\begin{mdframed}[backgroundcolor=gray!40,rightline=false,leftline=false,topline=false,bottomline=false]\begin{edtheorem}}
    {\end{edtheorem}\end{mdframed}\bigskip}
\newtheorem*{pdtheorem}{例}
\newenvironment{example}
{\bigskip\begin{mdframed}[backgroundcolor=gray!40,rightline=false,leftline=false,topline=false,bottomline=false]\begin{pdtheorem}}
    {\end{pdtheorem}\end{mdframed}\bigskip}

\usepackage[protrusion=true,expansion=true]{microtype} % Better typography
\usepackage{wrapfig} % Allows in-line images
\usepackage{mathpazo} % Use the Palatino font
\usepackage[T1]{fontenc} % Required for accented characters
\linespread{1.05} % Change line spacing here, Palatino benefits from a slight increase by default

\makeatletter
\renewcommand\@biblabel[1]{\textbf{#1.}} % Change the square brackets for each bibliography item from '[1]' to '1.'
\renewcommand{\@listI}{\itemsep=0pt} % Reduce the space between items in the itemize and enumerate environments and the bibliography

\renewcommand{\maketitle}{ % Customize the title - do not edit title
  % and author name here, see the TITLE block
  % below
  \renewcommand\refname{参考文献}
  \newcommand{\D}{\displaystyle}\newcommand{\ri}{\Rightarrow}
  \newcommand{\ds}{\displaystyle} \renewcommand{\ni}{\noindent}
  \newcommand{\pa}{\partial} \newcommand{\Om}{\Omega}
  \newcommand{\om}{\omega} \newcommand{\sik}{\sum_{i=1}^k}
  \newcommand{\vov}{\Vert\omega\Vert} \newcommand{\Umy}{U_{\mu_i,y^i}}
  \newcommand{\lamns}{\lambda_n^{^{\scriptstyle\sigma}}}
  \newcommand{\chiomn}{\chi_{_{\Omega_n}}}
  \newcommand{\ullim}{\underline{\lim}} \newcommand{\bsy}{\boldsymbol}
  \newcommand{\mvb}{\mathversion{bold}} \newcommand{\la}{\lambda}
  \newcommand{\La}{\Lambda} \newcommand{\va}{\varepsilon}
  \newcommand{\be}{\beta} \newcommand{\al}{\alpha}
  \newcommand{\dis}{\displaystyle} \newcommand{\R}{{\mathbb R}}
  \newcommand{\N}{{\mathbb N}} \newcommand{\cF}{{\mathcal F}}
  \newcommand{\gB}{{\mathfrak B}} \newcommand{\eps}{\epsilon}
  \begin{flushright} % Right align
    {\LARGE\@title} % Increase the font size of the title
    
    \vspace{50pt} % Some vertical space between the title and author name
    
    {\large\@author} % Author name
    \\\@date % Date
    
    \vspace{40pt} % Some vertical space between the author block and abstract
  \end{flushright}
}

% ----------------------------------------------------------------------------------------
%	TITLE
% ----------------------------------------------------------------------------------------
\begin{document}
\title{\textbf{习题2.5.1.1}} 
% \setlength\epigraphwidth{0.7\linewidth}
\author{\small{叶卢庆}\\{\small{杭州师范大学理学院,学号:1002011005}}\\{\small{Email:h5411167@gmail.com}}} % Institution
\renewcommand{\today}{\number\year. \number\month. \number\day}
\date{\today} % Date

% ----------------------------------------------------------------------------------------


\maketitle % Print the title section

% ----------------------------------------------------------------------------------------
%	ABSTRACT AND KEYWORDS
% ----------------------------------------------------------------------------------------

% \renewcommand{\abstractname}{摘要} % Uncomment to change the name of the abstract to something else

% \begin{abstract}

% \end{abstract}

% \hspace*{3,6mm}\textit{关键词:}  % Keywords

% \vspace{30pt} % Some vertical space between the abstract and first section

% ----------------------------------------------------------------------------------------
%	ESSAY BODY
% ----------------------------------------------------------------------------------------
\begin{exercise}[2.5.1.1]
求解下来微分方程:
$$
(3x^2y+2xy+y^3)dx+(x^2+y^2)dy=0.
$$
\end{exercise}
\begin{proof}[解]
微分方程两边同时乘以非零函数 $u(x,y)$,得到
\begin{equation}
  \label{eq:1}
  u(x,y)(3x^2y+2xy+y^3)dx+u(x,y)(x^2+y^2)dy=0.
\end{equation}
我们希望 \eqref{eq:1} 是恰当的,即
\begin{equation}
  \label{eq:2}
  \frac{\pa [u(x,y)(3x^2y+2xy+y^3)]}{\pa y}=\frac{\pa
    [u(x,y)(x^2+y^2)]}{\pa x}.
\end{equation}
也即,
\begin{align*}
&\frac{\pa u(x,y)}{\pa
  y}(3x^2y+2xy+y^3)+u(x,y)(3x^2+3y^2)\\&=\frac{\pa u(x,y)}{\pa x}(x^2+y^2).
\end{align*}
当 $x^2+y^2\neq 0$ 时,也即
$$
\frac{\pa u(x,y)}{\pa y}y(1+\frac{2x^2+2x}{x^2+y^2})+3u(x,y)=\frac{\pa
u(x,y)}{\pa x}.
$$
让 $u(x,y)$ 是只关于 $x$ 的函数,则我们得到
$$
3u(x,y)=\frac{\pa u(x,y)}{\pa x},
$$
不妨让 $u(x,y)=e^{3x}$.因此我们得到恰当微分方程
$$
e^{3x}(3x^2y+2xy+y^3)dx+e^{3x}(x^2+y^2)dy=0.
$$
设存在二元函数 $\phi(x,y)$,使得
\begin{equation}
  \label{eq:3}
  \frac{\pa\phi}{\pa y}=x^2e^{3x}+y^2e^{3x}\ri \phi=x^2e^{3x}y+\frac{1}{3}e^{3x}y^3+f(x).
\end{equation}
因此
\begin{equation}
  \label{eq:4}
  f'(x)=0\ri f(x)=C.
\end{equation}
于是得到通积分
\begin{equation}
  \label{eq:5}
  \phi\equiv x^2e^{3x}y+\frac{1}{3}e^{3x}y^3+C=0.
\end{equation}
\begin{comment}
我们发现做不下去了.因此我们考虑用分组法,怎么分组是个诀窍.我们发现
\begin{align*}
  (3x^2y+2xy+y^3)dx&=(2x^2y+2xy+x^2y+y^3)dx\\&=2xy(x+1)dx+y(x^2+y^2)dx.
\end{align*}
因此我们得到微分方程
\begin{equation}
  \label{eq:3}
  2xy(x+1)dx+y(x^2+y^2)dx+(x^2+y^2)dy=0.
\end{equation}
对于微分方程
\begin{equation}
  \label{eq:p}
  y(x^2+y^2)dx+(x^2+y^2)dy=0
\end{equation}
来说,当 $x^2+y^2\neq 0$ 且 $y\neq 0$ 时,易得积分因子为
$\frac{1}{y(x^2+y^2)}$,将积分因子乘以方程两侧,我们得到
$$
dx+\frac{1}{y}dy=0.
$$
因此可得通积分为
$$
\phi\equiv x+\ln |y|+C=0.
$$
易得 $\frac{1}{y(x^2+y^2)}g(x+\ln |y|)$ 也是微分方程 \eqref{eq:p} 的一
个积分因子,其中 $g$ 是可微函数.\\


对于微分方程
\begin{equation}
  \label{eq:q}
  2xy(x+1)dx=0
\end{equation}
来说,当 $y\neq 0$ 时,易得积分因子为 $\frac{1}{y}$.可得通
积分为
$$
\xi\equiv \frac{2}{3}x^{3}+x^2+C=0
$$
易得 $\frac{1}{y}h(\frac{2}{3}x^3+x^2)$ 也是微分方程 \eqref{eq:q} 的一个积分因子,
其中 $h$ 是可微函数.\\

我们希望,
\begin{equation}
  \label{eq:6}
  \frac{1}{y(x^2+y^2)}g(x+\ln |y|)=\frac{1}{y}h(\frac{2}{3}x^3+x^2).
\end{equation}
然后,我们又做不下去了!\\


我们从头来过.对于微分方程
$$
(3x^2y+2xy+y^3)dx+(x^2+y^2)dy=0.
$$
当 $y(x^2+y^{2})\neq 0$ 时,将其两边同除以 $y(x^{2}+y^{2})$,可得
$$
\frac{3x^2+2x+y^2}{x^2+y^2}dx+\frac{1}{y}dy=0.
$$
也即
$$
(x^{2}+y^{2})(dx+\frac{1}{y}dy)+(2x^{2}+2x)dx=0.
$$
我们先来看微分方程
$$
(x^2+y^2)dx+\frac{x^2+y^2}{y}dy=0.
$$
显然,其积分因子为 $\frac{1}{x^2+y^2}$.
\end{comment}
\end{proof}
% ----------------------------------------------------------------------------------------
%	BIBLIOGRAPHY
% ----------------------------------------------------------------------------------------

\bibliographystyle{unsrt}

\bibliography{sample}

% ----------------------------------------------------------------------------------------
\end{document}