\documentclass[a4paper]{article}
\usepackage{amsmath,amsfonts,amsthm,amssymb}
\usepackage{bm}
\usepackage{hyperref}
\usepackage{geometry}
\usepackage{yhmath}
\usepackage{pstricks-add}
\usepackage{framed,mdframed}
\usepackage{graphicx,color} 
\usepackage{mathrsfs,xcolor} 
\usepackage[all]{xy}
\usepackage{fancybox} 
\usepackage{xeCJK}
\newtheorem*{theo}{定理}
\newtheorem*{exe}{题目}
\newtheorem*{rem}{评论}
\newtheorem*{defi}{定义}
\newmdtheoremenv{lemma}{引理}
\newmdtheoremenv{corollary}{推论}
\newtheorem*{exa}{例}
\newenvironment{theorem}
{\bigskip\begin{mdframed}\begin{theo}}
    {\end{theo}\end{mdframed}\bigskip}
\newenvironment{definition}
{\bigskip\begin{mdframed}\begin{defi}}
    {\end{defi}\end{mdframed}\bigskip}
\newenvironment{exercise}
{\bigskip\begin{mdframed}\begin{exe}}
    {\end{exe}\end{mdframed}\bigskip}
\newenvironment{example}
{\bigskip\begin{mdframed}\begin{exa}}
    {\end{exa}\end{mdframed}\bigskip}
\geometry{left=2.5cm,right=2.5cm,top=2.5cm,bottom=2.5cm}
\setCJKmainfont[BoldFont=SimHei]{SimSun}
\renewcommand{\today}{\number\year 年 \number\month 月 \number\day 日}
\newcommand{\D}{\displaystyle}\newcommand{\ri}{\Rightarrow}
\newcommand{\ds}{\displaystyle} \renewcommand{\ni}{\noindent}
\newcommand{\ov}{\overrightarrow}
\newcommand{\pa}{\partial} \newcommand{\Om}{\Omega}
\newcommand{\om}{\omega} \newcommand{\sik}{\sum_{i=1}^k}
\newcommand{\vov}{\Vert\omega\Vert} \newcommand{\Umy}{U_{\mu_i,y^i}}
\newcommand{\lamns}{\lambda_n^{^{\scriptstyle\sigma}}}
\newcommand{\chiomn}{\chi_{_{\Omega_n}}}
\newcommand{\ullim}{\underline{\lim}} \newcommand{\bsy}{\boldsymbol}
\newcommand{\mvb}{\mathversion{bold}} \newcommand{\la}{\lambda}
\newcommand{\La}{\Lambda} \newcommand{\va}{\varepsilon}
\newcommand{\be}{\beta} \newcommand{\al}{\alpha}
\newcommand{\dis}{\displaystyle} \newcommand{\R}{{\mathbb R}}
\newcommand{\N}{{\mathbb N}} \newcommand{\cF}{{\mathcal F}}
\newcommand{\gB}{{\mathfrak B}} \newcommand{\eps}{\epsilon}
\renewcommand\refname{参考文献}\renewcommand\figurename{图}
\usepackage[]{caption2} 
\renewcommand{\captionlabeldelim}{}
\setlength\parindent{0pt}
\begin{document}
\title{\huge{\bf{$n$阶常微分方程通解中常数独立的意义}}} \author{\small{叶卢庆\footnote{叶卢庆(1992---),男,杭州师范大学理学院数学与应用数学专业本科在读,E-mail:yeluqingmathematics@gmail.com}}}
\maketitle
丁同仁,李承治编《常微分方程教程》第二版的定义1.3给出了 $ n$ 阶常微分方 程

\begin{equation} {\displaystyle F(x,y,y',\cdots,y^{(n)})=0}\end{equation}
的通解的定义:
\begin{definition}
如果 $ y=\phi(x,C_1,C_2,\cdots,C_n)$ 是方程 1的解,且常 数 $ C_1,C_2,\cdots,C_n$ 是独立的,那么 称$ y=\phi(x,C_1,C_2,\cdots,C_n)$ 是方程 1 的通 解.所谓$ C_1,C_2,\cdots,C_n$ 独立,其含义是 Jacobi 行列式 
\begin{equation} {\displaystyle \begin{vmatrix} \frac{\partial \phi}{\partial C_1}&\frac{\partial \phi}{\partial C_2}&\cdots&\frac{\partial\phi}{\partial C_n}\\ \frac{\partial \phi'}{\partial C_1}&\frac{\partial \phi'}{\partial C_2}&\cdots&\frac{\partial \phi'}{\partial C_n}\\ \vdots&\vdots& &\vdots\\ \frac{\partial \phi^{(n-1)}}{\partial C_1}&\frac{\partial \phi^{(n-1)}}{\partial C_2}&\cdots&\frac{\partial \phi^{(n-1)}}{\partial C_n}\\ \end{vmatrix}\neq 0. }\end{equation}

 其中 
\begin{equation} {\displaystyle \begin{cases} \phi=\phi(x,C_1,\cdots,C_n),\\ \phi^{(1)}=\phi^{(1)}(x,C_1,\cdots,C_n),\\ \phi^{(2)}=\phi^{(2)}(x,C_1,\cdots,C_n),\\ \vdots\\ \phi^{(n-1)}=\phi^{(n-1)}(x,C_1,\cdots,C_n). \end{cases}}\end{equation}
\end{definition}

有些人可能会看不懂,书上 为什么用这么晦涩的方式来定义$ C_1,C_2,\cdots,C_n$ 的独立性?这到底是什么 意思?下面我利用反函数定理来 解释.
对于微分方程 (1),我们给出初值条件:

$$ {\displaystyle y(x_0)=y_0,y'(x_0)=y_1,\cdots,y^{(n-1)}(x_0)=y_{n-1}, }$$

把这些初值条件代入 (3) 时,得到 

\begin{equation} {\displaystyle \begin{cases} y_0=\phi(x_0,C_1,\cdots,C_n),\\ y_1=\phi^{(1)}(x_0,C_1,\cdots,C_n),\\ \vdots\\ y_{n-1}=\phi^{(n-1)}(x_0,C_1,\cdots,C_n) \end{cases} }\end{equation}

 由于行列式 (2) 不为0,因此根据多元反函数定理,可得方程组 (4) 中的$ C_1,\cdots,C_n$ 能被解出,也即,$ C_1,\cdots,C_n$ 能分别被表达成 $ y_0,\cdots,y_{n-1},x_0$的关系式.这就是常数 $ C_1,\cdots,C_n$ 独立的意义.
\end{document}
