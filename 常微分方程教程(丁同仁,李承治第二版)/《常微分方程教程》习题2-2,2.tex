\documentclass[a4paper, 11pt]{article} % Font size (can be 10pt, 11pt or 12pt) and paper size (remove a4paper for US letter paper)
\usepackage{amsmath,amsfonts,bm}
\usepackage{hyperref}
\usepackage{amsthm} 
\usepackage{amssymb}
\usepackage{framed,mdframed}
\usepackage{graphicx,color} 
\usepackage{mathrsfs,xcolor} 
\usepackage[all]{xy}
\usepackage{fancybox} 
\usepackage{xeCJK}
\newtheorem{adtheorem}{定理}
\setCJKmainfont[BoldFont=FZYaoTi,ItalicFont=FZYaoTi]{FZYaoTi}
\definecolor{shadecolor}{rgb}{1.0,0.9,0.9} %背景色为浅红色
\newenvironment{theorem}
{\begin{mdframed}[backgroundcolor=gray!40,rightline=false,leftline=false,topline=false,bottomline=false]\begin{adtheorem}}
    {\end{adtheorem}\end{mdframed}}
\newtheorem*{bdtheorem}{定义}
\newenvironment{definition}
{\begin{mdframed}[backgroundcolor=gray!40,rightline=false,leftline=false,topline=false,bottomline=false]\begin{bdtheorem}}
    {\end{bdtheorem}\end{mdframed}}
\newtheorem*{cdtheorem}{习题}
\newenvironment{exercise}
{\begin{mdframed}[backgroundcolor=gray!40,rightline=false,leftline=false,topline=false,bottomline=false]\begin{cdtheorem}}
    {\end{cdtheorem}\end{mdframed}}
\newtheorem{ddtheorem}{注}
\newenvironment{remark}
{\begin{mdframed}[backgroundcolor=gray!40,rightline=false,leftline=false,topline=false,bottomline=false]\begin{ddtheorem}}
    {\end{ddtheorem}\end{mdframed}}
\newtheorem{edtheorem}{引理}
\newenvironment{lemma}
{\begin{mdframed}[backgroundcolor=gray!40,rightline=false,leftline=false,topline=false,bottomline=false]\begin{edtheorem}}
    {\end{edtheorem}\end{mdframed}}
\usepackage[protrusion=true,expansion=true]{microtype} % Better typography
\usepackage{wrapfig} % Allows in-line images
\usepackage{mathpazo} % Use the Palatino font
\usepackage[T1]{fontenc} % Required for accented characters
\linespread{1.05} % Change line spacing here, Palatino benefits from a slight increase by default

\makeatletter
\renewcommand\@biblabel[1]{\textbf{#1.}} % Change the square brackets for each bibliography item from '[1]' to '1.'
\renewcommand{\@listI}{\itemsep=0pt} % Reduce the space between items in the itemize and enumerate environments and the bibliography

\renewcommand{\maketitle}{ % Customize the title - do not edit title
  % and author name here, see the TITLE block
  % below
  \renewcommand\refname{参考文献}
  \newcommand{\D}{\displaystyle}\newcommand{\ri}{\Rightarrow}
  \newcommand{\ds}{\displaystyle} \renewcommand{\ni}{\noindent}
  \newcommand{\pa}{\partial} \newcommand{\Om}{\Omega}
  \newcommand{\om}{\omega} \newcommand{\sik}{\sum_{i=1}^k}
  \newcommand{\vov}{\Vert\omega\Vert} \newcommand{\Umy}{U_{\mu_i,y^i}}
  \newcommand{\lamns}{\lambda_n^{^{\scriptstyle\sigma}}}
  \newcommand{\chiomn}{\chi_{_{\Omega_n}}}
  \newcommand{\ullim}{\underline{\lim}} \newcommand{\bsy}{\boldsymbol}
  \newcommand{\mvb}{\mathversion{bold}} \newcommand{\la}{\lambda}
  \newcommand{\La}{\Lambda} \newcommand{\va}{\varepsilon}
  \newcommand{\be}{\beta} \newcommand{\al}{\alpha}
  \newcommand{\dis}{\displaystyle} \newcommand{\R}{{\mathbb R}}
  \newcommand{\N}{{\mathbb N}} \newcommand{\cF}{{\mathcal F}}
  \newcommand{\gB}{{\mathfrak B}} \newcommand{\eps}{\epsilon}
  \begin{flushright} % Right align
    {\LARGE\@title} % Increase the font size of the title
    
    \vspace{50pt} % Some vertical space between the title and author name
    
    {\large\@author} % Author name
    \\\@date % Date
    
    \vspace{40pt} % Some vertical space between the author block and abstract
  \end{flushright}
}

% ----------------------------------------------------------------------------------------
%	TITLE
% ----------------------------------------------------------------------------------------

\title{\textbf{《常微分方程教程》\footnote{丁同仁,李承治编著,高等教育出版社第二版.}习题2-2,2}} 

\author{\small{叶卢庆}\\{\small{杭州师范大学理学院,学号:1002011005}}\\{\small{Email:h5411167@gmail.com}}} % Institution
\renewcommand{\today}{\number\year. \number\month. \number\day}
\date{\today} % Date

% ----------------------------------------------------------------------------------------

\begin{document}
  \maketitle % Print the title section
  
  % ----------------------------------------------------------------------------------------
  %	ABSTRACT AND KEYWORDS
  % ----------------------------------------------------------------------------------------
  
  % \renewcommand{\abstractname}{摘要} % Uncomment to change the name of the abstract to something else
  
  % \begin{abstract}
  
  % \end{abstract}
  
  % \hspace*{3,6mm}\textit{关键词:}  % Keywords
  
  % \vspace{30pt} % Some vertical space between the abstract and first section
  
  % ----------------------------------------------------------------------------------------
  %	ESSAY BODY
  % ----------------------------------------------------------------------------------------
求解下来微分方程的初值问题.
\begin{exercise}[2-2,2,(1)]
$$
\sin 2xdx+\cos 3ydy=0,y(\frac{\pi}{2})=\frac{\pi}{3}.
$$
\end{exercise}
\begin{proof}[解]
我们先求通解.显然题目中的微分方程是恰当微分方程.设
\begin{equation}
  \label{eq:12.35am}
  \frac{\pa\phi}{\pa x}=\sin 2x,
\end{equation}
得到
\begin{equation}
  \label{eq:12.44}
  \phi=-\frac{1}{2}\cos 2x+f(y).
\end{equation}
代入下式
\begin{equation}
  \label{eq:12.43}
  \frac{\pa\phi}{\pa y}=\cos 3y.
\end{equation}
可得
\begin{equation}
  \label{eq:1.00am}
  f'(y)=\cos 3y\ri  f(y)=\frac{1}{3}\sin 3y+C.
\end{equation}
于是通积分为
\begin{equation}
  \label{eq:1.07am}
  -\frac{1}{2}\cos2x+\frac{1}{3}\sin3y+C=0.
\end{equation}
将初值条件代入,可得
$$
\frac{1}{2}+C=0\ri C=-\frac{1}{2}.
$$
于是可得微分方程的积分为
$$
-\frac{1}{2}\cos 2x+\frac{1}{3}\sin3y-\frac{1}{2}=0.
$$
\end{proof}
\begin{exercise}[2-2,2,(2)]
$$
xdx+ye^{-x}dy=0,y(0)=1.
$$  
\end{exercise}
\begin{proof}[解]
显然,题目中不是恰当方程.那我们该怎么办呢?我们另寻出路.我们发现,
\begin{equation}\label{eq:5.59}
e^xxdx+ydy=0.
\end{equation}
是一个恰当方程,我们在下面粉色的阴影部分证明方程 \eqref{eq:5.59} 和题目中的方程等价.
\begin{shaded}
我们来看恰当微分方程
\begin{equation}
  \label{eq:8.35}
P(x,y)dx+Q(x,y)dy=0.  
\end{equation}
在任意一个点 $(x_0,y_0)$ 处,函数 $P,Q$ 的取值分别为
$P(x_0,y_0)$,$Q(x_0,y_0)$.不妨设 $Q(x_0,y_0)\neq 0$.根据隐函数定理,可
知在点 $(x_0,y_0)$ 的某个邻域内,$y$ 是 $x$ 的可微函数 $g(x)=y$.然后点
$(x_0,y_0)$ 中横坐标变量改变一个微小的量 $dx$ ,纵坐标变量相应改变 $dy=g'(x_0)dx$,于是点 $(x_0,y_0)$ 变成了点
$(x_0+dx,y_0+dy)$.方程 \eqref{eq:8.35} 告诉了我们 $d
x$ 和 $d y$ 的关系:
\begin{equation}
  \label{eq:9.07}
  P(x_0,y_0)dx+Q(x_0,y_0)dy=0.
\end{equation}
方程 \eqref{eq:8.35} 两边同时乘以一个非零函数 $U(x,y)$,就可能得到一个
非恰当微分方程
\begin{equation}
  \label{eq:3.17pm}
  U(x,y)P(x,y)dx+U(x,y)Q(x,y)dy=0.
\end{equation}
对于如上的非恰当微分方程来说,显然在 $(x_0,y_0)$ 的某个邻域内,$y$ 和 $x$ 仍拥有相同的函数关系 $y=g(x)$.
\end{shaded}
于是$$
\frac{\pa\phi}{\pa y}=y\ri \phi=\frac{1}{2}y^2+f(x)\ri f'(x)=xe^x\ri f(x)=xe^x-e^x+C.
$$
于是通积分为
$$
\frac{1}{2}y^2+xe^x-e^x+C=0
$$
将初始条件 $x=0,y=1$ 代入通积分,解得
$$
C=\frac{1}{2}.
$$
\end{proof}
\begin{exercise}[2-2,2,(3)]
$$
\frac{dr}{d\theta}=r,r(0)=2.
$$
\end{exercise}
\begin{proof}[解]
显然,
$$
r=2e^{\theta}.
$$
\end{proof}
\begin{exercise}[2-2,2,(4)]
$$
\frac{dy}{dx}=\frac{\ln |x|}{1+y^2},y(1)=0.
$$  
\end{exercise}
\begin{proof}[解]
可得
$$
(1+y^2)dy-\ln |x|dx=0.
$$
这是一个恰当微分方程.可得
\begin{align*}
\frac{\pa\phi}{\pa y}&=1+y^2\ri
\phi\\&=\frac{1}{3}y^3+\frac{1}{2}y^2+f(x)\ri f'(x)=-\ln |x|\ri f(x)=-x\ln |x|+x+C.
\end{align*}
于是通积分为
$$
\frac{1}{3}y^3+\frac{1}{2}y^2-x\ln |x|+x+C=0.
$$
将 $x=1,y=0$ 代入,解得 $C=-1$.
\end{proof}
\begin{exercise}[2-2,2,(5)]
$$
\sqrt{1+x^2}\frac{dy}{dx}=xy^3,y(0)=1.
$$  
\end{exercise}
\begin{proof}[解]
当 $y\neq 0$ 时,可得
$$
\frac{1}{y^3}dy-\frac{x}{\sqrt{1+x^2}}dx=0.
$$
这是一个恰当微分方程.于是,
\begin{align*}
\frac{\pa\phi}{\pa y}&=\frac{1}{y^3}\ri
\phi=\frac{-1}{2}y^{-2}+f(x)\ri f'(x)=\frac{-x}{\sqrt{1+x^2}}\ri f(x)=-\sqrt{1+x^{2}}+C.
\end{align*}
于是通积分为
$$
\frac{-1}{2}y^{-2}-(1+x^2)^{\frac{1}{2}}+C=0.
$$
将 $x=0,y=1$ 代入,解得 $C=\frac{3}{2}$.\\

当 $y=0$ 时,易得无解.
\end{proof}
  % ----------------------------------------------------------------------------------------
  %	BIBLIOGRAPHY
  % ----------------------------------------------------------------------------------------
  
  \bibliographystyle{unsrt}
  \bibliography{sample}
  
  % ----------------------------------------------------------------------------------------
\end{document}