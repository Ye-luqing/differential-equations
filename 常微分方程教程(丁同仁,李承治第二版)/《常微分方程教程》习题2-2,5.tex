\documentclass[a4paper, 12pt]{article} % Font size (can be 10pt, 11pt or 12pt) and paper size (remove a4paper for US letter paper)
\usepackage{amsmath,amsfonts,bm}
\usepackage{hyperref,epigraph}
\usepackage{amsthm,verbatim}
\usepackage{amssymb}
\usepackage{framed,mdframed}
\usepackage{graphicx,color}
\usepackage{mathrsfs,xcolor}
\usepackage[all]{xy}
\usepackage{fancybox}
\usepackage{xeCJK}
\newtheorem*{adtheorem}{定理}
\setCJKmainfont[BoldFont=Adobe Song Std,ItalicFont=Adobe Song Std]{Adobe Song Std}
\definecolor{shadecolor}{rgb}{1.0,0.9,0.9} %背景色为浅红色
\newenvironment{theorem}
{\bigskip\begin{mdframed}[backgroundcolor=gray!40,rightline=false,leftline=false,topline=false,bottomline=false]\begin{adtheorem}}
    {\end{adtheorem}\end{mdframed}\bigskip}
\newtheorem*{bdtheorem}{定义}
\newenvironment{definition}
{\bigskip\begin{mdframed}[backgroundcolor=gray!40,rightline=false,leftline=false,topline=false,bottomline=false]\begin{bdtheorem}}
    {\end{bdtheorem}\end{mdframed}\bigskip}
\newtheorem*{cdtheorem}{习题}
\newenvironment{exercise}
{\bigskip\begin{mdframed}[backgroundcolor=gray!40,rightline=false,leftline=false,topline=false,bottomline=false]\begin{cdtheorem}}
    {\end{cdtheorem}\end{mdframed}\bigskip}
\newtheorem{ddtheorem}{注}
\newenvironment{remark}
{\bigskip\begin{mdframed}[backgroundcolor=gray!40,rightline=false,leftline=false,topline=false,bottomline=false]\begin{ddtheorem}}
    {\end{ddtheorem}\end{mdframed}\bigskip}
\newtheorem{edtheorem}{引理}
\newenvironment{lemma}
{\bigskip\begin{mdframed}[backgroundcolor=gray!40,rightline=false,leftline=false,topline=false,bottomline=false]\begin{edtheorem}}
    {\end{edtheorem}\end{mdframed}\bigskip}
\usepackage[protrusion=true,expansion=true]{microtype} % Better typography
\usepackage{wrapfig} % Allows in-line images
\usepackage{mathpazo} % Use the Palatino font
\usepackage[T1]{fontenc} % Required for accented characters
\linespread{1.05} % Change line spacing here, Palatino benefits from a slight increase by default
\makeatletter
\renewcommand\@biblabel[1]{\textbf{#1.}} % Change the square brackets for each bibliography item from '[1]' to '1.'
\renewcommand{\@listI}{\itemsep=0pt} % Reduce the space between items in the itemize and enumerate environments and the bibliography

\renewcommand{\maketitle}{ % Customize the title - do not edit title
  % and author name here, see the TITLE block
  % below
  \renewcommand\refname{参考文献}
  \newcommand{\D}{\displaystyle}\newcommand{\ri}{\Rightarrow}
  \newcommand{\ds}{\displaystyle} \renewcommand{\ni}{\noindent}
  \newcommand{\pa}{\partial} \newcommand{\Om}{\Omega}
  \newcommand{\om}{\omega} \newcommand{\sik}{\sum_{i=1}^k}
  \newcommand{\vov}{\Vert\omega\Vert} \newcommand{\Umy}{U_{\mu_i,y^i}}
  \newcommand{\lamns}{\lambda_n^{^{\scriptstyle\sigma}}}
  \newcommand{\chiomn}{\chi_{_{\Omega_n}}}
  \newcommand{\ullim}{\underline{\lim}} \newcommand{\bsy}{\boldsymbol}
  \newcommand{\mvb}{\mathversion{bold}} \newcommand{\la}{\lambda}
  \newcommand{\La}{\Lambda} \newcommand{\va}{\varepsilon}
  \newcommand{\be}{\beta} \newcommand{\al}{\alpha}
  \newcommand{\dis}{\displaystyle} \newcommand{\R}{{\mathbb R}}
  \newcommand{\N}{{\mathbb N}} \newcommand{\cF}{{\mathcal F}}
  \newcommand{\gB}{{\mathfrak B}} \newcommand{\eps}{\epsilon}
  \begin{flushright} % Right align
    {\LARGE\@title} % Increase the font size of the title

    \vspace{50pt} % Some vertical space between the title and author name

    {\large\@author} % Author name
    \\\@date % Date

    \vspace{40pt} % Some vertical space between the author block and abstract
  \end{flushright}
}
\setlength\epigraphwidth{0.6\linewidth}
\begin{document}
\title{\textbf{《常微分方程教程》\footnote{丁同仁,李承治编著,高等教育
      出版社第二版.}习题2-2,5}}

\author{\small{叶卢庆}\\{\small{杭州师范大学理学院,学
      号:1002011005}}\\{\small{Email:h5411167@gmail.com}}} % Institution
\renewcommand{\today}{\number\year. \number\month. \number\day}
\date{\today} % Date

\maketitle
\begin{exercise}[2-2,5]
  设微分方程
  \begin{equation}
    \label{eq:3.41}
    \frac{dy}{dx}=f(y),
  \end{equation}
  其中 $f(y)$ 在 $y=a$ 的某邻域(例如区间 $|y-a|\leq\varepsilon$)内连
  续,且 $f(y)=0$ 当且仅当 $y=a$.则在直线 $y=a$ 上的每一
  点,方程 \eqref{eq:3.41} 的解是局部唯一的,当且仅当瑕积分
$$
|\int_a^{a\pm \varepsilon}\frac{dy}{f(y)}|=\infty.
$$
\end{exercise}
\begin{remark}
  首先我得抱怨一下,书上在给我做这个题目之前并没有介绍“局部唯一”是什么意
  思.我将其理解成,对于直线 $y=a$ 上的任意一个点 $(x_0,a)$,当 $(x_0,a)$
  的邻域足够小的时候,满足方程 \eqref{eq:3.41} 的所有积分曲线中,仅有一条
  积分曲线通过了 $(x_0,a)$ 的该邻域.另外,在本文里,笔者规定所有的邻域都
  是圆形的.
\end{remark}

\begin{remark}
  该题目有明显的物理意义.将 $x$ 看作时间,将 $y$ 看作质点在一条直线上的
  一维位置.则$\frac{dy}{dx}=f(y)$ 表示质点在直线上的一维速度.质点的速度
  在位置区间$[a-\varepsilon,a+\varepsilon]$ 上连续(这在物理中是自明
  的,速度不会突变.),且只有在位置 $y=a$ 的地方速度才为0.且质点的速度是位
  置的函
  数.我们知道,满足如上条件的质点可能有无数种运动方式.\\

  如果质点所有的运动方式都有如下特征:当质点还没达到 $y=a$ 这个位置,但是
  已经很接近$y=a$ 这个位置的时候,如果质点每接近 $y=a$ 这个位置相同的距
  离,所花费的时间会越来越长,最后是寸步难进(虽然始终在运动),即使
  向 $y=a$ 前进哪怕是很小很小的一段距离,所消耗的时间也会惊人地多,以至于
  无论给这个质点多长的时间,质点都将无法到达
  $y=a$ 这个位置,那么想要质点在一定的时间内通过 $y=a$ 的哪怕是很小很小的邻域都是做不到的,唯一能做的就是把质点直接放在 $y=a$ 这个地方保持静止.\\

  而相反地,如果质点无法在一定的时间内通过 $y=a$ 的哪怕是很小很小的邻域,那
  么在别处的质点只有经过无穷长的时间才能达到 $y=a$ 处(也就是永远达不到).
\end{remark}
\begin{proof}[证明]
  当 $y\neq a$ 时,我们把 \eqref{eq:3.41} 化为
  \begin{equation}
    \label{eq:3.50}
    \frac{1}{f(y)}dy-dx=0.
  \end{equation}
  对 \eqref{eq:3.50} 进行积分,得到通积分
  \begin{equation}\label{eq:1.31}
    \int \frac{1}{f(y)}dy-x+C=0.
  \end{equation}
  其中 $C$ 是一个常数.该通积分在 $y\neq a$ 的时候,确定了 $x$ 和 $y$ 的
  函数关系 $x=g(y)$.\\




  \ni$\Leftarrow:$假如存在直线 $y=a$ 上的某个点$(x_0,a)$,微分方程
  在 $(x_0,a)$ 的任意邻域内解不唯一,即除了 $y=a$ 这个解外还有其它解,那
  么有如下几个关键点:
  \begin{itemize}
  \item \label{item:1}瑕积分
    \begin{equation}
      \label{eq:12.58}
      |\int_a^{a\pm \varepsilon}\frac{dy}{f(y)}|=\infty,\mbox{意思就
        是}|\int_a^{a+\va}\frac{dy}{f(y)}|=\infty\mbox{以及}|\int_a^{a-\va}\frac{dy}{f(y)}|=\infty.
    \end{equation}
  \item \label{item:2}
    $\frac{1}{f(y)}$ 在 $(a-\varepsilon,a+\varepsilon)$ 的去心邻域内连
    续,可得对于任意给定的正实数 $\delta\in (0,\va)$,
    \begin{equation}
      \label{eq:1.01}
      |\int_{a-\delta}^{a-\va}\frac{dy}{f(y)}|\mbox{和
      }|\int_{a+\delta}^{a+\va}\frac{dy}{f(y)}|\mbox{都有限.}
    \end{equation}
    将第1个关键点和第2个关键点结合起来,可得如下关键点
  \item\label{item:3} 对于任意给定的正实数 $\delta\in (0,\va)$,
    \begin{equation}
      \label{eq:1.17pm}
      |\int_{a-\delta}^a \frac{dy}{f(y)}|=\infty\mbox{以及}|\int_a^{a+\delta}\frac{dy}{f(y)}|=\infty.
    \end{equation}
  \end{itemize}
  下面我们把 \eqref{eq:1.17pm} 和 \eqref{eq:1.31} 结合起来,推出矛
  盾.由\eqref{eq:1.31} 结合Newton-Leibniz公式可知,当 $(x,y)$ 处于
  点 $(x_0,a)$的足够小的邻域 $V$ 内时,$\frac{dy}{f(y)}$ 在邻域 $V$ 内的
  积分也会足够小(不会超
  过邻域$V$ 的直径),这与 \eqref{eq:1.17pm} 矛盾.\\\\


  \ni$\Rightarrow:$易得满足方程 \eqref{eq:3.41} 的积分曲线是可微的,因此
  连续.当在直线 $y=a$ 的每一点上微分方程 \eqref{eq:3.41} 的解都是局部唯
  一的时候,也就是说,方程\eqref{eq:3.41} 在直线 $y=a$ 的任意一
  点 $(x_0,a)$ 的足够小的邻域 $U$上的唯一解是 $y=a$ 本身,此时,任意给定
  的积分曲线
  \begin{equation}
    \label{eq:12.47}
    \int \frac{1}{f(y)}dy-x+C_{0}=0.
  \end{equation}
  都将避开邻域 $U$.假如
$$
|\int_a^{a\pm \varepsilon}\frac{dy}{f(y)}|\neq\infty,
$$
那么
\begin{equation}
  \label{eq:3.13}
  \int_a^{a\pm\va}\frac{dy}{f(y)}=M,
\end{equation}
其中 $M$ 是一个实数.当然,在这里,我们不考虑
$\int_a^{a\pm\va}\frac{dy}{f(y)}$ 不存在的情形,因为由于
$\frac{1}{f(y)}$ 在 $a$ 的去心邻域上的连续
性,$\int_a^{a\pm\va}\frac{dy}{f(y)}$ 不存在的现象不会发生.\\

式 \eqref{eq:3.13} 结合 \eqref{eq:1.31},可得积分曲线会在点 $(M,a)$ 处
和直线 $y=a$ 相交!矛盾!
\end{proof}
\end{document}
