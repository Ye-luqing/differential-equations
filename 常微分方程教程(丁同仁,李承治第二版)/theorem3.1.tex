\documentclass[a4paper, 12pt]{article} % Font size (can be 10pt, 11pt or 12pt) and paper size (remove a4paper for US letter paper)
\usepackage{amsmath,amsfonts,bm}
\usepackage{hyperref,verbatim}
\usepackage{amsthm,epigraph} 
\usepackage{amssymb}
\usepackage{framed,mdframed}
\usepackage{graphicx,color} 
\usepackage{mathrsfs,xcolor} 
\usepackage[all]{xy}
\usepackage{fancybox} 
% \usepackage{xeCJK}
\usepackage{CJKutf8}
\newtheorem*{adtheorem}{定理}
% \setCJKmainfont[BoldFont=FZYaoTi,ItalicFont=FZYaoTi]{FZYaoTi}
\definecolor{shadecolor}{rgb}{1.0,0.9,0.9} %背景色为浅红色
\newenvironment{theorem}
{\bigskip\begin{mdframed}[backgroundcolor=gray!40,rightline=false,leftline=false,topline=false,bottomline=false]\begin{adtheorem}}
    {\end{adtheorem}\end{mdframed}\bigskip}
\newtheorem*{bdtheorem}{定义}
\newenvironment{definition}
{\bigskip\begin{mdframed}[backgroundcolor=gray!40,rightline=false,leftline=false,topline=false,bottomline=false]\begin{bdtheorem}}
    {\end{bdtheorem}\end{mdframed}\bigskip}
\newtheorem*{cdtheorem}{习题}
\newenvironment{exercise}
{\bigskip\begin{mdframed}[backgroundcolor=gray!40,rightline=false,leftline=false,topline=false,bottomline=false]\begin{cdtheorem}}
    {\end{cdtheorem}\end{mdframed}\bigskip}
\newtheorem*{ddtheorem}{注}
\newenvironment{remark}
{\bigskip\begin{mdframed}[backgroundcolor=gray!40,rightline=false,leftline=false,topline=false,bottomline=false]\begin{ddtheorem}}
    {\end{ddtheorem}\end{mdframed}\bigskip}
\newtheorem*{edtheorem}{引理}
\newenvironment{lemma}
{\bigskip\begin{mdframed}[backgroundcolor=gray!40,rightline=false,leftline=false,topline=false,bottomline=false]\begin{edtheorem}}
    {\end{edtheorem}\end{mdframed}\bigskip}
\newtheorem*{pdtheorem}{例}
\newenvironment{example}
{\bigskip\begin{mdframed}[backgroundcolor=gray!40,rightline=false,leftline=false,topline=false,bottomline=false]\begin{pdtheorem}}
    {\end{pdtheorem}\end{mdframed}\bigskip}

\usepackage[protrusion=true,expansion=true]{microtype} % Better typography
\usepackage{wrapfig} % Allows in-line images
\usepackage{mathpazo} % Use the Palatino font
\usepackage[T1]{fontenc} % Required for accented characters
\linespread{1.05} % Change line spacing here, Palatino benefits from a slight increase by default

\makeatletter
\renewcommand\@biblabel[1]{\textbf{#1.}} % Change the square brackets for each bibliography item from '[1]' to '1.'
\renewcommand{\@listI}{\itemsep=0pt} % Reduce the space between items in the itemize and enumerate environments and the bibliography

\renewcommand{\maketitle}{ % Customize the title - do not edit title
  % and author name here, see the TITLE block
  % below
  \renewcommand\refname{参考文献}
  \newcommand{\D}{\displaystyle}\newcommand{\ri}{\Rightarrow}
  \newcommand{\ds}{\displaystyle} \renewcommand{\ni}{\noindent}
  \newcommand{\pa}{\partial} \newcommand{\Om}{\Omega}
  \newcommand{\om}{\omega} \newcommand{\sik}{\sum_{i=1}^k}
  \newcommand{\vov}{\Vert\omega\Vert} \newcommand{\Umy}{U_{\mu_i,y^i}}
  \newcommand{\lamns}{\lambda_n^{^{\scriptstyle\sigma}}}
  \newcommand{\chiomn}{\chi_{_{\Omega_n}}}
  \newcommand{\ullim}{\underline{\lim}} \newcommand{\bsy}{\boldsymbol}
  \newcommand{\mvb}{\mathversion{bold}} \newcommand{\la}{\lambda}
  \newcommand{\La}{\Lambda} \newcommand{\va}{\varepsilon}
  \newcommand{\be}{\beta} \newcommand{\al}{\alpha}
  \newcommand{\dis}{\displaystyle} \newcommand{\R}{{\mathbb R}}
  \newcommand{\N}{{\mathbb N}} \newcommand{\cF}{{\mathcal F}}
  \newcommand{\gB}{{\mathfrak B}} \newcommand{\eps}{\epsilon}
  \begin{flushright} % Right align
    {\LARGE\@title} % Increase the font size of the title
    
    \vspace{50pt} % Some vertical space between the title and author name
    
    {\large\@author} % Author name
    \\\@date % Date
    
    \vspace{40pt} % Some vertical space between the author block and abstract
  \end{flushright}
}

% ----------------------------------------------------------------------------------------
%	TITLE
% ----------------------------------------------------------------------------------------
\begin{document}
\begin{CJK}{UTF8}{gkai}
  \title{\textbf{定理3.1:皮卡定理(唯一性)}}
  % \setlength\epigraphwidth{0.7\linewidth}
  \author{\small{叶卢庆}\\{\small{杭州师范大学理学院,学
        号:1002011005}}\\{\small{Email:h5411167@gmail.com}}} % Institution
  \renewcommand{\today}{\number\year. \number\month. \number\day}
  \date{\today} % Date
  
  % ----------------------------------------------------------------------------------------
  
  
  \maketitle % Print the title section
  
  % ----------------------------------------------------------------------------------------
  %	ABSTRACT AND KEYWORDS
  % ----------------------------------------------------------------------------------------
  
  % \renewcommand{\abstractname}{摘要} % Uncomment to change the name of the abstract to something else
  
  % \begin{abstract}
  
  % \end{abstract}
  
  % \hspace*{3,6mm}\textit{关键词:} % Keywords
  
  % \vspace{30pt} % Some vertical space between the abstract and first section
  
  % ----------------------------------------------------------------------------------------
  %	ESSAY BODY
  % ----------------------------------------------------------------------------------------
  \begin{theorem}[皮卡定理]
 设初值问题:
$$
(E):\frac{dy}{dx}=f(x,y),y(x_0)=y_0,
$$
其中 $f(x,y)$ 在矩形区域
$$
R:|x-x_0|\leq a,|y-y_0|\leq b
$$
内连续,而且对 $y$ 满足李普希兹条件.则 (E) 在区间 $I=[x_0-h,x_0+h]$ 上有且只
有一个解,其中常数
$$
h=\min\{a,\frac{b}{M}\},M>\max_{(x,y)\in R}|f(x,y)|.
$$
\end{theorem}
\begin{proof}[证明]
首先,笔者要做一个声明,就是笔者不打算过分地关注 $h$ 的具体数值.而且笔者
在此文章中,只是考虑唯一性的证明.至于存在性的证明,笔者在将来考虑.\\

我们考虑该问题的物理意义.物理意义是质点的一维运动.其中 $x$ 是时间,$y$
  是位移.$f(x,y)$ 是速度.质点的速度是连续的.质点在时间 $x_0$ 从点 $y_0$ 出发,初始速度为 $f(x_0,y_0)$.\\

  而李普希兹条件是为了给速度一个控制.质点可能存在多种运动方式,但是李普
  希兹条件保证了,在同一个时刻 $t$,速度不会超过在 $t$ 时刻不同运动方式导致的
  位移差的某个固定倍数.题目就是叫我们证明,李普希兹条件导致了该质点在出发不
  久的时间里运动方式的唯一性.\\\\
\begin{comment}
  我们先来探索一些情形.如果质点速度连续.在时间 $x_0$ 的时候位于位
  置 $y_0$,且此时的速度为$f(x_0,y_0)$,且之后质点受到变力 $F(t)$ 的作用,我
  们知道,经历时间 $t$ 后,质点的速度为
$$
\int_{0}^{t}\frac{F(s)}{m}+f(x_0,y_{0})ds.
$$
其中 $m$ 是质点的质量.因此,经历时间 $t$ 后,质点的位移为
$$
\int_{0}^{t}\left(\int_0^r \frac{F(s)}{m}ds+f(x_0,y_0)r\right)dr.
$$
同理,如果质点受到的力的作用是 $G(t)$,则经过时间 $t$ 后,质点的位移为
$$
\int_{0}^{t}\left(\int_0^r \frac{G(s)}{m}ds+f(x_0,y_0)r\right)dr.
$$
因此两种运动方式在经过时间 $t$ 后导致的位移差为
\begin{align*}
  &\int_{0}^{t}\left(\int_0^r
    \frac{F(s)}{m}ds+f(x_0,y_0)r\right)dr-\int_{0}^{t}\left(\int_0^r
    \frac{G(s)}{m}ds+f(x_0,y_0)r\right)dr\\&=\int_0^t\left(\int_0^{r}\frac{F(s)-G(s)}{m}ds\right)dr.
\end{align*}
如果 $F(t)$ 是恒力 $Q_1$,$G(t)$ 是和 $F(t)$ 不相等的另一个恒力 $Q_2$,下
面我们来看
$$
\frac{\int_0^{t}\frac{F(s)-G(s)}{m}ds}{\int_0^t\left(\int_0^{r}\frac{F(s)-G(s)}{m}ds\right)dr}
$$
上式化简为
$$
\frac{\frac{t}{m}(Q_1-Q_2)}{\frac{1}{2}\frac{t^{2}}{m}(Q_1-Q_2)}
$$
此时,当 $t\to 0$ 时,显然不满足李普希兹条件.\\
\end{comment}
我们先来看下面这种特殊情形.如果质点在时间 $x_0$,位置 $y_0$的速度
为0,即 $f(x_0,y_0)=0$,且 $f(x,y)$ 在矩形区域 $R$ 内连续且对$y$ 满足李
普希兹条件,则质点在时间 $x_0$ 附近足够短的时间段内的唯一运动方式就是静止在 $y_0$ 处.下面
我们来证明这一点.我们用反证法.假如质点在 $x_0$ 附近的任意短的时间段内,除了静止之外,
还有一种运动方式,设依着这种运动方
式,质点的速度随时间的函数为 $h(t)$,则我们知道,$h(x_0)=0$.经过时
间 $t$,质点的位移为
$$
\int_{x_{0}}^{x_0+t}h(s)ds.
$$
由于当 $t\to 0$ 时,
$$
\int_{x_0}^{x_0+t}h(s)ds=o(h(x_0+t)),
$$
因此当 $\int_{x_0}^{x_0+t}h(s)ds\neq 0$ 时,
$$
\lim_{t\to 0}\frac{h(x_0+t)}{\int_{x_{0}}^{x_0+t}h(s)ds}=\infty.
$$
可见,李普希兹条件不满足.因此质点只能是静止这一种运动方式.\\

接下来我们看一般情形.假如质点在时间 $x_0$,位置 $y_0$的速度
为 $f(x_0,y_0)$,其中 $f(x_0,y_0)$ 未必为0,且 $f(x,y)$ 在矩形区域 $R$ 内连续且对$y$ 满足李
普希兹条件,则质点在时间 $x_0$ 附近足够短的时间段内的运动方式是唯一的.我
们采用反证法.假设质点在时间 $x_0$ 附近的任意短的时间段内的运动方式不唯一,则设存在两种运动方式 $A$ 和
$B$.当质点按照运动方式 $A$ 的时候,设速度随时间的函数为 $V_A(t)$.当质点按照运动方式 $B$ 的时候,设速度随时间的函数为 $V_B(t)$.我们知
道,$V_A(x_0)=f(x_0,y_0)$,且经过时间 $t$ 后,质点按照运动方式 $A$ 的位移为
$$
\int_{x_0}^{x_0+t}V_A(s)ds.
$$
按照运动方式 $B$ 的位移为
$$
\int_{x_0}^{x_0+t}V_B(s)ds.
$$
下面我们来看
$$
V_A(t)-V_B(t)
$$
与
$$
\int_{x_0}^{x_0+t}(V_A(s)-V_B(s))ds.
$$
易得当 $t\to 0$ 时,
$$
\int_{x_0}^{x_0+t}(V_A(s)-V_B(s))ds=o(V_A(t)-V_B(t)).
$$
因此,当 $\int_{x_0}^{x_0+t}(V_A(s)-V_B(s))ds\neq 0$ 时,我们有
$$
\lim_{t\to 0}\frac{V_A(t)-V_B(t)}{\int_{x_0}^{x_0+t}(V_A(s)-V_B(s))ds}=\infty.
$$
因此,李普希兹条件不满足,矛盾.
\end{proof}
% ----------------------------------------------------------------------------------------
% BIBLIOGRAPHY
% ----------------------------------------------------------------------------------------
  
\bibliographystyle{unsrt}
  
\bibliography{sample}
  
% ----------------------------------------------------------------------------------------
\end{CJK}
\end{document}