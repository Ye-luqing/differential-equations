\documentclass[a4paper, 12pt]{article} % Font size (can be 10pt, 11pt or 12pt) and paper size (remove a4paper for US letter paper)
\usepackage{amsmath,amsfonts,bm}
\usepackage{hyperref,epigraph}
\usepackage{amsthm} 
\usepackage{amssymb}
\usepackage{framed,mdframed}
\usepackage{graphicx,color} 
\usepackage{mathrsfs,xcolor} 
\usepackage[all]{xy}
\usepackage{fancybox} 
\usepackage{xeCJK}
\newtheorem*{adtheorem}{定理}
\setCJKmainfont[BoldFont=FZYaoTi,ItalicFont=FZYaoTi]{FZYaoTi}
\definecolor{shadecolor}{rgb}{1.0,0.9,0.9} %背景色为浅红色
\newenvironment{theorem}
{\bigskip\begin{mdframed}[backgroundcolor=gray!40,rightline=false,leftline=false,topline=false,bottomline=false]\begin{adtheorem}}
    {\end{adtheorem}\end{mdframed}\bigskip}
\newtheorem*{bdtheorem}{定义}
\newenvironment{definition}
{\bigskip\begin{mdframed}[backgroundcolor=gray!40,rightline=false,leftline=false,topline=false,bottomline=false]\begin{bdtheorem}}
    {\end{bdtheorem}\end{mdframed}\bigskip}
\newtheorem*{cdtheorem}{习题}
\newenvironment{exercise}
{\bigskip\begin{mdframed}[backgroundcolor=gray!40,rightline=false,leftline=false,topline=false,bottomline=false]\begin{cdtheorem}}
    {\end{cdtheorem}\end{mdframed}\bigskip}
\newtheorem*{ddtheorem}{注}
\newenvironment{remark}
{\bigskip\begin{mdframed}[backgroundcolor=gray!40,rightline=false,leftline=false,topline=false,bottomline=false]\begin{ddtheorem}}
    {\end{ddtheorem}\end{mdframed}\bigskip}
\newtheorem*{edtheorem}{引理}
\newenvironment{lemma}
{\bigskip\begin{mdframed}[backgroundcolor=gray!40,rightline=false,leftline=false,topline=false,bottomline=false]\begin{edtheorem}}
    {\end{edtheorem}\end{mdframed}\bigskip}
\newtheorem*{pdtheorem}{例}
\newenvironment{example}
{\bigskip\begin{mdframed}[backgroundcolor=gray!40,rightline=false,leftline=false,topline=false,bottomline=false]\begin{pdtheorem}}
    {\end{pdtheorem}\end{mdframed}\bigskip}

\usepackage[protrusion=true,expansion=true]{microtype} % Better typography
\usepackage{wrapfig} % Allows in-line images
\usepackage{mathpazo} % Use the Palatino font
\usepackage[T1]{fontenc} % Required for accented characters
\linespread{1.05} % Change line spacing here, Palatino benefits from a slight increase by default

\makeatletter
\renewcommand\@biblabel[1]{\textbf{#1.}} % Change the square brackets for each bibliography item from '[1]' to '1.'
\renewcommand{\@listI}{\itemsep=0pt} % Reduce the space between items in the itemize and enumerate environments and the bibliography

\renewcommand{\maketitle}{ % Customize the title - do not edit title
  % and author name here, see the TITLE block
  % below
  \renewcommand\refname{参考文献}
  \newcommand{\D}{\displaystyle}\newcommand{\ri}{\Rightarrow}
  \newcommand{\ds}{\displaystyle} \renewcommand{\ni}{\noindent}
  \newcommand{\pa}{\partial} \newcommand{\Om}{\Omega}
  \newcommand{\om}{\omega} \newcommand{\sik}{\sum_{i=1}^k}
  \newcommand{\vov}{\Vert\omega\Vert} \newcommand{\Umy}{U_{\mu_i,y^i}}
  \newcommand{\lamns}{\lambda_n^{^{\scriptstyle\sigma}}}
  \newcommand{\chiomn}{\chi_{_{\Omega_n}}}
  \newcommand{\ullim}{\underline{\lim}} \newcommand{\bsy}{\boldsymbol}
  \newcommand{\mvb}{\mathversion{bold}} \newcommand{\la}{\lambda}
  \newcommand{\La}{\Lambda} \newcommand{\va}{\varepsilon}
  \newcommand{\be}{\beta} \newcommand{\al}{\alpha}
  \newcommand{\dis}{\displaystyle} \newcommand{\R}{{\mathbb R}}
  \newcommand{\N}{{\mathbb N}} \newcommand{\cF}{{\mathcal F}}
  \newcommand{\gB}{{\mathfrak B}} \newcommand{\eps}{\epsilon}
  \begin{flushright} % Right align
    {\LARGE\@title} % Increase the font size of the title
    
    \vspace{50pt} % Some vertical space between the title and author name
    
    {\large\@author} % Author name
    \\\@date % Date
    
    \vspace{40pt} % Some vertical space between the author block and abstract
  \end{flushright}
}

% ----------------------------------------------------------------------------------------
%	TITLE
% ----------------------------------------------------------------------------------------
\begin{document}
\title{\textbf{《常微分方程教程》习题2.3.5,(2)}} 
\author{\small{叶卢庆}\\{\small{杭州师范大学理学院,学号:1002011005}}\\{\small{Email:h5411167@gmail.com}}} % Institution
\renewcommand{\today}{\number\year. \number\month. \number\day}
\date{\today} % Date
\setlength\epigraphwidth{0.7\linewidth}
\epigraph{一切有为法,如梦幻泡影,如露亦如电,应作如是观.}{《金刚经》}
% ----------------------------------------------------------------------------------------


\maketitle % Print the title section

% ----------------------------------------------------------------------------------------
%	ABSTRACT AND KEYWORDS
% ----------------------------------------------------------------------------------------

% \renewcommand{\abstractname}{摘要} % Uncomment to change the name of the abstract to something else

% \begin{abstract}

% \end{abstract}

% \hspace*{3,6mm}\textit{关键词:}  % Keywords

% \vspace{30pt} % Some vertical space between the abstract and first section

% ----------------------------------------------------------------------------------------
%	ESSAY BODY
% ----------------------------------------------------------------------------------------
\begin{exercise}[2.3.5,(2)]
考虑方程
\begin{equation}
  \label{eq:1}
  \frac{dy}{dx}+p(x)y=q(x),
\end{equation}
其中 $p(x),q(x)$ 都是以 $w>0$ 为周期的连续函数.试证若 $q(x)$ 不恒为0,
则如上方程有唯一的 $w$ 周期解当且仅当
$$
\frac{1}{w}\int_0^wp(x)dx\neq 0.
$$
试求出此解.
\end{exercise}
\begin{proof}[解]
将方程 \eqref{eq:1} 化为
\begin{equation}
  \label{eq:2}
  dy+(p(x)y-q(x))dx=0.
\end{equation}
方程 \eqref{eq:2} 不一定是恰当微分方程,我们在方程两边同时乘以非零函数
$u(x)$,得到
\begin{equation}
  \label{eq:3}
  u(x)dy+u(x)(p(x)y-q(x))dx=0.
\end{equation}
假设方程 \eqref{eq:3} 是一个恰当方程,则
$$
\frac{du(x)}{dx}=u(x)p(x)\ri u(x)=be^{\int p(x)dx}.
$$
不妨设 $b=1$.于是我们得到恰当微分方程
\begin{equation}
  \label{eq:4}
  e^{\int p(x)dx}dy+e^{\int p(x)dx}(p(x)y-q(x))dx=0.
\end{equation}
在方程 \eqref{eq:4} 中,两个 $\int p(x)dx$ 是同一个函数.设存在二元函数
$\phi(x,y)$,使得
$$
\frac{\pa\phi}{\pa y}=e^{\int p(x)dx}\ri \phi=ye^{\int p(x)dx}+f(x).
$$
因此
$$
\frac{\pa \phi}{\pa x}=yp(x)e^{\int p(x)dx}+f'(x)=yp(x)e^{\int
  p(x)dx}-q(x)e^{\int p(x)dx}.
$$
因此
$$
f(x)=-\int q(x)e^{\int p(x)dx}dx+C.
$$
可得通积分为
\begin{equation}
  \label{eq:5}
  ye^{\int p(x)dx}-\int q(x)e^{\int p(x)dx}dx+C=0.
\end{equation}
设 $H(x)=\int p(x)dx$,则方程 \eqref{eq:5} 变为
\begin{equation}
  \label{eq:6}
  ye^{H(x)}-\int q(x)e^{H(x)}dx+C=0.
\end{equation}
设 $K(x)=\int q(x)e^{H(x)}dx$.我们知道,
\begin{equation}
  \label{eq:7}
  y_te^{H(t)}-K(t)+C=0.
\end{equation}
\begin{equation}
  \label{eq:8}
  y_{t+w}e^{H(t+w)}-K(t+w)+C=0.
\end{equation}
如果 \eqref{eq:5} 是周期解,说明 $y_t=y_{t+w}$.在方程 \eqref{eq:7} 的两
边同时乘以 $e^{H(t+w)-H(t)}$,可得
\begin{equation}
  \label{eq:9}
  y_te^{H(t+w)}-K(t)e^{H(t+w)-H(t)}+Ce^{H(t+w)-H(t)}=0.
\end{equation}
\eqref{eq:9} 减去 \eqref{eq:8},可得
\begin{equation}
  \label{eq:10}
  K(t+w)-K(t)e^{H(t+w)-H(t)}+C(e^{H(t+w)-H(t)}-1)=0.
\end{equation}
根据 $p(x)$ 的周期性,可得\eqref{eq:10} 也就是
\begin{equation}
  \label{eq:11}
  K(t+w)-K(t)e^{\int_{0}^{w}p(x)dx}+C(e^{\int_{0}^wp(x)dx}-1)=0.
\end{equation}
从 \eqref{eq:11} 可以看出,$\int_{0}^{w}p(x)dx=0$ 当且仅当 $C$ 被唯一确
定.
\end{proof}
% ----------------------------------------------------------------------------------------
%	BIBLIOGRAPHY
% ----------------------------------------------------------------------------------------

\bibliographystyle{unsrt}

\bibliography{sample}

% ----------------------------------------------------------------------------------------
\end{document}