\documentclass[a4paper]{article}
\usepackage{amsmath,amsfonts,amsthm,amssymb}
\usepackage{bm}
\usepackage{euler}
\usepackage{hyperref}
\usepackage{geometry}
\usepackage{yhmath}
\usepackage{pstricks-add}
\usepackage{framed,mdframed}
\usepackage{graphicx,color} 
\usepackage{mathrsfs,xcolor} 
\usepackage[all]{xy}
\usepackage{fancybox} 
\usepackage{xeCJK}
\newtheorem*{theo}{定理}
\newtheorem*{exe}{题目}
\newtheorem*{rem}{评论}
\newmdtheoremenv{lemma}{引理}
\newmdtheoremenv{corollary}{推论}
\newmdtheoremenv{example}{例}
\newenvironment{theorem}
{\bigskip\begin{mdframed}\begin{theo}}
    {\end{theo}\end{mdframed}\bigskip}
\newenvironment{exercise}
{\bigskip\begin{mdframed}\begin{exe}}
    {\end{exe}\end{mdframed}\bigskip}
\geometry{left=2.5cm,right=2.5cm,top=2.5cm,bottom=2.5cm}
\setCJKmainfont[BoldFont=SimHei]{SimSun}
\renewcommand{\today}{\number\year 年 \number\month 月 \number\day 日}
\newcommand{\D}{\displaystyle}\newcommand{\ri}{\Rightarrow}
\newcommand{\ds}{\displaystyle} \renewcommand{\ni}{\noindent}
\newcommand{\ov}{\overrightarrow}
\newcommand{\pa}{\partial} \newcommand{\Om}{\Omega}
\newcommand{\om}{\omega} \newcommand{\sik}{\sum_{i=1}^k}
\newcommand{\vov}{\Vert\omega\Vert} \newcommand{\Umy}{U_{\mu_i,y^i}}
\newcommand{\lamns}{\lambda_n^{^{\scriptstyle\sigma}}}
\newcommand{\chiomn}{\chi_{_{\Omega_n}}}
\newcommand{\ullim}{\underline{\lim}} \newcommand{\bsy}{\boldsymbol}
\newcommand{\mvb}{\mathversion{bold}} \newcommand{\la}{\lambda}
\newcommand{\La}{\Lambda} \newcommand{\va}{\varepsilon}
\newcommand{\be}{\beta} \newcommand{\al}{\alpha}
\newcommand{\dis}{\displaystyle} \newcommand{\R}{{\mathbb R}}
\newcommand{\N}{{\mathbb N}} \newcommand{\cF}{{\mathcal F}}
\newcommand{\gB}{{\mathfrak B}} \newcommand{\eps}{\epsilon}
\renewcommand\refname{参考文献}\renewcommand\figurename{图}
\usepackage[]{caption2} 
\renewcommand{\captionlabeldelim}{}
\setlength\parindent{0pt}
\begin{document}
\title{\huge{\bf{常微分方程教程习题1.1.4}}} \author{\small{叶卢
    庆\footnote{叶卢庆(1992---),男,杭州师范大学理学院数学与应用数学专业
      本科在读,E-mail:yeluqingmathematics@gmail.com}}}
\maketitle
\begin{exercise}
  证明:设 $ y=g(x,C_1,C_2,\cdots,C_n)$ 是一个充分光滑的函数 族,其中 $
  x$是自变量,而 $ C_1,C_2,\cdots,C_n$ 是 $ n$ 个独立的参数(任 意常
  数),则存在一个形如
$$F(x,y,y',\cdots,y^{(n)})=0$$
的 $ n$ 阶微分方程,使得它的通解恰好是上述函数族.
\end{exercise}
\begin{proof}[\textbf{证明}]
  可得
  \begin{equation}\begin{cases} y=g(x,C_1,\cdots,C_n),\\
      y^{(1)}=g^{(1)}(x,C_1,\cdots,C_n),\\
      y^{(2)}=g^{(2)}(x,C_1,\cdots,C_n),\\ \vdots\\
      y^{(n-1)}=g^{(n-1)}(x,C_1,\cdots,C_n) \end{cases}. 
    \end{equation}
  以及

  $$ {\displaystyle y^{(n)}=g^{(n)}(x,C_1,\cdots,C_n). \ \ \ \ \ (2)}$$

  由于常数 $ C_1,\cdots,C_n$ 独立,因此我们可以反解出

  $$ {\displaystyle \begin{cases}
      C_1=p_1(x,y,y^{(1)},\cdots,y^{(n-1)}),\\
      C_2=p_2(x,y,y^{(1)},\cdots,y^{(n-1)}),\\ \vdots\\
      C_n=p_n(x,y,y^{(1)},\cdots,y^{(n-1)})\\ \end{cases}. \ \ \ \ \
    (3)}$$

  其中 $ p_1,p_2,\cdots,p_n$ 都是从 $ \mathbf{R}^n$ 到 $ \mathbf{R}$ 的
  函数.将 式 (3) 代入 (2),得到

  $$ {\displaystyle
    y^{(n)}=g^{(n)}(x,p_1(x,y,y^{(1)},\cdots,y^{(n-1)}),\cdots,p_n(x,y,y^{(1)},\cdots,y^{(n-1)})). \
    \ \ \ \ (4)}$$式 (4)即为我们所求的微分方程.
\end{proof}
\end{document}
