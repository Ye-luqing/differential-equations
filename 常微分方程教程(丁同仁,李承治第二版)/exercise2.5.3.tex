\documentclass[a4paper, 12pt]{article} % Font size (can be 10pt, 11pt or 12pt) and paper size (remove a4paper for US letter paper)
\usepackage{amsmath,amsfonts,bm}
\usepackage{hyperref,verbatim}
\usepackage{amsthm,epigraph} 
\usepackage{amssymb}
\usepackage{framed,mdframed}
\usepackage{graphicx,color} 
\usepackage{mathrsfs,xcolor} 
\usepackage[all]{xy}
\usepackage{fancybox} 
% \usepackage{xeCJK}
\usepackage{CJKutf8}
\newtheorem*{adtheorem}{定理}
% \setCJKmainfont[BoldFont=FZYaoTi,ItalicFont=FZYaoTi]{FZYaoTi}
\definecolor{shadecolor}{rgb}{1.0,0.9,0.9} %背景色为浅红色
\newenvironment{theorem}
{\bigskip\begin{mdframed}[backgroundcolor=gray!40,rightline=false,leftline=false,topline=false,bottomline=false]\begin{adtheorem}}
    {\end{adtheorem}\end{mdframed}\bigskip}
\newtheorem*{bdtheorem}{定义}
\newenvironment{definition}
{\bigskip\begin{mdframed}[backgroundcolor=gray!40,rightline=false,leftline=false,topline=false,bottomline=false]\begin{bdtheorem}}
    {\end{bdtheorem}\end{mdframed}\bigskip}
\newtheorem*{cdtheorem}{习题}
\newenvironment{exercise}
{\bigskip\begin{mdframed}[backgroundcolor=gray!40,rightline=false,leftline=false,topline=false,bottomline=false]\begin{cdtheorem}}
    {\end{cdtheorem}\end{mdframed}\bigskip}
\newtheorem*{ddtheorem}{注}
\newenvironment{remark}
{\bigskip\begin{mdframed}[backgroundcolor=gray!40,rightline=false,leftline=false,topline=false,bottomline=false]\begin{ddtheorem}}
    {\end{ddtheorem}\end{mdframed}\bigskip}
\newtheorem*{edtheorem}{引理}
\newenvironment{lemma}
{\bigskip\begin{mdframed}[backgroundcolor=gray!40,rightline=false,leftline=false,topline=false,bottomline=false]\begin{edtheorem}}
    {\end{edtheorem}\end{mdframed}\bigskip}
\newtheorem*{pdtheorem}{例}
\newenvironment{example}
{\bigskip\begin{mdframed}[backgroundcolor=gray!40,rightline=false,leftline=false,topline=false,bottomline=false]\begin{pdtheorem}}
    {\end{pdtheorem}\end{mdframed}\bigskip}

\usepackage[protrusion=true,expansion=true]{microtype} % Better typography
\usepackage{wrapfig} % Allows in-line images
\usepackage{mathpazo} % Use the Palatino font
\usepackage[T1]{fontenc} % Required for accented characters
\linespread{1.05} % Change line spacing here, Palatino benefits from a slight increase by default

\makeatletter
\renewcommand\@biblabel[1]{\textbf{#1.}} % Change the square brackets for each bibliography item from '[1]' to '1.'
\renewcommand{\@listI}{\itemsep=0pt} % Reduce the space between items in the itemize and enumerate environments and the bibliography

\renewcommand{\maketitle}{ % Customize the title - do not edit title
  % and author name here, see the TITLE block
  % below
  \renewcommand\refname{参考文献}
  \newcommand{\D}{\displaystyle}\newcommand{\ri}{\Rightarrow}
  \newcommand{\ds}{\displaystyle} \renewcommand{\ni}{\noindent}
  \newcommand{\pa}{\partial} \newcommand{\Om}{\Omega}
  \newcommand{\om}{\omega} \newcommand{\sik}{\sum_{i=1}^k}
  \newcommand{\vov}{\Vert\omega\Vert} \newcommand{\Umy}{U_{\mu_i,y^i}}
  \newcommand{\lamns}{\lambda_n^{^{\scriptstyle\sigma}}}
  \newcommand{\chiomn}{\chi_{_{\Omega_n}}}
  \newcommand{\ullim}{\underline{\lim}} \newcommand{\bsy}{\boldsymbol}
  \newcommand{\mvb}{\mathversion{bold}} \newcommand{\la}{\lambda}
  \newcommand{\La}{\Lambda} \newcommand{\va}{\varepsilon}
  \newcommand{\be}{\beta} \newcommand{\al}{\alpha}
  \newcommand{\dis}{\displaystyle} \newcommand{\R}{{\mathbb R}}
  \newcommand{\N}{{\mathbb N}} \newcommand{\cF}{{\mathcal F}}
  \newcommand{\gB}{{\mathfrak B}} \newcommand{\eps}{\epsilon}
  \begin{flushright} % Right align
    {\LARGE\@title} % Increase the font size of the title
    
    \vspace{50pt} % Some vertical space between the title and author name
    
    {\large\@author} % Author name
    \\\@date % Date
    
    \vspace{40pt} % Some vertical space between the author block and abstract
  \end{flushright}
}

% ----------------------------------------------------------------------------------------
%	TITLE
% ----------------------------------------------------------------------------------------
\begin{document}
\begin{CJK}{UTF8}{gkai}
  \title{\textbf{习题2.5.3}} 
  % \setlength\epigraphwidth{0.7\linewidth}
  \author{\small{叶卢庆}\\{\small{杭州师范大学理学院,学号:1002011005}}\\{\small{Email:h5411167@gmail.com}}} % Institution
  \renewcommand{\today}{\number\year. \number\month. \number\day}
  \date{\today} % Date
  
  % ----------------------------------------------------------------------------------------
  
  
  \maketitle % Print the title section
  
  % ----------------------------------------------------------------------------------------
  %	ABSTRACT AND KEYWORDS
  % ----------------------------------------------------------------------------------------
  
  % \renewcommand{\abstractname}{摘要} % Uncomment to change the name of the abstract to something else
  
  % \begin{abstract}
  
  % \end{abstract}
  
  % \hspace*{3,6mm}\textit{关键词:}  % Keywords
  
  % \vspace{30pt} % Some vertical space between the abstract and first section
  
  % ----------------------------------------------------------------------------------------
  %	ESSAY BODY
  % ----------------------------------------------------------------------------------------
  \begin{exercise}[2.5.3]
证明齐次方程
\begin{equation}
  \label{eq:0}
  P(x,y)dx+Q(x,y)dy=0
\end{equation}
有积分因子 $u=\frac{1}{xP+yQ}$.    
  \end{exercise}
  \begin{proof}[证明]
不妨设 $Q(x,y)\neq 0$.令 $y=ux$,其中 $u$ 是 $x$ 的函数,则
$$
x^mP(1,u)dx+x^mQ(1,u)dy=0.
$$
设 $x\neq 0$,则
$$
P(1,u)dx+Q(1,u)dy=0.
$$
我们知道,
$$
dy=xdu+udx.
$$
因此
$$
P(1,u)dx+Q(1,u)(xdu+udx)=0.
$$
即
\begin{equation}
  \label{eq:1}
  (P(1,u)+uQ(1,u))dx+xQ(1,u)du=0.
\end{equation}
易得 \eqref{eq:1} 的积分因子为
$$
\frac{1}{x(P(1,u)+uQ(1,u))}.
$$
因此,\eqref{eq:0} 的积分因子为
$$
\frac{1}{x^{m+1}(P(1,u)+uQ(1,u))}=\frac{1}{xP+yQ}.
$$
\begin{comment}
令 $y=ux$,其中 $u$ 是 $x$ 的函数.则我们得到
$$
x^mP(1,u)dx+x^mQ(1,u)dy=0.
$$
\end{comment}


\begin{comment}
由于 $P,Q$ 为齐次方程,因此
$$
P(tx,ty)=t^mP(x,y).
$$
令 $t=\frac{1}{x}$,可得
$$
P(1,\frac{y}{x})=\frac{1}{x^m}P(x,y).
$$
且我们有
$$
Q(tx,ty)=t^mQ(x,y).
$$
令 $t=\frac{1}{x}$,可得
$$
Q(1,\frac{y}{x})=\frac{1}{x^m}Q(x,y).
$$
因此我们得到微分方程
$$
x^mP(1,\frac{y}{x})dx+x^mQ(1,\frac{y}{x})dy=0.
$$
当 $x\neq 0$ 时,即
$$
P(1,\frac{y}{x})dx+Q(1,\frac{y}{x})dy=0.
$$



利用分组法.我们先看微分方程
$$
P(x,y)dx=0,
$$
\end{comment}
\begin{comment}
我们在齐次方程两边同时乘以非零函数 $\frac{1}{xP+yQ}$,可得
\begin{equation}
  \label{eq:1}
  \frac{1}{xP+yQ}P(x,y)dx+\frac{1}{xP+yQ}Q(x,y)dy=0.
\end{equation}
我们来证明 \eqref{eq:1} 是恰当微分方程,即证明
$$
\frac{\pa \frac{1}{xP+yQ}}{\pa y}P+\frac{1}{xP+yQ}\frac{\pa P}{\pa
  y}=\frac{\frac{1}{xP+yQ}}{\pa x}Q+\frac{1}{xP+yQ}\frac{\pa Q}{\pa x}.
$$
也即证明
\begin{align*}
&  \frac{-1}{(xP+yQ)^2}(x \frac{\pa P}{\pa y}+Q+y \frac{\pa Q}{\pa
    y})P+\frac{1}{xP+yQ}\frac{\pa P}{\pa y}\\&=\frac{-1}{(xP+yQ)^2}(P+x
  \frac{\pa P}{\pa x}+y \frac{\pa Q}{\pa x})Q+\frac{1}{xP+yQ}\frac{\pa
  Q}{\pa x}.
\end{align*}
即证明
\begin{align*}
&  \frac{-1}{(xP+yQ)}(x \frac{\pa P}{\pa y}+y \frac{\pa Q}{\pa
    y})P+\frac{\pa P}{\pa y}\\&=\frac{-1}{(xP+yQ)}(x
  \frac{\pa P}{\pa x}+y \frac{\pa Q}{\pa x})Q+\frac{\pa
  Q}{\pa x}.
\end{align*}
两边同时乘以 $xP+yQ$,即证明
\begin{align*}
  -(x \frac{\pa P}{\pa y}+y \frac{\pa Q}{\pa y})P+\frac{\pa P}{\pa
    y}(xP+yQ)=-(x \frac{\pa P}{\pa x}+y \frac{\pa Q}{\pa
    x})Q+\frac{\pa Q}{\pa x}(xP+yQ).
\end{align*}
我们知道,
$$
P(tx,ty)=t^mP(x,y),Q(tx,ty)=t^mQ(x,y),
$$
\end{comment}
  \end{proof}

  % ----------------------------------------------------------------------------------------
  %	BIBLIOGRAPHY
  % ----------------------------------------------------------------------------------------
  
  \bibliographystyle{unsrt}
  
  \bibliography{sample}
  
  % ----------------------------------------------------------------------------------------
\end{CJK}
\end{document}