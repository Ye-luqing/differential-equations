\documentclass[a4paper, 12pt]{article} % Font size (can be 10pt, 11pt or 12pt) and paper size (remove a4paper for US letter paper)
\usepackage{amsmath,amsfonts,bm}
\usepackage{hyperref}
\usepackage{amsthm,epigraph} 
\usepackage{amssymb}
\usepackage{framed,mdframed}
\usepackage{graphicx,color} 
\usepackage{mathrsfs,xcolor} 
\usepackage[all]{xy}
\usepackage{fancybox} 
% \usepackage{xeCJK}
\usepackage{CJKutf8}
\newtheorem*{adtheorem}{定理}
% \setCJKmainfont[BoldFont=FZYaoTi,ItalicFont=FZYaoTi]{FZYaoTi}
\definecolor{shadecolor}{rgb}{1.0,0.9,0.9} %背景色为浅红色
\newenvironment{theorem}
{\bigskip\begin{mdframed}[backgroundcolor=gray!40,rightline=false,leftline=false,topline=false,bottomline=false]\begin{adtheorem}}
    {\end{adtheorem}\end{mdframed}\bigskip}
\newtheorem*{bdtheorem}{定义}
\newenvironment{definition}
{\bigskip\begin{mdframed}[backgroundcolor=gray!40,rightline=false,leftline=false,topline=false,bottomline=false]\begin{bdtheorem}}
    {\end{bdtheorem}\end{mdframed}\bigskip}
\newtheorem*{cdtheorem}{习题}
\newenvironment{exercise}
{\bigskip\begin{mdframed}[backgroundcolor=gray!40,rightline=false,leftline=false,topline=false,bottomline=false]\begin{cdtheorem}}
    {\end{cdtheorem}\end{mdframed}\bigskip}
\newtheorem*{ddtheorem}{注}
\newenvironment{remark}
{\bigskip\begin{mdframed}[backgroundcolor=gray!40,rightline=false,leftline=false,topline=false,bottomline=false]\begin{ddtheorem}}
    {\end{ddtheorem}\end{mdframed}\bigskip}
\newtheorem*{edtheorem}{引理}
\newenvironment{lemma}
{\bigskip\begin{mdframed}[backgroundcolor=gray!40,rightline=false,leftline=false,topline=false,bottomline=false]\begin{edtheorem}}
    {\end{edtheorem}\end{mdframed}\bigskip}
\newtheorem*{pdtheorem}{例}
\newenvironment{example}
{\bigskip\begin{mdframed}[backgroundcolor=gray!40,rightline=false,leftline=false,topline=false,bottomline=false]\begin{pdtheorem}}
    {\end{pdtheorem}\end{mdframed}\bigskip}

\usepackage[protrusion=true,expansion=true]{microtype} % Better typography
\usepackage{wrapfig} % Allows in-line images
\usepackage{mathpazo} % Use the Palatino font
\usepackage[T1]{fontenc} % Required for accented characters
\linespread{1.05} % Change line spacing here, Palatino benefits from a slight increase by default

\makeatletter
\renewcommand\@biblabel[1]{\textbf{#1.}} % Change the square brackets for each bibliography item from '[1]' to '1.'
\renewcommand{\@listI}{\itemsep=0pt} % Reduce the space between items in the itemize and enumerate environments and the bibliography

\renewcommand{\maketitle}{ % Customize the title - do not edit title
  % and author name here, see the TITLE block
  % below
  \renewcommand\refname{参考文献}
  \newcommand{\D}{\displaystyle}\newcommand{\ri}{\Rightarrow}
  \newcommand{\ds}{\displaystyle} \renewcommand{\ni}{\noindent}
  \newcommand{\pa}{\partial} \newcommand{\Om}{\Omega}
  \newcommand{\om}{\omega} \newcommand{\sik}{\sum_{i=1}^k}
  \newcommand{\vov}{\Vert\omega\Vert} \newcommand{\Umy}{U_{\mu_i,y^i}}
  \newcommand{\lamns}{\lambda_n^{^{\scriptstyle\sigma}}}
  \newcommand{\chiomn}{\chi_{_{\Omega_n}}}
  \newcommand{\ullim}{\underline{\lim}} \newcommand{\bsy}{\boldsymbol}
  \newcommand{\mvb}{\mathversion{bold}} \newcommand{\la}{\lambda}
  \newcommand{\La}{\Lambda} \newcommand{\va}{\varepsilon}
  \newcommand{\be}{\beta} \newcommand{\al}{\alpha}
  \newcommand{\dis}{\displaystyle} \newcommand{\R}{{\mathbb R}}
  \newcommand{\N}{{\mathbb N}} \newcommand{\cF}{{\mathcal F}}
  \newcommand{\gB}{{\mathfrak B}} \newcommand{\eps}{\epsilon}
  \begin{flushright} % Right align
    {\LARGE\@title} % Increase the font size of the title
    
    \vspace{50pt} % Some vertical space between the title and author name
    
    {\large\@author} % Author name
    \\\@date % Date
    
    \vspace{40pt} % Some vertical space between the author block and abstract
  \end{flushright}
}

% ----------------------------------------------------------------------------------------
%	TITLE
% ----------------------------------------------------------------------------------------
\begin{document}
\begin{CJK}{UTF8}{gkai}
  \title{\textbf{《常微分方程教
      程》\footnote{\cite{dinglichang}}习题2-4,1}}
  % \setlength\epigraphwidth{0.7\linewidth}
  \author{\small{叶卢庆}\\{\small{杭州师范大学理学院,学
        号:1002011005}}\\{\small{Email:h5411167@gmail.com}}} % Institution
  \renewcommand{\today}{\number\year. \number\month. \number\day}
  \date{\today} % Date
  
  % ----------------------------------------------------------------------------------------
  
  
  \maketitle % Print the title section
  
  % ----------------------------------------------------------------------------------------
  %	ABSTRACT AND KEYWORDS
  % ----------------------------------------------------------------------------------------
  
  % \renewcommand{\abstractname}{摘要} % Uncomment to change the name of the abstract to something else
  
  % \begin{abstract}
  
  % \end{abstract}
  
  % \hspace*{3,6mm}\textit{关键词:} % Keywords
  
  % \vspace{30pt} % Some vertical space between the abstract and first section
  
  % ----------------------------------------------------------------------------------------
  %	ESSAY BODY
  % ----------------------------------------------------------------------------------------
  求解下列微分方程
  \begin{exercise}[2.4.1,(1)]
$$
y'=\frac{2y-x}{2x-y}.
$$
\end{exercise}
\begin{proof}[解]
  这是一个齐次方程,令 $y(x)=u(x)x$,则在 $x\neq 0$ 时,化为
$$
y'=\frac{2u-1}{2-u}.
$$
我们知道,
$$
\frac{dy}{dx}=\frac{du}{dx}+u.
$$
因此
$$
\frac{du}{dx}+u=\frac{2u-1}{2-u}.
$$
也即
$$
\frac{du}{dx}=\frac{u^2-1}{2-u}.
$$
当 $u\neq \pm 1$ 时,
$$
\frac{2-u}{u^2-1}du-dx=0.
$$
这是一个恰当方程.设二元函数 $\phi(u,x)$ 满足
$$
\frac{\pa\phi}{\pa u}=\frac{2-u}{u^2-1}.
$$
我们先来求
$$
\int \frac{2-u}{u^2-1}du.
$$
\begin{shaded}
为此,我们先来求
$$
\int \frac{1}{u^2-1}du.
$$
我们知道,
$$
\cosh^2t-\sinh^2t=1\ri \tanh^2t-1=\frac{-1}{\cosh^2t},
$$
于是,我们不妨令 $u=\tanh t$,那么
$$
\int \frac{1}{u^2-1}du=\int -\cosh^2tdu.
$$
下面我们给 $\tanh t$ 求导.易得
$$
(\tanh t)'=(\frac{\sinh t}{\cosh t})'=\frac{1}{\cosh^2 t}.
$$
因此
$$
\frac{du}{dt}=\frac{1}{\cosh^2t}.
$$
可见,
$$
\int \frac{1}{u^2-1}du=\int -1dt=-t+C_1=-\tanh^{-1}u+C_1.
$$
\end{shaded}
\begin{shaded}
  我们再来求
$$
\int \frac{u}{u^2-1}du.
$$
易得其为
$$
\frac{1}{2}\ln |u^2-1|+C_2.
$$
\end{shaded}
综上所述,可得
$$
\int \frac{2-u}{u^2-1}du=-2\tanh^{-1}u-\frac{1}{2}\ln|u^2-1|+C.
$$
可见,
$$
\phi=-2\tanh^{-1}u-\frac{1}{2}\ln|u^2-1|+h(x)\ri h'(x)=-1\ri h(x)=-x+D.
$$
综上可见,通积分为
$$
-2\tanh^{-1}\frac{y}{x}-\frac{1}{2}\ln|\frac{y^2}{x^2}-1|-x+D=0.(x\neq
0,\frac{y}{x}\neq \pm 1).
$$
而当 $x=0$ 时,可得 $y=-2x+D$.当 $x\neq 0$ 且 $\frac{y}{x}= 1$
时,$y=x+D$,当 $x\neq 0$ 且 $\frac{y}{x}=-1$ 时,$y=-x+D$.
\end{proof}
% ----------------------------------------------------------------------------------------
%	BIBLIOGRAPH
% ----------------------------------------------------------------------------------------
  
\bibliographystyle{unsrt}
  
\bibliography{sample}
  
% ----------------------------------------------------------------------------------------
\end{CJK}
\end{document}