\documentclass[a4paper, 12pt]{article} % Font size (can be 10pt, 11pt or 12pt) and paper size (remove a4paper for US letter paper)
\usepackage{amsmath,amsfonts,bm}
\usepackage{hyperref,verbatim}
\usepackage{amsthm,epigraph} 
\usepackage{amssymb}
\usepackage{framed,mdframed}
\usepackage{graphicx,color} 
\usepackage{mathrsfs,xcolor} 
\usepackage[all]{xy}
\usepackage{fancybox} 
% \usepackage{xeCJK}
\usepackage{CJKutf8}
\newtheorem*{adtheorem}{定理}
% \setCJKmainfont[BoldFont=FZYaoTi,ItalicFont=FZYaoTi]{FZYaoTi}
\definecolor{shadecolor}{rgb}{1.0,0.9,0.9} %背景色为浅红色
\newenvironment{theorem}
{\bigskip\begin{mdframed}[backgroundcolor=gray!40,rightline=false,leftline=false,topline=false,bottomline=false]\begin{adtheorem}}
    {\end{adtheorem}\end{mdframed}\bigskip}
\newtheorem*{bdtheorem}{定义}
\newenvironment{definition}
{\bigskip\begin{mdframed}[backgroundcolor=gray!40,rightline=false,leftline=false,topline=false,bottomline=false]\begin{bdtheorem}}
    {\end{bdtheorem}\end{mdframed}\bigskip}
\newtheorem*{cdtheorem}{习题}
\newenvironment{exercise}
{\bigskip\begin{mdframed}[backgroundcolor=gray!40,rightline=false,leftline=false,topline=false,bottomline=false]\begin{cdtheorem}}
    {\end{cdtheorem}\end{mdframed}\bigskip}
\newtheorem*{ddtheorem}{注}
\newenvironment{remark}
{\bigskip\begin{mdframed}[backgroundcolor=gray!40,rightline=false,leftline=false,topline=false,bottomline=false]\begin{ddtheorem}}
    {\end{ddtheorem}\end{mdframed}\bigskip}
\newtheorem*{edtheorem}{引理}
\newenvironment{lemma}
{\bigskip\begin{mdframed}[backgroundcolor=gray!40,rightline=false,leftline=false,topline=false,bottomline=false]\begin{edtheorem}}
    {\end{edtheorem}\end{mdframed}\bigskip}
\newtheorem*{pdtheorem}{例}
\newenvironment{example}
{\bigskip\begin{mdframed}[backgroundcolor=gray!40,rightline=false,leftline=false,topline=false,bottomline=false]\begin{pdtheorem}}
    {\end{pdtheorem}\end{mdframed}\bigskip}

\usepackage[protrusion=true,expansion=true]{microtype} % Better typography
\usepackage{wrapfig} % Allows in-line images
\usepackage{mathpazo} % Use the Palatino font
\usepackage[T1]{fontenc} % Required for accented characters
\linespread{1.05} % Change line spacing here, Palatino benefits from a slight increase by default

\makeatletter
\renewcommand\@biblabel[1]{\textbf{#1.}} % Change the square brackets for each bibliography item from '[1]' to '1.'
\renewcommand{\@listI}{\itemsep=0pt} % Reduce the space between items in the itemize and enumerate environments and the bibliography

\renewcommand{\maketitle}{ % Customize the title - do not edit title
  % and author name here, see the TITLE block
  % below
  \renewcommand\refname{参考文献}
  \newcommand{\D}{\displaystyle}\newcommand{\ri}{\Rightarrow}
  \newcommand{\ds}{\displaystyle} \renewcommand{\ni}{\noindent}
  \newcommand{\pa}{\partial} \newcommand{\Om}{\Omega}
  \newcommand{\om}{\omega} \newcommand{\sik}{\sum_{i=1}^k}
  \newcommand{\vov}{\Vert\omega\Vert} \newcommand{\Umy}{U_{\mu_i,y^i}}
  \newcommand{\lamns}{\lambda_n^{^{\scriptstyle\sigma}}}
  \newcommand{\chiomn}{\chi_{_{\Omega_n}}}
  \newcommand{\ullim}{\underline{\lim}} \newcommand{\bsy}{\boldsymbol}
  \newcommand{\mvb}{\mathversion{bold}} \newcommand{\la}{\lambda}
  \newcommand{\La}{\Lambda} \newcommand{\va}{\varepsilon}
  \newcommand{\be}{\beta} \newcommand{\al}{\alpha}
  \newcommand{\dis}{\displaystyle} \newcommand{\R}{{\mathbb R}}
  \newcommand{\N}{{\mathbb N}} \newcommand{\cF}{{\mathcal F}}
  \newcommand{\gB}{{\mathfrak B}} \newcommand{\eps}{\epsilon}
  \begin{flushright} % Right align
    {\LARGE\@title} % Increase the font size of the title
    
    \vspace{50pt} % Some vertical space between the title and author name
    
    {\large\@author} % Author name
    \\\@date % Date
    
    \vspace{40pt} % Some vertical space between the author block and abstract
  \end{flushright}
}

% ----------------------------------------------------------------------------------------
%	TITLE
% ----------------------------------------------------------------------------------------
\begin{document}
\begin{CJK}{UTF8}{gkai}
  \title{\textbf{习题2.5.1.7}} 
  % \setlength\epigraphwidth{0.7\linewidth}
  \author{\small{叶卢庆}\\{\small{杭州师范大学理学院,学号:1002011005}}\\{\small{Email:h5411167@gmail.com}}} % Institution
  \renewcommand{\today}{\number\year. \number\month. \number\day}
  \date{\today} % Date
  
  % ----------------------------------------------------------------------------------------
  
  
  \maketitle % Print the title section
  
  % ----------------------------------------------------------------------------------------
  %	ABSTRACT AND KEYWORDS
  % ----------------------------------------------------------------------------------------
  
  % \renewcommand{\abstractname}{摘要} % Uncomment to change the name of the abstract to something else
  
  % \begin{abstract}
  
  % \end{abstract}
  
  % \hspace*{3,6mm}\textit{关键词:}  % Keywords
  
  % \vspace{30pt} % Some vertical space between the abstract and first section
  
  % ----------------------------------------------------------------------------------------
  %	ESSAY BODY
  % ----------------------------------------------------------------------------------------
  \begin{exercise}[2.5.1.7]
求解微分方程
\begin{equation}
  \label{eq:0}
  y^3dx+2(x^2-xy^2)dy=0.
\end{equation}
  \end{exercise}
  \begin{proof}[证明]
\begin{comment}
进行分组,可得
$$
y^3dx+2x^2dy-2xy^2dy=0.
$$
对于微分方程
\begin{equation}
  \label{eq:1}
  y^3dx+2x^2dy=0
\end{equation}
来说,其积分因子为 $\frac{1}{y^3x^2}$,\eqref{eq:1} 式两边乘以积分因子后,得
到
$$
\frac{1}{x^2}dx+\frac{2}{y^3}dy=0.
$$
可得通积分为
$$
-\frac{1}{x}-\frac{1}{y^2}+C=0.
$$
易得 $\frac{1}{y^3x^2}g(\frac{1}{x}+\frac{1}{y^2})$ 也是 \eqref{eq:1}
的一个积分因子,其中 $g$ 是可微函数.对于微分方程
\begin{equation}
  \label{eq:2}
  -2xy^2dy=0
\end{equation}
来说,其积分因子为 $\frac{1}{x}$,易得乘以积分因子后得到
$$
-2y^2dy=0,
$$
可见通积分为
$$
\frac{-2}{3}y^3+D=0.
$$
易得 $\frac{1}{x}h(y^3)$ 也是 \eqref{eq:2} 的一个积分因子,
其中 $h$ 是可微函数.我们希望,
$$
\frac{1}{y^3x^2}g(\frac{1}{x}+\frac{1}{y^2})=\frac{1}{x}h(y^3)
$$
即
$$
\frac{1}{y^3x}g(\frac{1}{x}+\frac{1}{y^2})=h(y^3).
$$
\end{comment}




我们进行分组.可得
$$
y^3dx-2xy^2dy+2x^2dy=0.
$$
对于微分方程
\begin{equation}
  \label{eq:1}
  y^3dx-2xy^2dy=0
\end{equation}
来说,当 $y\neq 0$ 时,我们首先在两边乘以 $\frac{1}{y^2}$,得到微分方程
\begin{equation}
  \label{eq:2}
  ydx-2xdy=0.
\end{equation}
再在微分方程 \eqref{eq:2} 的两边乘以 $\frac{1}{y^3}$,得到微分方程
\begin{equation}
  \label{eq:3}
  \frac{1}{y^2}dx-\frac{2x}{y^3}dy=0.
\end{equation}
\eqref{eq:3} 是一个恰当微分方程.可见,微分方程 \eqref{eq:1} 的一个积分
因子为 $\frac{1}{y^5}$.我们得到通积分
$$
xy^{-2}+C=0.
$$
易得 $\frac{1}{y^5}g(\frac{x}{y^2})$ 也是 \eqref{eq:1} 的一个积分因子,
其中 $g$ 是可微函数.下面我们来看微分方程
\begin{equation}
  \label{eq:4}
  2x^2dy=0.
\end{equation}
当 $x\neq 0$ 时,易得该微分方程的积分因子为
$\frac{1}{x^2}$,\eqref{eq:4} 两边乘以该积分因子后,得到微分方程
$$
2dy=0,
$$
可见通积分为
$$
2y+D=0.
$$
易得 $\frac{1}{x^2}h(y)$ 也是 \eqref{eq:4} 的一个积分因子,其中 $h$ 是
可微函数.我们希望,
$$
\frac{1}{y^5}g(\frac{x}{y^2})=\frac{1}{x^2}h(y).
$$
令 $g(\frac{x}{y^2})=\frac{y^4}{x^2}$,$h(y)=\frac{1}{y}$.可见,微分方程
\eqref{eq:0} 的积分因子为 $\frac{1}{x^2y}$.在 \eqref{eq:0} 两边乘以积
分因子后,我们得到微分方程
\begin{equation}
  \label{eq:7}
  \frac{y^2}{x^2}dx+2(\frac{1}{y}-\frac{y}{x})dy=0.
\end{equation}
易得通积分为
$$
-\frac{y^2}{x}+2\ln |y|+E=0.
$$
而当 $x=0,y\neq 0$ 或 $y=0,x\neq 0$ 或 $x=y=0$ 时,情况易于讨论.
\begin{comment}
来说,积分因子为 $\frac{1}{xy^{3}}$.将积分因子乘上 \eqref{eq:1} 的两侧后,得
到
$$
\frac{1}{x}dx-\frac{2}{y}dy=0.
$$
可得通积分为
$$
\ln |x|-2\ln |y|+C=0.
$$
易得 $\frac{1}{xy^3}g($
\end{comment}
\begin{comment}
微分方程两边同时乘以非零函数 $u(x,y)$,得到
\begin{equation}\label{eq:1}
uy^3dx+2u(x^2-xy^2)dy=0.
\end{equation}
我们希望 \eqref{eq:1} 是恰当的,即
$$
\frac{\pa u}{\pa y}y^3+3y^2u=\frac{\pa u}{\oa x}(2x^2-2xy^2)+u(4x-2y^2),
$$
即
$$
\frac{\pa u}{\pa y}y^3+(5y^2-4x)u=\frac{\pa u}{\pa x}(2x^2-2xy^2).
$$
我们让 $u$ 是只关于 $x$ 的函数,可得
$$
4(y^2-x)u=2x(x-y^{2})\frac{du}{dx}
$$
当 $x\neq 0$ 时,不妨让 $u=\frac{1}{x^{2}}$.因此我们得到恰当微分方程
\begin{equation}
  \label{eq:2}
  \frac{y^3}{x^2}dx+2(1-\frac{y^2}{x})dy=0.
\end{equation}
设二元函数 $\phi(x,y)$ 满足
$$
\frac{\pa \phi}{\pa x}=\frac{y^3}{x^2}\ri \phi=-y^3 \frac{1}{x}+f(y).
$$
因此
$$
-\frac{3}{x}y^2+f'(y)=
$$
\end{comment}
  \end{proof}
  % ----------------------------------------------------------------------------------------
  %	BIBLIOGRAPHY
  % ----------------------------------------------------------------------------------------
  
  \bibliographystyle{unsrt}
  
  \bibliography{sample}
  
  % ----------------------------------------------------------------------------------------
\end{CJK}
\end{document}