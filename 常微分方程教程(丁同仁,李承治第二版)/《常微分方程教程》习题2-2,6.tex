\documentclass[a4paper, 12pt]{article} % Font size (can be 10pt, 11pt or 12pt) and paper size (remove a4paper for US letter paper)
\usepackage{amsmath,amsfonts,bm}
\usepackage{hyperref}
\usepackage{amsthm} 
\usepackage{amssymb}
\usepackage{framed,mdframed}
\usepackage{graphicx,color} 
\usepackage{mathrsfs,xcolor} 
\usepackage[all]{xy}
\usepackage{fancybox} 
\usepackage{xeCJK,asymptote}
\usepackage{pstricks-add}
\usepackage{pgf,tikz}
\usetikzlibrary{arrows}
\pagestyle{empty}
\newtheorem*{adtheorem}{定理}
\setCJKmainfont[BoldFont=Adobe Song Std,ItalicFont=Adobe Song Std]{Adobe Song Std}
\definecolor{shadecolor}{rgb}{1.0,0.9,0.9} %背景色为浅红色
\newenvironment{theorem}
{\bigskip\begin{mdframed}[backgroundcolor=gray!40,rightline=false,leftline=false,topline=false,bottomline=false]\begin{adtheorem}}
    {\end{adtheorem}\end{mdframed}\bigskip}
\newtheorem*{bdtheorem}{定义}
\newenvironment{definition}
{\bigskip\begin{mdframed}[backgroundcolor=gray!40,rightline=false,leftline=false,topline=false,bottomline=false]\begin{bdtheorem}}
    {\end{bdtheorem}\end{mdframed}\bigskip}
\newtheorem*{cdtheorem}{习题}
\newenvironment{exercise}
{\bigskip\begin{mdframed}[backgroundcolor=gray!40,rightline=false,leftline=false,topline=false,bottomline=false]\begin{cdtheorem}}
    {\end{cdtheorem}\end{mdframed}\bigskip}
\newtheorem{ddtheorem}{注}
\newenvironment{remark}
{\bigskip\begin{mdframed}[backgroundcolor=gray!40,rightline=false,leftline=false,topline=false,bottomline=false]\begin{ddtheorem}}
    {\end{ddtheorem}\end{mdframed}\bigskip}
\newtheorem{edtheorem}{引理}
\newenvironment{lemma}
{\bigskip\begin{mdframed}[backgroundcolor=gray!40,rightline=false,leftline=false,topline=false,bottomline=false]\begin{edtheorem}}
    {\end{edtheorem}\end{mdframed}\bigskip}
\usepackage[protrusion=true,expansion=true]{microtype} % Better typography
\usepackage{wrapfig} % Allows in-line images
\usepackage{mathpazo} % Use the Palatino font
\usepackage[T1]{fontenc} % Required for accented characters
\linespread{1.05} % Change line spacing here, Palatino benefits from a slight increase by default

\makeatletter
\renewcommand\@biblabel[1]{\textbf{#1.}} % Change the square brackets for each bibliography item from '[1]' to '1.'
\renewcommand{\@listI}{\itemsep=0pt} % Reduce the space between items in the itemize and enumerate environments and the bibliography

\renewcommand{\maketitle}{ % Customize the title - do not edit title
  % and author name here, see the TITLE block
  % below
  \renewcommand\refname{参考文献}
  \newcommand{\D}{\displaystyle}\newcommand{\ri}{\Rightarrow}
  \newcommand{\ds}{\displaystyle} \renewcommand{\ni}{\noindent}
  \newcommand{\pa}{\partial} \newcommand{\Om}{\Omega}
  \newcommand{\om}{\omega} \newcommand{\sik}{\sum_{i=1}^k}
  \newcommand{\vov}{\Vert\omega\Vert} \newcommand{\Umy}{U_{\mu_i,y^i}}
  \newcommand{\lamns}{\lambda_n^{^{\scriptstyle\sigma}}}
  \newcommand{\chiomn}{\chi_{_{\Omega_n}}}
  \newcommand{\ullim}{\underline{\lim}} \newcommand{\bsy}{\boldsymbol}
  \newcommand{\mvb}{\mathversion{bold}} \newcommand{\la}{\lambda}
  \newcommand{\La}{\Lambda} \newcommand{\va}{\varepsilon}
  \newcommand{\be}{\beta} \newcommand{\al}{\alpha}
  \newcommand{\dis}{\displaystyle} \newcommand{\R}{{\mathbb R}}
  \newcommand{\N}{{\mathbb N}} \newcommand{\cF}{{\mathcal F}}
  \newcommand{\gB}{{\mathfrak B}} \newcommand{\eps}{\epsilon}
  \begin{flushright} % Right align
    {\LARGE\@title} % Increase the font size of the title
    
    \vspace{50pt} % Some vertical space between the title and author name
    
    {\large\@author} % Author name
    \\\@date % Date
    
    \vspace{40pt} % Some vertical space between the author block and abstract
  \end{flushright}
}

% ----------------------------------------------------------------------------------------
%	TITLE
% ----------------------------------------------------------------------------------------

\title{\textbf{《常微分方程教程》习题2-2,6}} 

\author{\small{叶卢庆}\\{\small{杭州师范大学理学院,学号:1002011005}}\\{\small{Email:h5411167@gmail.com}}} % Institution
\renewcommand{\today}{\number\year. \number\month. \number\day}
\date{\today} % Date

% ----------------------------------------------------------------------------------------

\begin{document}
\maketitle % Print the title section

% ----------------------------------------------------------------------------------------
%	ABSTRACT AND KEYWORDS
% ----------------------------------------------------------------------------------------

% \renewcommand{\abstractname}{摘要} % Uncomment to change the name of the abstract to something else

% \begin{abstract}

% \end{abstract}

% \hspace*{3,6mm}\textit{关键词:}  % Keywords

% \vspace{30pt} % Some vertical space between the abstract and first section

% ----------------------------------------------------------------------------------------
%	ESSAY BODY
% ----------------------------------------------------------------------------------------
利用上题结果\footnote{我把上题结果呈现如下:设微分方程
  \begin{equation}
    \label{eq:3.41}
    \frac{dy}{dx}=f(y),
  \end{equation}
  其中 $f(y)$ 在 $y=a$ 的某邻域(例如区间 $|y-a|\leq\varepsilon$)内连
  续,且 $f(y)=0$ 当且仅当 $y=a$.则在直线 $y=a$ 上的每一
  点,方程 \eqref{eq:3.41} 的解是局部唯一的,当且仅当瑕积分
$$
|\int_a^{a\pm \varepsilon}\frac{dy}{f(y)}|=\infty.
$$}(而不解方程),作出下列微分方程积分曲线族的草图
\begin{exercise}[2-2,6,(1)]
$$
\frac{dy}{dx}=\sqrt{|y|}.
$$
\end{exercise}
\begin{proof}[解]
该微分方程不满足上题的条件,因为对于任意给定的正实数 $\va$,
$$
\int_0^{\pm \va} \frac{1}{\sqrt{y}}dy
$$
不是无穷.我先作个弊,把方程解掉.显然 $y=0$ 是一个解.当 $y>0$ 时,根据反函数定理,方程变为
$$
\frac{dy}{dx}=\sqrt{y}\ri \frac{dx}{dy}=\frac{1}{\sqrt{y}}.
$$
于是,
$$
x=2\sqrt{y}+C.
$$
其中 $x>C$.当 $y<0$ 时,根据反函数定理,方程变为
$$
\frac{dx}{dy}=\frac{1}{\sqrt{-y}}.
$$
于是
$$
x=-2 \sqrt{-y}+C.
$$
其中 $x<C$.做图如下:\\\\
\includegraphics[width=\textwidth]{/home/luqing/11111.png}
注意上面的图中我们只画了有限条积分曲线,事实上积分曲线有无数条,填满了整
个平面.我们不可能把它们全画出来,否则我要直接泼墨了,给你一个黑屏.
\end{proof}
\begin{exercise}[2-2,6,(2)]
$$
\frac{dy}{dx}=
\begin{cases}
  y\ln|y|,y\neq 0,\\
0,y=0.
\end{cases}
$$
\end{exercise}
\begin{proof}[解]
我们发现,只有当 $y=\pm 1$ 的时候,条件 $y\ln |y|=0$和
$$
\int_1^{1\pm \va}\frac{1}{y\ln |y|}=\infty
$$
同时满足.我们现在画出积分曲线族验证一下.我们也作弊把方程解掉.易得
$y\neq 0$ 时,
$$
x=\ln\ln |y|+C.
$$
做出积分曲线族可得\\\\
\psset{xunit=1.0cm,yunit=1.0cm,algebraic=true,dotstyle=o,dotsize=3pt 0,linewidth=0.8pt,arrowsize=3pt 2,arrowinset=0.25}
\begin{pspicture*}(-4.3,-5.88)(23.02,6.3)
\psaxes[labelFontSize=\scriptstyle,xAxis=true,yAxis=true,Dx=1,Dy=1,ticksize=-2pt 0,subticks=2]{->}(0,0)(-4.3,-5.88)(23.02,6.3)
\psplot[plotpoints=200]{-4.3}{23.02}{2.718281828^(2.718281828^x)}
\psplot[plotpoints=200]{-4.3}{23.02}{2.718281828^(2.718281828^(x-1))}
\psplot[plotpoints=200]{-4.3}{23.02}{2.718281828^(2.718281828^(x-2))}
\psplot[plotpoints=200]{-4.3}{23.02}{2.718281828^(2.718281828^(x-3))}
\psplot[plotpoints=200]{-4.3}{23.02}{2.718281828^(2.718281828^(x-4))}
\psplot[plotpoints=200]{-4.3}{23.02}{2.718281828^(2.718281828^(x-5))}
\psplot[plotpoints=200]{-4.3}{23.02}{2.718281828^(2.718281828^(x-6))}
\psplot[plotpoints=200]{-4.3}{23.02}{2.718281828^(2.718281828^(x-7))}
\psplot[plotpoints=200]{-4.3}{23.02}{2.718281828^(2.718281828^(x-8))}
\psplot[plotpoints=200]{-4.3}{23.02}{2.718281828^(2.718281828^(x-9))}
\psplot[plotpoints=200]{-4.3}{23.02}{2.718281828^(2.718281828^(x-10))}
\psplot[plotpoints=200]{-4.3}{23.02}{2.718281828^(2.718281828^(x-11))}
\psplot[plotpoints=200]{-4.3}{23.02}{2.718281828^(2.718281828^(x-12))}
\psplot[plotpoints=200]{-4.3}{23.02}{2.718281828^(2.718281828^(x-13))}
\psplot[plotpoints=200]{-4.3}{23.02}{-2.718281828^(2.718281828^(x-2))}
\psplot[plotpoints=200]{-4.3}{23.02}{-2.718281828^(2.718281828^x)}
\psplot[plotpoints=200]{-4.3}{23.02}{-2.718281828^(2.718281828^(x-1))}
\psplot[plotpoints=200]{-4.3}{23.02}{-2.718281828^(2.718281828^(x-3))}
\psplot[plotpoints=200]{-4.3}{23.02}{-2.718281828^(2.718281828^(x-4))}
\psplot[plotpoints=200]{-4.3}{23.02}{-2.718281828^(2.718281828^(x-5))}
\psplot[plotpoints=200]{-4.3}{23.02}{-2.718281828^(2.718281828^(x-6))}
\psplot[plotpoints=200]{-4.3}{23.02}{-2.718281828^(2.718281828^(x-7))}
\psplot[plotpoints=200]{-4.3}{23.02}{-2.718281828^(2.718281828^(x-8))}
\psplot[plotpoints=200]{-4.3}{23.02}{-2.718281828^(2.718281828^(x-9))}
\psplot[plotpoints=200]{-4.3}{23.02}{-2.718281828^(2.718281828^(x-10))}
\psplot[plotpoints=200]{-4.3}{23.02}{-2.718281828^(2.718281828^(x-11))}
\psplot[plotpoints=200]{-4.3}{23.02}{-2.718281828^(2.718281828^(x-12))}
\psplot[plotpoints=200]{-4.3}{23.02}{-2.718281828^(2.718281828^(x-13))}
\psplot{-4.3}{23.02}{(--1-0*x)/1}
\psplot{-4.3}{23.02}{(-1-0*x)/1}
\end{pspicture*}
我们发现符合我们的预期要求.
\end{proof}
% ----------------------------------------------------------------------------------------
%	BIBLIOGRAPHY
% ----------------------------------------------------------------------------------------

\bibliographystyle{unsrt}

\bibliography{sample}

% ----------------------------------------------------------------------------------------
\end{document}