\documentclass[twoside,11pt]{article} 
\usepackage{amsmath,amsfonts,bm}
\usepackage{hyperref}
\usepackage{amsthm} 
\usepackage{amssymb}
\usepackage{framed,mdframed}
\usepackage{graphicx,color} 
\usepackage{mathrsfs,xcolor} 
\usepackage[all]{xy}
\usepackage{fancybox} 
\usepackage{xeCJK}
\newtheorem{adtheorem}{定理}
\setCJKmainfont[BoldFont=FangSong_GB2312,ItalicFont=FangSong_GB2312]{FangSong_GB2312}
\newenvironment{theorem}
{\begin{mdframed}[backgroundcolor=gray!40,rightline=false,leftline=false,topline=false,bottomline=false]\begin{adtheorem}}
    {\end{adtheorem}\end{mdframed}}
\newtheorem{bdtheorem}{定义}
\newenvironment{definition}
{\begin{mdframed}[backgroundcolor=gray!40,rightline=false,leftline=false,topline=false,bottomline=false]\begin{bdtheorem}}
    {\end{bdtheorem}\end{mdframed}}
\newtheorem*{cdtheorem}{习题}
\newenvironment{exercise}
{\begin{mdframed}[backgroundcolor=gray!40,rightline=false,leftline=false,topline=false,bottomline=false]\begin{cdtheorem}}
    {\end{cdtheorem}\end{mdframed}}
\newtheorem*{ddtheorem}{注}
\newenvironment{remark}
{\begin{mdframed}[backgroundcolor=gray!40,rightline=false,leftline=false,topline=false,bottomline=false]\begin{ddtheorem}}
    {\end{ddtheorem}\end{mdframed}}
\newtheorem{edtheorem}{引理}
\newenvironment{lemma}
{\begin{mdframed}[backgroundcolor=gray!40,rightline=false,leftline=false,topline=false,bottomline=false]\begin{edtheorem}}
    {\end{edtheorem}\end{mdframed}}
% \usepackage{latexdef}
\def\ZZ{\mathbb{Z}} \topmargin -0.40in \oddsidemargin 0.08in
\evensidemargin 0.08in \marginparwidth 0.00in \marginparsep 0.00in
\textwidth 16cm \textheight 24cm \newcommand{\D}{\displaystyle}
\newcommand{\ds}{\displaystyle} \renewcommand{\ni}{\noindent}
\newcommand{\pa}{\partial} \newcommand{\Om}{\Omega}
\newcommand{\om}{\omega} \newcommand{\sik}{\sum_{i=1}^k}
\newcommand{\vov}{\Vert\omega\Vert} \newcommand{\Umy}{U_{\mu_i,y^i}}
\newcommand{\lamns}{\lambda_n^{^{\scriptstyle\sigma}}}
\newcommand{\chiomn}{\chi_{_{\Omega_n}}}
\newcommand{\ullim}{\underline{\lim}} \newcommand{\bsy}{\boldsymbol}
\newcommand{\mvb}{\mathversion{bold}} \newcommand{\la}{\lambda}
\newcommand{\La}{\Lambda} \newcommand{\va}{\varepsilon}
\newcommand{\be}{\beta} \newcommand{\al}{\alpha}
\newcommand{\dis}{\displaystyle} \newcommand{\R}{{\mathbb R}}
\newcommand{\N}{{\mathbb N}} \newcommand{\cF}{{\mathcal F}}
\newcommand{\gB}{{\mathfrak B}} \newcommand{\eps}{\epsilon}
\renewcommand\refname{参考文献} \def \qed {\hfill \vrule height6pt
  width 6pt depth 0pt} \topmargin -0.40in \oddsidemargin 0.08in
\evensidemargin 0.08in \marginparwidth0.00in \marginparsep 0.00in
\textwidth 15.5cm \textheight 24cm \pagestyle{myheadings}
\markboth{\rm \centerline{}} {\rm \centerline{}}
\begin{document}
  \title{{\bf {《常微分方程教程》\footnote{丁同仁,李承治著,第二版,高等
          教育出版社}习题1-1,3}}} \author{{叶卢庆} \\{{ \small{杭州师范
          大学理学院,学
          号:1002011005}}}\\\small{Email:h5411167@gmail.com}}
  \maketitle
  \begin{exercise}[习题1-1,3,(1)]
求出曲线族 $y=Cx+x^2$ 所满足的微分方程.
  \end{exercise}
  \begin{proof}[解]
易得为
\begin{equation}
  \label{eq:8.17pm}
  y''=2.
\end{equation}
\end{proof}
\begin{remark}
  注意 $y=Cx+x^2$ 并非 $y''=2$ 的通解.
\end{remark}
\bigskip
\begin{exercise}[习题1-1,3,(2)]
求曲线族 $y=C_1e^x+C_2xe^x$ 所满足的微分方程.  
\end{exercise}
\begin{proof}[解]
  可得
  \begin{equation}
    \label{eq:9.24pm}
    \begin{cases}
      y=\phi(x,C_{1},C_{2})=C_1e^x+C_2xe^x,\\
y'=\phi'(x,C_{1},C_{2})=C_1e^x+C_2e^x+C_2xe^x
    \end{cases}.
  \end{equation}
且
\begin{equation}
  \label{eq:9.26pm}
  y''=C_1e^x+2C_2e^x+C_2xe^x.
\end{equation}
由式 \eqref{eq:9.24pm} 可得雅可比行列式
\begin{equation}
  \label{eq:9.28pm}
  \begin{vmatrix}
    \frac{\partial \phi}{\partial C_1}&\frac{\partial \phi}{\partial
      C_2}\\
\frac{\partial \phi'}{\partial C_{1}}&\frac{\partial \phi'}{\partial C_2}
  \end{vmatrix}=\begin{vmatrix}
    e^x&xe^x\\
e^x&e^x+xe^x
  \end{vmatrix}=e^{2x}\neq 0.
\end{equation}
因此 $C_1,C_2$ 是独立的.因此可以反解出
\begin{equation}
  \label{eq:9.39pm}
  \begin{cases}
    C_2=\frac{y'-y}{e^x}\\
C_1=\frac{y+xy-xy'}{e^x}\\
  \end{cases}.
\end{equation}
将式 \eqref{eq:9.39pm} 代入式 \eqref{eq:9.26pm},可得
\begin{equation}
  \label{eq:10.05pm}
  y''=2y'-y.
\end{equation}
式 \eqref{eq:10.05pm} 即为曲线族 $y=C_1e^x+C_2xe^x$ 所满足的微分方程.
\end{proof}
\begin{exercise}[习题1-1,3,(3)]
求平面上以原点为中心的一切圆所满足的微分方程.  
\end{exercise}
\begin{proof}[解]
  平面上以原点为中心的任意圆都可以表达成
  \begin{equation}
    \label{eq:10.09pm}
    x^2+y^2=r^2
  \end{equation}
的形式,其中 $r$ 是任意给定的非零实数.将 $y$ 看作 $x$ 的隐函数,则得
\begin{equation}
  \label{eq:10.21pm}
  2x+2yy'=0.
\end{equation}
即
\begin{equation}
  \label{eq:10.22pm}
  x+yy'=0.
\end{equation}
此即以原点为中心的一切圆所满足的微分方程.
\end{proof}
\begin{exercise}[习题1-1,3,(4)]
试求平面上一切圆所满足的微分方程.  
\end{exercise}
\begin{proof}[解]
平面上的任意一个圆的一般方程为
  \begin{equation}
    \label{eq:10.28pm}
    (x-a)^2+(y-b)^2=r^2,r\in \mathbf{R}^{+}.
  \end{equation}
将 $y$ 看作 $x$ 的隐函数,可得
\begin{subequations}\label{hehe}
  \begin{align}
(x-a)^2+(y-b)^2&=r^2\\
(x-a)+(y-b)y'&=0\\
1+y'^2+(y-b)y''&=0
  \end{align}
\end{subequations}
将式 (11b) 两边同时乘以 $y''$,可得
\begin{equation}
  \label{eq:11.05am}
  (x-a)y''+(y-b)y''y'=0.
\end{equation}
将式 (11c) 代入式 \eqref{eq:11.05am} 可得
\begin{equation}
  \label{eq:11.06am}
  (x-a)y''=(1+y'^2)y'.
\end{equation}
将式 (11a) 两边同时乘以 $y''^2$,可得
\begin{equation}
  \label{eq:11.08am}
  [(x-a)y'']^2+[(y-b)y'']^2=r^2y''^2.
\end{equation}
将式 (11c),\eqref{eq:11.06am} 代入 \eqref{eq:11.08am},可得
\begin{equation}
  \label{11.10am}
  (1+y'^2)^{3}=r^2y''^2.
\end{equation}
于是
\begin{equation}
  \label{eq:11.17am}
  \frac{(1+y'^2)^3}{y''^2}=r^2.
\end{equation}
将式子 \eqref{eq:11.17am} 两边对 $x$ 求导,最终可得
\begin{equation}
  \label{eq:11.19am}
 3y'y''^2-(1+y'^2)y'''=0. 
\end{equation}
式 \eqref{eq:11.19am} 就是平面上一切圆所满足的微分方程.
\end{proof}
\end{document}
