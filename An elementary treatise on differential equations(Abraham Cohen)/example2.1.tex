\documentclass[a4paper]{article}
\usepackage{amsmath,amsfonts,amsthm,amssymb}
\usepackage{bm}
\usepackage{euler}
\usepackage{hyperref}
\usepackage{geometry}
\usepackage{yhmath}
\usepackage{pstricks-add}
\usepackage{framed,mdframed}
\usepackage{graphicx,color} 
\usepackage{mathrsfs,xcolor} 
\usepackage[all]{xy}
\usepackage{fancybox} 
\usepackage{xeCJK}
\newtheorem*{theo}{定理}
\newtheorem*{exe}{题目}
\newtheorem*{rem}{评论}
\newmdtheoremenv{lemma}{引理}
\newmdtheoremenv{corollary}{推论}
\newtheorem*{exam}{例}
\newenvironment{theorem}
{\bigskip\begin{mdframed}\begin{theo}}
    {\end{theo}\end{mdframed}\bigskip}
\newenvironment{exercise}
{\bigskip\begin{mdframed}\begin{exe}}
    {\end{exe}\end{mdframed}\bigskip}
\newenvironment{example}
{\bigskip\begin{mdframed}\begin{exam}}
    {\end{exam}\end{mdframed}\bigskip}
\geometry{left=2.5cm,right=2.5cm,top=2.5cm,bottom=2.5cm}
\setCJKmainfont[BoldFont=SimHei]{SimSun}
\renewcommand{\today}{\number\year 年 \number\month 月 \number\day 日}
\newcommand{\D}{\displaystyle}\newcommand{\ri}{\Rightarrow}
\newcommand{\ds}{\displaystyle} \renewcommand{\ni}{\noindent}
\newcommand{\ov}{\overrightarrow}
\newcommand{\pa}{\partial} \newcommand{\Om}{\Omega}
\newcommand{\om}{\omega} \newcommand{\sik}{\sum_{i=1}^k}
\newcommand{\vov}{\Vert\omega\Vert} \newcommand{\Umy}{U_{\mu_i,y^i}}
\newcommand{\lamns}{\lambda_n^{^{\scriptstyle\sigma}}}
\newcommand{\chiomn}{\chi_{_{\Omega_n}}}
\newcommand{\ullim}{\underline{\lim}} \newcommand{\bsy}{\boldsymbol}
\newcommand{\mvb}{\mathversion{bold}} \newcommand{\la}{\lambda}
\newcommand{\La}{\Lambda} \newcommand{\va}{\varepsilon}
\newcommand{\be}{\beta} \newcommand{\al}{\alpha}
\newcommand{\dis}{\displaystyle} \newcommand{\R}{{\mathbb R}}
\newcommand{\N}{{\mathbb N}} \newcommand{\cF}{{\mathcal F}}
\newcommand{\gB}{{\mathfrak B}} \newcommand{\eps}{\epsilon}
\renewcommand\refname{参考文献}\renewcommand\figurename{图}
\usepackage[]{caption2} 
\renewcommand{\captionlabeldelim}{}
\setlength\parindent{0pt}
\begin{document}
\title{\huge{\bf{微分方程初步,例 2.1}}} \author{\small{叶卢庆\footnote{叶卢庆(1992---),男,杭州师范大学理学院数学与应用数学专业本科在读,E-mail:yeluqingmathematics@gmail.com}}}
\maketitle
\begin{example}
由根式求微分方程:
$$
y=c_1e^{a_1x}+c_2e^{a_2x}.
$$
其中$c_1,c_2$为任意常数.
\end{example}
\begin{proof}[\textbf{解}]
$$
\begin{cases}
  y'=c_1a_1e^{a_1x}+c_2a_2e^{a_2x},\\
y''=c_1a_1^2e^{a_1x}+c_2a_2^2e^{a_2x}.
\end{cases}
$$
当 $a_1a_2^{2}e^{(a_1+a_{2})x}-a_2a_{1}^{2}e^{(a_2+a_{1})x}\neq 0$,即
$a_1a_2(a_2-a_1)\neq 0$时,$c_1$和$c_2$有唯一解.解得
$$
c_1=\frac{a_1y'-y''}{a_1a_2e^{a_2x}-a_2^2e^{a_2x}},c_2=\frac{a_2y'-y''}{a_1a_2e^{a_1x}-a_1^2e^{a_1x}}.
$$
将$c_1,c_2$代回原式即可得到相应的微分方程.而当
$a_1=a_2\neq 0$时,$y=(c_1+c_2)e^{a_1x}$,此时,$y'=(c_1+c_2)a_1e^{a_1x}$,于是,
此时微分方程是$a_1y=y'$.当$a_1=0$时,
$$
y=c_1+c_2e^{a_2x},
$$
此时$y'=c_2a_2e^{a_2x}$,因此$a_2y'=y''$.类似地讨论$a_2=0$的情形.
\end{proof}
\end{document}
