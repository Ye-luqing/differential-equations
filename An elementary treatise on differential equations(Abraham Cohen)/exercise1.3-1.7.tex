\documentclass[a4paper]{article}
\usepackage{amsmath,amsfonts,amsthm,amssymb}
\usepackage{bm}
\usepackage{hyperref}
\usepackage{geometry}
\usepackage{yhmath}
\usepackage{pstricks-add}
\usepackage{framed,mdframed}
\usepackage{graphicx,color} 
\usepackage{mathrsfs,xcolor} 
\usepackage[all]{xy}
\usepackage{fancybox} 
\usepackage{xeCJK}
\newtheorem*{theo}{定理}
\newtheorem*{exe}{Exercise}
\newtheorem*{rem}{评论}
\newmdtheoremenv{lemma}{引理}
\newmdtheoremenv{corollary}{推论}
\newtheorem*{exa}{例}
\newenvironment{theorem}
{\bigskip\begin{mdframed}\begin{theo}}
    {\end{theo}\end{mdframed}\bigskip}
\newenvironment{exercise}
{\bigskip\begin{mdframed}\begin{exe}}
    {\end{exe}\end{mdframed}\bigskip}
\newenvironment{example}
{\bigskip\begin{mdframed}\begin{exa}}
    {\end{exa}\end{mdframed}\bigskip}
\geometry{left=2.5cm,right=2.5cm,top=2.5cm,bottom=2.5cm}
\setCJKmainfont[BoldFont=SimHei]{SimSun}
\newcommand{\D}{\displaystyle}\newcommand{\ri}{\Rightarrow}
\newcommand{\ds}{\displaystyle} \renewcommand{\ni}{\noindent}
\newcommand{\ov}{\overrightarrow}
\newcommand{\pa}{\partial} \newcommand{\Om}{\Omega}
\newcommand{\om}{\omega} \newcommand{\sik}{\sum_{i=1}^k}
\newcommand{\vov}{\Vert\omega\Vert} \newcommand{\Umy}{U_{\mu_i,y^i}}
\newcommand{\lamns}{\lambda_n^{^{\scriptstyle\sigma}}}
\newcommand{\chiomn}{\chi_{_{\Omega_n}}}
\newcommand{\ullim}{\underline{\lim}} \newcommand{\bsy}{\boldsymbol}
\newcommand{\mvb}{\mathversion{bold}} \newcommand{\la}{\lambda}
\newcommand{\La}{\Lambda} \newcommand{\va}{\varepsilon}
\newcommand{\be}{\beta} \newcommand{\al}{\alpha}
\newcommand{\dis}{\displaystyle} \newcommand{\R}{{\mathbb R}}
\newcommand{\N}{{\mathbb N}} \newcommand{\cF}{{\mathcal F}}
\newcommand{\gB}{{\mathfrak B}} \newcommand{\eps}{\epsilon}
\renewcommand\refname{参考文献}\renewcommand\figurename{图}
\usepackage[]{caption2} 
\renewcommand{\captionlabeldelim}{}
\setlength\parindent{0pt}
\begin{document}
\title{\huge{\bf{Exercise1.3-1.7}}} \author{\small{Luqing Ye\footnote{叶卢庆(1992---),男,杭州师范大学理学院数学与应用数学专业本科在读,E-mail:yeluqingmathematics@gmail.com}}}
\maketitle
Find the differential equations corresponding to the following
primitives.
\begin{exercise}[1.3]

$$
y=cx+\sqrt{1-c^2}.
$$
\end{exercise}
\begin{proof}[\textbf{Solve}]
$$
y'=x.
$$  
\end{proof}
\begin{exercise}[1.4]
$$
(x-c_1)^2+(y-c_2)^2=r^2.
$$
\end{exercise}
\begin{proof}[\textbf{Solve}]
$$
(x-c_1)+(y-c_2)\frac{dy}{dx}=0.
$$
$$
1+\left(\frac{dy}{dx}\right)^{2}+(y-c_{2}) \frac{d^{2}y}{dx^2}=0.
$$
$$
2\left(\frac{dy}{dx}\right)\frac{d^2y}{dx^2}+\frac{d^2y}{dx^2}+(y-c_2)\frac{d^3y}{dx^3}=0.
$$
So 
$$
-\frac{d^3y}{dx^3}-\frac{d^3y}{dx^3}\left(\frac{dy}{dx}\right)^2+2\left(\frac{dy}{dx}\right)\left(\frac{d^2y}{dx^2}\right)^{2}+\left(\frac{d^2y}{dx^{2}}\right)^{2}=0.
$$
\end{proof}
\begin{exercise}[1.5]
$$
y=c_1x^2+c_2.
$$
\end{exercise}
\begin{proof}[\textbf{Solve}]
$$
y'=2c_1x,y''=2c_1.
$$
So $y'=y''x$.
\end{proof}
\begin{exercise}[1.6]
  $$ y^2+c_1x=0 .$$
\end{exercise}
\begin{proof}[\textbf{Solve}]
$$
2yy'+c_1=0,
$$
So
$$
y^2-2yy'x=0.
$$
\end{proof}
\begin{exercise}[1.7]
$$
x^2=2cy+c^2.
$$
\end{exercise}
\begin{proof}[\textbf{Solve}]
$$
2x=2cy',
$$
So
$$
x^2y'^{2}=2cy'yy'+c^2y'^2=2xyy'+x^2.
$$
\end{proof}
\end{document}
