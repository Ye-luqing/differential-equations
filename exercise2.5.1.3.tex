\documentclass[a4paper, 12pt]{article} % Font size (can be 10pt, 11pt or 12pt) and paper size (remove a4paper for US letter paper)
\usepackage{amsmath,amsfonts,bm}
\usepackage{hyperref,verbatim}
\usepackage{amsthm,epigraph} 
\usepackage{amssymb}
\usepackage{framed,mdframed}
\usepackage{graphicx,color} 
\usepackage{mathrsfs,xcolor} 
\usepackage[all]{xy}
\usepackage{fancybox} 
\usepackage{xeCJK}
\newtheorem*{adtheorem}{定理}
\setCJKmainfont[BoldFont=FZYaoTi,ItalicFont=FZYaoTi]{FZYaoTi}
\definecolor{shadecolor}{rgb}{1.0,0.9,0.9} %背景色为浅红色
\newenvironment{theorem}
{\bigskip\begin{mdframed}[backgroundcolor=gray!40,rightline=false,leftline=false,topline=false,bottomline=false]\begin{adtheorem}}
    {\end{adtheorem}\end{mdframed}\bigskip}
\newtheorem*{bdtheorem}{定义}
\newenvironment{definition}
{\bigskip\begin{mdframed}[backgroundcolor=gray!40,rightline=false,leftline=false,topline=false,bottomline=false]\begin{bdtheorem}}
    {\end{bdtheorem}\end{mdframed}\bigskip}
\newtheorem*{cdtheorem}{习题}
\newenvironment{exercise}
{\bigskip\begin{mdframed}[backgroundcolor=gray!40,rightline=false,leftline=false,topline=false,bottomline=false]\begin{cdtheorem}}
    {\end{cdtheorem}\end{mdframed}\bigskip}
\newtheorem*{ddtheorem}{注}
\newenvironment{remark}
{\bigskip\begin{mdframed}[backgroundcolor=gray!40,rightline=false,leftline=false,topline=false,bottomline=false]\begin{ddtheorem}}
    {\end{ddtheorem}\end{mdframed}\bigskip}
\newtheorem*{edtheorem}{引理}
\newenvironment{lemma}
{\bigskip\begin{mdframed}[backgroundcolor=gray!40,rightline=false,leftline=false,topline=false,bottomline=false]\begin{edtheorem}}
    {\end{edtheorem}\end{mdframed}\bigskip}
\newtheorem*{pdtheorem}{例}
\newenvironment{example}
{\bigskip\begin{mdframed}[backgroundcolor=gray!40,rightline=false,leftline=false,topline=false,bottomline=false]\begin{pdtheorem}}
    {\end{pdtheorem}\end{mdframed}\bigskip}

\usepackage[protrusion=true,expansion=true]{microtype} % Better typography
\usepackage{wrapfig} % Allows in-line images
\usepackage{mathpazo} % Use the Palatino font
\usepackage[T1]{fontenc} % Required for accented characters
\linespread{1.05} % Change line spacing here, Palatino benefits from a slight increase by default

\makeatletter
\renewcommand\@biblabel[1]{\textbf{#1.}} % Change the square brackets for each bibliography item from '[1]' to '1.'
\renewcommand{\@listI}{\itemsep=0pt} % Reduce the space between items in the itemize and enumerate environments and the bibliography

\renewcommand{\maketitle}{ % Customize the title - do not edit title
  % and author name here, see the TITLE block
  % below
  \renewcommand\refname{参考文献}
  \newcommand{\D}{\displaystyle}\newcommand{\ri}{\Rightarrow}
  \newcommand{\ds}{\displaystyle} \renewcommand{\ni}{\noindent}
  \newcommand{\pa}{\partial} \newcommand{\Om}{\Omega}
  \newcommand{\om}{\omega} \newcommand{\sik}{\sum_{i=1}^k}
  \newcommand{\vov}{\Vert\omega\Vert} \newcommand{\Umy}{U_{\mu_i,y^i}}
  \newcommand{\lamns}{\lambda_n^{^{\scriptstyle\sigma}}}
  \newcommand{\chiomn}{\chi_{_{\Omega_n}}}
  \newcommand{\ullim}{\underline{\lim}} \newcommand{\bsy}{\boldsymbol}
  \newcommand{\mvb}{\mathversion{bold}} \newcommand{\la}{\lambda}
  \newcommand{\La}{\Lambda} \newcommand{\va}{\varepsilon}
  \newcommand{\be}{\beta} \newcommand{\al}{\alpha}
  \newcommand{\dis}{\displaystyle} \newcommand{\R}{{\mathbb R}}
  \newcommand{\N}{{\mathbb N}} \newcommand{\cF}{{\mathcal F}}
  \newcommand{\gB}{{\mathfrak B}} \newcommand{\eps}{\epsilon}
  \begin{flushright} % Right align
    {\LARGE\@title} % Increase the font size of the title
    
    \vspace{50pt} % Some vertical space between the title and author name
    
    {\large\@author} % Author name
    \\\@date % Date
    
    \vspace{40pt} % Some vertical space between the author block and abstract
  \end{flushright}
}

% ----------------------------------------------------------------------------------------
%	TITLE
% ----------------------------------------------------------------------------------------
\begin{document}
\title{\textbf{习题2.5.1.3}} 
% \setlength\epigraphwidth{0.7\linewidth}
\author{\small{叶卢庆}\\{\small{杭州师范大学理学院,学号:1002011005}}\\{\small{Email:h5411167@gmail.com}}} % Institution
\renewcommand{\today}{\number\year. \number\month. \number\day}
\date{\today} % Date

% ----------------------------------------------------------------------------------------


\maketitle % Print the title section

% ----------------------------------------------------------------------------------------
%	ABSTRACT AND KEYWORDS
% ----------------------------------------------------------------------------------------

% \renewcommand{\abstractname}{摘要} % Uncomment to change the name of the abstract to something else

% \begin{abstract}

% \end{abstract}

% \hspace*{3,6mm}\textit{关键词:}  % Keywords

% \vspace{30pt} % Some vertical space between the abstract and first section

% ----------------------------------------------------------------------------------------
%	ESSAY BODY
% ----------------------------------------------------------------------------------------
\begin{exercise}[2.5.1.3]
$$
(3x+\frac{6}{y})dx+(\frac{x^2}{y}+\frac{3y}{x})dy=0.
$$
\end{exercise}
\begin{proof}[解]
两边同时乘以 $xy$,可得
\begin{equation}
  \label{eq:1}
  (3x^2y+6x)dx+(x^3+3y^2)dy=0.
\end{equation}
易得
$$
\frac{\pa (3x^2y+6x)}{y}=3x^2,\frac{\pa (x^3+3y^2)}{\pa x}=3x^2.
$$
因此 \eqref{eq:1} 是恰当微分方程.易得通积分为
$$
x^3y+y^3+3x^{2}+C=0.
$$
其中 $x,y\neq 0$.
\begin{comment}
先分组,得到微分方程
\begin{equation}
  \label{eq:1}
  3xdx+\frac{x^2}{y}dy=0.
\end{equation}
可得积分因子为 $\frac{1}{x^2}$,乘以积分因子后得到微分方程
$$
\frac{3}{x}dx+\frac{1}{y}dy=0.
$$
可得通积分为
$$
3\ln |x|+\ln |y|+C=0,
$$
即
$$
\ln |x|^3|y|=-C.
$$
易得 $\frac{1}{x^2}g(|x|^{3}|y|)$ 也是微分方程 \eqref{eq:1} 的积分因子,其
中 $g$ 是可微函数.我们还有微分方程
\begin{equation}
  \label{eq:2}
  \frac{6}{y}dx+\frac{3y}{x}dy=0,
\end{equation}
可得积分因子为 $xy$,乘以积分因子后得到微分方程
$$
6xdx+3y^2dy=0.
$$
可得通积分为
$$
3x^2+\frac{3}{2}y^2+C=0.
$$
易得 $xyh(2x^2+y^2)$ 也是微分方程 \eqref{eq:2} 的积分因子,其中 $h$ 是
可微函数.我们希望,
$$
\frac{1}{x^2}g(|x|^3|y|)=xyh(2x^2+y^2).
$$
令
$$
g(|x|^3|y|)=|x|^3|y|,h(2x^2+y^2)=\frac{|xy|}{xy}.
$$
\end{comment}
\begin{comment}
两边同时乘以非零的函数 $u(x,y)$,得到
$$
u(x,y)(3x+\frac{6}{y})dx+u(x,y)(\frac{x^2}{y}+\frac{3y}{x})dy=0.
$$
我们希望这是一个恰当微分方程,即
$$
\frac{\pa [u(x,y)(3x+\frac{6}{y})]}{\pa y}=\frac{\pa [u(x,y)(\frac{x^2}{y}+\frac{3y}{x})]}{dx}.
$$
也即
$$
\frac{\pa u}{\pa y}(3x+\frac{6}{y})-6u \frac{1}{y^2}=\frac{\pa u}{\pa x}(\frac{x^2}{y}+\frac{3y}{x})+u(\frac{2x}{y}-\frac{3y}{x^2}).
$$
\end{comment}
\end{proof}
% ----------------------------------------------------------------------------------------
%	BIBLIOGRAPHY
% ----------------------------------------------------------------------------------------

\bibliographystyle{unsrt}

\bibliography{sample}

% ----------------------------------------------------------------------------------------
\end{document}