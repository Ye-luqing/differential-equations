\documentclass[a4paper, 12pt]{article} % Font size (can be 10pt, 11pt or 12pt) and paper size (remove a4paper for US letter paper)
\usepackage{amsmath,amsfonts,bm}
\usepackage{hyperref}
\usepackage{amsthm} 
\usepackage{amssymb}
\usepackage{framed,mdframed}
\usepackage{graphicx,color} 
\usepackage{mathrsfs,xcolor} 
\usepackage[all]{xy}
\usepackage{fancybox} 
\usepackage{xeCJK}
\newtheorem*{adtheorem}{定理}
\setCJKmainfont[BoldFont=FZYaoTi,ItalicFont=FZYaoTi]{FZYaoTi}
\definecolor{shadecolor}{rgb}{1.0,0.9,0.9} %背景色为浅红色
\newenvironment{theorem}
{\bigskip\begin{mdframed}[backgroundcolor=gray!40,rightline=false,leftline=false,topline=false,bottomline=false]\begin{adtheorem}}
    {\end{adtheorem}\end{mdframed}\bigskip}
\newtheorem*{bdtheorem}{定义}
\newenvironment{definition}
{\bigskip\begin{mdframed}[backgroundcolor=gray!40,rightline=false,leftline=false,topline=false,bottomline=false]\begin{bdtheorem}}
    {\end{bdtheorem}\end{mdframed}\bigskip}
\newtheorem*{cdtheorem}{习题}
\newenvironment{exercise}
{\bigskip\begin{mdframed}[backgroundcolor=gray!40,rightline=false,leftline=false,topline=false,bottomline=false]\begin{cdtheorem}}
    {\end{cdtheorem}\end{mdframed}\bigskip}
\newtheorem*{ddtheorem}{注}
\newenvironment{remark}
{\bigskip\begin{mdframed}[backgroundcolor=gray!40,rightline=false,leftline=false,topline=false,bottomline=false]\begin{ddtheorem}}
    {\end{ddtheorem}\end{mdframed}\bigskip}
\newtheorem*{edtheorem}{引理}
\newenvironment{lemma}
{\bigskip\begin{mdframed}[backgroundcolor=gray!40,rightline=false,leftline=false,topline=false,bottomline=false]\begin{edtheorem}}
    {\end{edtheorem}\end{mdframed}\bigskip}
\newtheorem*{pdtheorem}{例}
\newenvironment{example}
{\bigskip\begin{mdframed}[backgroundcolor=gray!40,rightline=false,leftline=false,topline=false,bottomline=false]\begin{pdtheorem}}
    {\end{pdtheorem}\end{mdframed}\bigskip}

\usepackage[protrusion=true,expansion=true]{microtype} % Better typography
\usepackage{wrapfig} % Allows in-line images
\usepackage{mathpazo} % Use the Palatino font
\usepackage[T1]{fontenc} % Required for accented characters
\linespread{1.05} % Change line spacing here, Palatino benefits from a slight increase by default

\makeatletter
\renewcommand\@biblabel[1]{\textbf{#1.}} % Change the square brackets for each bibliography item from '[1]' to '1.'
\renewcommand{\@listI}{\itemsep=0pt} % Reduce the space between items in the itemize and enumerate environments and the bibliography

\renewcommand{\maketitle}{ % Customize the title - do not edit title
  % and author name here, see the TITLE block
  % below
  \renewcommand\refname{参考文献}
  \newcommand{\D}{\displaystyle}\newcommand{\ri}{\Rightarrow}
  \newcommand{\ds}{\displaystyle} \renewcommand{\ni}{\noindent}
  \newcommand{\pa}{\partial} \newcommand{\Om}{\Omega}
  \newcommand{\om}{\omega} \newcommand{\sik}{\sum_{i=1}^k}
  \newcommand{\vov}{\Vert\omega\Vert} \newcommand{\Umy}{U_{\mu_i,y^i}}
  \newcommand{\lamns}{\lambda_n^{^{\scriptstyle\sigma}}}
  \newcommand{\chiomn}{\chi_{_{\Omega_n}}}
  \newcommand{\ullim}{\underline{\lim}} \newcommand{\bsy}{\boldsymbol}
  \newcommand{\mvb}{\mathversion{bold}} \newcommand{\la}{\lambda}
  \newcommand{\La}{\Lambda} \newcommand{\va}{\varepsilon}
  \newcommand{\be}{\beta} \newcommand{\al}{\alpha}
  \newcommand{\dis}{\displaystyle} \newcommand{\R}{{\mathbb R}}
  \newcommand{\N}{{\mathbb N}} \newcommand{\cF}{{\mathcal F}}
  \newcommand{\gB}{{\mathfrak B}} \newcommand{\eps}{\epsilon}
  \begin{flushright} % Right align
    {\LARGE\@title} % Increase the font size of the title
    
    \vspace{50pt} % Some vertical space between the title and author name
    
    {\large\@author} % Author name
    \\\@date % Date
    
    \vspace{40pt} % Some vertical space between the author block and abstract
  \end{flushright}
}

% ----------------------------------------------------------------------------------------
%	TITLE
% ----------------------------------------------------------------------------------------

\title{\textbf{《常微分方程教程》习题2.3.6}} 

\author{\small{叶卢庆}\\{\small{杭州师范大学理学院,学号:1002011005}}\\{\small{Email:h5411167@gmail.com}}} % Institution
\renewcommand{\today}{\number\year. \number\month. \number\day}
\date{\today} % Date

% ----------------------------------------------------------------------------------------

\begin{document}
\maketitle % Print the title section

% ----------------------------------------------------------------------------------------
%	ABSTRACT AND KEYWORDS
% ----------------------------------------------------------------------------------------

% \renewcommand{\abstractname}{摘要} % Uncomment to change the name of the abstract to something else

% \begin{abstract}

% \end{abstract}

% \hspace*{3,6mm}\textit{关键词:}  % Keywords

% \vspace{30pt} % Some vertical space between the abstract and first section

% ----------------------------------------------------------------------------------------
%	ESSAY BODY
% ----------------------------------------------------------------------------------------
\begin{exercise}[2.3.6]
设连续函数 $f(x)$ 在区间 $-\infty<x<+\infty$ 上有界.证明:方程
$$
y'+y=f(x)
$$
在区间 $-\infty<x<+\infty$ 上有且仅有一个有界解.试求出这个有界解,并进
而证明:当 $f(x)$ 还是以 $w$ 为周期的周期函数时,这个有界解也是一个以
$w$ 为周期的周期函数.
\end{exercise}
\begin{proof}[证明]
我们用反证法.假如方程有两个有界解 $y_1(x),y_2(x)$,即
\begin{equation}
  \label{eq:1}
  y_1'(x)+y_1(x)=f(x),
\end{equation}
\begin{equation}
  \label{eq:2}
  y_2'(x)+y_2(x)=f(x),
\end{equation}
则可得
\begin{equation}
  \label{eq:3}
  (y_1(x)-y_2(x))'=-(y_1(x)-y_2(x)).
\end{equation}
设 $p(x)=y_1(x)-y_2(x)$,则
\begin{equation}
  \label{eq:4}
  p'(x)=-p(x)\ri p(x)=ce^{-x}.
\end{equation}
由于 $\exists x_0\in \mathbf{R}$,使得 $y_1(x_0)\neq y_2(x_0)$,因此
$c\neq 0$.令 $x\to -\infty$,可得 $|p(x)|=|y_1(x)-y_2(x)|\to \infty$,这
与 $y_1(x),y_2(x)$ 有界矛盾.可见只可能存在一个有界解.下面求出这个有界解 $y(x)$.把题目中的微分方程化为
\begin{equation}
  \label{eq:11}
  dy+(y-f(x))dx=0.
\end{equation}
方程两边同时乘以积分因子 $e^x$,得到恰当微分方程
\begin{equation}
  \label{eq:12}
  e^xdy+e^x(y-f(x))dx=0.
\end{equation}
设存在二元函数 $\phi(x,y)$,使得
$$
\frac{\pa\phi}{\pa y}=e^x\ri \phi=ye^x+h(x).
$$
因此
$$
ye^x+h'(x)=ye^x-f(x)e^x.\ri h'(x)=-f(x)e^x.
$$
于是得到
$$
e^xy+C=\int f(x)e^xdx.
$$
则
$$
e^{x_a}y_a-e^{x_b}y_b=\int_{x_b}^{x_a}f(x)e^xdx.
$$
令 $x_b\to -\infty$,由于 $y$ 有界,因此 $-e^{x_b}y_{b}\to 0$,由于
$f(x)$ 有界,因此
$$
\int_{-\infty}^{x_a}f(x)e^xdx
$$
是一个实数.可见,
$$
y_a=\frac{\int_{-\infty}^{x_{a}}f(x)e^{x}dx}{e^{x_a}}.
$$
至于有界解和 $f(x)$ 有同样的周期从如上公式可以显然看出.
\end{proof}
% ----------------------------------------------------------------------------------------
%	BIBLIOGRAPHY
% ----------------------------------------------------------------------------------------

\bibliographystyle{unsrt}

\bibliography{sample}

% ----------------------------------------------------------------------------------------
\end{document}