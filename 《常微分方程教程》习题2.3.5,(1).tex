\documentclass[a4paper, 12pt]{article} % Font size (can be 10pt, 11pt or 12pt) and paper size (remove a4paper for US letter paper)
\usepackage{amsmath,amsfonts,bm}
\usepackage{hyperref}
\usepackage{amsthm} 
\usepackage{amssymb}
\usepackage{framed,mdframed}
\usepackage{graphicx,color} 
\usepackage{mathrsfs,xcolor} 
\usepackage[all]{xy}
\usepackage{fancybox} 
\usepackage{xeCJK}
\newtheorem*{adtheorem}{定理}
\setCJKmainfont[BoldFont=Adobe Song Std,ItalicFont=Adobe Song Std]{Adobe Song Std}
\definecolor{shadecolor}{rgb}{1.0,0.9,0.9} %背景色为浅红色
\newenvironment{theorem}
{\bigskip\begin{mdframed}[backgroundcolor=gray!40,rightline=false,leftline=false,topline=false,bottomline=false]\begin{adtheorem}}
    {\end{adtheorem}\end{mdframed}\bigskip}
\newtheorem*{bdtheorem}{定义}
\newenvironment{definition}
{\bigskip\begin{mdframed}[backgroundcolor=gray!40,rightline=false,leftline=false,topline=false,bottomline=false]\begin{bdtheorem}}
    {\end{bdtheorem}\end{mdframed}\bigskip}
\newtheorem*{cdtheorem}{习题}
\newenvironment{exercise}
{\bigskip\begin{mdframed}[backgroundcolor=gray!40,rightline=false,leftline=false,topline=false,bottomline=false]\begin{cdtheorem}}
    {\end{cdtheorem}\end{mdframed}\bigskip}
\newtheorem*{ddtheorem}{注}
\newenvironment{remark}
{\bigskip\begin{mdframed}[backgroundcolor=gray!40,rightline=false,leftline=false,topline=false,bottomline=false]\begin{ddtheorem}}
    {\end{ddtheorem}\end{mdframed}\bigskip}
\newtheorem*{edtheorem}{引理}
\newenvironment{lemma}
{\bigskip\begin{mdframed}[backgroundcolor=gray!40,rightline=false,leftline=false,topline=false,bottomline=false]\begin{edtheorem}}
    {\end{edtheorem}\end{mdframed}\bigskip}
\newtheorem*{pdtheorem}{例}
\newenvironment{example}
{\bigskip\begin{mdframed}[backgroundcolor=gray!40,rightline=false,leftline=false,topline=false,bottomline=false]\begin{pdtheorem}}
    {\end{pdtheorem}\end{mdframed}\bigskip}

\usepackage[protrusion=true,expansion=true]{microtype} % Better typography
\usepackage{wrapfig} % Allows in-line images
\usepackage{mathpazo} % Use the Palatino font
\usepackage[T1]{fontenc} % Required for accented characters
\linespread{1.05} % Change line spacing here, Palatino benefits from a slight increase by default

\makeatletter
\renewcommand\@biblabel[1]{\textbf{#1.}} % Change the square brackets for each bibliography item from '[1]' to '1.'
\renewcommand{\@listI}{\itemsep=0pt} % Reduce the space between items in the itemize and enumerate environments and the bibliography

\renewcommand{\maketitle}{ % Customize the title - do not edit title
  % and author name here, see the TITLE block
  % below
  \renewcommand\refname{参考文献}
  \newcommand{\D}{\displaystyle}\newcommand{\ri}{\Rightarrow}
  \newcommand{\ds}{\displaystyle} \renewcommand{\ni}{\noindent}
  \newcommand{\pa}{\partial} \newcommand{\Om}{\Omega}
  \newcommand{\om}{\omega} \newcommand{\sik}{\sum_{i=1}^k}
  \newcommand{\vov}{\Vert\omega\Vert} \newcommand{\Umy}{U_{\mu_i,y^i}}
  \newcommand{\lamns}{\lambda_n^{^{\scriptstyle\sigma}}}
  \newcommand{\chiomn}{\chi_{_{\Omega_n}}}
  \newcommand{\ullim}{\underline{\lim}} \newcommand{\bsy}{\boldsymbol}
  \newcommand{\mvb}{\mathversion{bold}} \newcommand{\la}{\lambda}
  \newcommand{\La}{\Lambda} \newcommand{\va}{\varepsilon}
  \begin{flushright} % Right align
    {\LARGE\@title} % Increase the font size of the title
    
    \vspace{50pt} % Some vertical space between the title and author name
    
    {\large\@author} % Author name
    \\\@date % Date
    
    \vspace{40pt}
  \end{flushright}
}

\title{\textbf{《常微分方程教程》习题2.3.5,(1)}} 

\author{\small{叶卢庆}\\{\small{杭州师范大学理学院,学号:1002011005}}\\{\small{Email:h5411167@gmail.com}}}
\renewcommand{\today}{\number\year. \number\month. \number\day}
\date{\today}


\begin{document}
\maketitle
\begin{exercise}[2.3.5,(1)]
  考虑方程
  \begin{equation}
    \label{eq:1}
    \frac{dy}{dx}+p(x)y=q(x),
  \end{equation}
  其中 $p(x),q(x)$ 都是以 $w>0$ 为周期的连续函数.试证:若 $q(x)\equiv
  0$,则方程 \eqref{eq:1} 的任一非零解以 $w$ 为周期,当且仅当函
  数 $p(x)$的平均值
$$
\frac{1}{w}\int_0^wp(x)dx=0.
$$
\end{exercise}
\begin{proof}[证明]
  我们知道,方程 \eqref{eq:1} 的通解为
  \begin{equation}
    \label{eq:2}
    y=e^{-\int p(x)dx}\left(C+\int q(x)e^{\int p(x)dx}dx\right),
  \end{equation}
  其中 $C$ 是任一常数.当 $q(x)\equiv 0$ 时,\eqref{eq:2} 变为
  \begin{equation}
    \label{eq:3}
    y=Ce^{-\int p(x)dx}.
  \end{equation}
  设 $H(x)=\int p(x)dx$,可得
$$
y_0H(0)=C,y_wH(w)=C.
$$
当解 \eqref{eq:3} 以 $w$ 为周期时,可得 $y_0=y_w$,因此
$$
H(w)-H(0)=0,
$$
即
$$
\frac{1}{w}\int_0^wp(x)dx=0.
$$
\\\\
而当
$$
\frac{1}{w}\int_0^wp(x)dx=0
$$
时,鉴于 $p(x)$ 的周期性,说明
$$
\int_t^{t+w}p(x)dx=0.
$$
(为什么?注:画个图有助于理解.)因此
$$
H(t+w)-H(t)=0\ri H(t+w)=H(t).
$$
我们知道,
$$
y_tH(t)=C,y_{t+w}H(t+w)=C.
$$
于是
$$
y_tH(t)=y_{t+w}H(t+w).
$$
当 $H(t)=H(t+w)\neq 0$ 时,
$y_t=y_{t+w}$.当 $H(t)=H(t+w)=0$ 时,可得$C=0$,因此 $y=0$,照样
有 $y_t=y_{t+w}$.
\end{proof}

%\begin{exercise}[2.3.5,(2)]
%  若 $q(x)$ 不恒为0,则方程 \eqref{eq:1} 有唯一的 $w$ 周期解,当且仅当
%$$
%\frac{1}{w}\int_0^wp(x)dx\neq 0.
%$$
%试求出此解.
%\end{exercise}
%\begin{proof}[解]
%  我们知道,方程 \eqref{eq:1} 的通解为
%  \begin{equation}
%    \label{eq:4}
%    y=e^{-\int p(x)dx}\left(C+\int q(x)e^{\int p(x)dx}dx\right),
%  \end{equation}
%  其中 $C$ 是任一常数.设 $H(x)=\int p(x)dx$,$K(x)=\int q(x)e^{\int
%    p(x)dx}dx$.当函数 \eqref{eq:4} 是周期函数的时候,我们知道,
%  \begin{equation}
%    \label{eq:5}
%    y_te^{H(t)}=C+K(t).
%  \end{equation}
%  \begin{equation}
%    \label{eq:6}
%    y_{t+w}e^{H(t+w)}=C+K(t+w).
%  \end{equation}
%  可见
%  \begin{equation}
%    \label{eq:7}
%    y_{t+w}e^{H(t+w)}-y_te^{H(t)}=K(t+w)-K(t)
%  \end{equation}
%  \eqref{eq:7} 即
%  \begin{equation}
%    \label{eq:8}
%    y_te^{H(t)}e^{\int_t^{t+w}p(x)dx}=\int_t^{t+w}q(x)e^{\int p(x)dx}dx.
%  \end{equation}
%  \eqref{eq:8} 即
%  \begin{equation}
%    \label{eq:9}
%    y_t=\frac{\int_t^{t+w}q(x)e^{\int p(x)dx}dx}{e^{H(t)}e^{\int _0^wp(x)dx}}.
%  \end{equation}
%易得
%\begin{equation}
%  \label{eq:10}
%  \int_t^{t+w}q(x)e^{\int p(x)dx}dx=\int_0^wq(x)e^{\int
%    p(x)dx}dx.(\mbox{因为}q(x),p(x)\mbox{的周期性.})
%\end{equation}
%\end{proof}
\bibliographystyle{unsrt}

\bibliography{sample}


\end{document}