\documentclass[a4paper, 12pt]{article} % Font size (can be 10pt, 11pt or 12pt) and paper size (remove a4paper for US letter paper)
\usepackage{amsmath,amsfonts,bm}
\usepackage{hyperref}
\usepackage{amsthm} 
\usepackage{amssymb}
\usepackage{framed,mdframed}
\usepackage{graphicx,color} 
\usepackage{mathrsfs,xcolor} 
\usepackage[all]{xy}
\usepackage{fancybox} 
\usepackage{xeCJK}
\newtheorem*{adtheorem}{定理}
\setCJKmainfont[BoldFont=FZYaoTi,ItalicFont=FZYaoTi]{FZYaoTi}
\definecolor{shadecolor}{rgb}{1.0,0.9,0.9} %背景色为浅红色
\newenvironment{theorem}
{\bigskip\begin{mdframed}[backgroundcolor=gray!40,rightline=false,leftline=false,topline=false,bottomline=false]\begin{adtheorem}}
    {\end{adtheorem}\end{mdframed}\bigskip}
\newtheorem*{bdtheorem}{定义}
\newenvironment{definition}
{\bigskip\begin{mdframed}[backgroundcolor=gray!40,rightline=false,leftline=false,topline=false,bottomline=false]\begin{bdtheorem}}
    {\end{bdtheorem}\end{mdframed}\bigskip}
\newtheorem*{cdtheorem}{习题}
\newenvironment{exercise}
{\bigskip\begin{mdframed}[backgroundcolor=gray!40,rightline=false,leftline=false,topline=false,bottomline=false]\begin{cdtheorem}}
    {\end{cdtheorem}\end{mdframed}\bigskip}
\newtheorem*{ddtheorem}{注}
\newenvironment{remark}
{\bigskip\begin{mdframed}[backgroundcolor=gray!40,rightline=false,leftline=false,topline=false,bottomline=false]\begin{ddtheorem}}
    {\end{ddtheorem}\end{mdframed}\bigskip}
\newtheorem*{edtheorem}{引理}
\newenvironment{lemma}
{\bigskip\begin{mdframed}[backgroundcolor=gray!40,rightline=false,leftline=false,topline=false,bottomline=false]\begin{edtheorem}}
    {\end{edtheorem}\end{mdframed}\bigskip}
\newtheorem*{pdtheorem}{例}
\newenvironment{example}
{\bigskip\begin{mdframed}[backgroundcolor=gray!40,rightline=false,leftline=false,topline=false,bottomline=false]\begin{pdtheorem}}
    {\end{pdtheorem}\end{mdframed}\bigskip}

\usepackage[protrusion=true,expansion=true]{microtype} % Better typography
\usepackage{wrapfig} % Allows in-line images
\usepackage{mathpazo} % Use the Palatino font
\usepackage[T1]{fontenc} % Required for accented characters
\linespread{1.05} % Change line spacing here, Palatino benefits from a slight increase by default

\makeatletter
\renewcommand\@biblabel[1]{\textbf{#1.}} % Change the square brackets for each bibliography item from '[1]' to '1.'
\renewcommand{\@listI}{\itemsep=0pt} % Reduce the space between items in the itemize and enumerate environments and the bibliography

\renewcommand{\maketitle}{ % Customize the title - do not edit title
  % and author name here, see the TITLE block
  % below
  \renewcommand\refname{参考文献}
  \newcommand{\D}{\displaystyle}\newcommand{\ri}{\Rightarrow}
  \newcommand{\ds}{\displaystyle} \renewcommand{\ni}{\noindent}
  \newcommand{\pa}{\partial} \newcommand{\Om}{\Omega}
  \newcommand{\om}{\omega} \newcommand{\sik}{\sum_{i=1}^k}
  \newcommand{\vov}{\Vert\omega\Vert} \newcommand{\Umy}{U_{\mu_i,y^i}}
  \newcommand{\lamns}{\lambda_n^{^{\scriptstyle\sigma}}}
  \newcommand{\chiomn}{\chi_{_{\Omega_n}}}
  \newcommand{\ullim}{\underline{\lim}} \newcommand{\bsy}{\boldsymbol}
  \newcommand{\mvb}{\mathversion{bold}} \newcommand{\la}{\lambda}
  \newcommand{\La}{\Lambda} \newcommand{\va}{\varepsilon}
  \newcommand{\be}{\beta} \newcommand{\al}{\alpha}
  \newcommand{\dis}{\displaystyle} \newcommand{\R}{{\mathbb R}}
  \newcommand{\N}{{\mathbb N}} \newcommand{\cF}{{\mathcal F}}
  \newcommand{\gB}{{\mathfrak B}} \newcommand{\eps}{\epsilon}
  \begin{flushright} % Right align
    {\LARGE\@title} % Increase the font size of the title
    
    \vspace{50pt} % Some vertical space between the title and author name
    
    {\large\@author} % Author name
    \\\@date % Date
    
    \vspace{40pt} % Some vertical space between the author block and abstract
  \end{flushright}
}

% ----------------------------------------------------------------------------------------
%	TITLE
% ----------------------------------------------------------------------------------------

\title{\textbf{《常微分方程教程》\footnote{\cite{dinglichang}}例2.3.2}} 

\author{\small{叶卢庆}\\{\small{杭州师范大学理学院,学号:1002011005}}\\{\small{Email:h5411167@gmail.com}}} % Institution
\renewcommand{\today}{\number\year. \number\month. \number\day}
\date{\today} % Date

% ----------------------------------------------------------------------------------------

\begin{document}
\maketitle % Print the title section

% ----------------------------------------------------------------------------------------
%	ABSTRACT AND KEYWORDS
% ----------------------------------------------------------------------------------------

% \renewcommand{\abstractname}{摘要} % Uncomment to change the name of the abstract to something else

% \begin{abstract}

% \end{abstract}

% \hspace*{3,6mm}\textit{关键词:}  % Keywords

% \vspace{30pt} % Some vertical space between the abstract and first section

% ----------------------------------------------------------------------------------------
%	ESSAY BODY
% ----------------------------------------------------------------------------------------
\begin{example}[2.3.2]
设微分方程
\begin{equation}
  \label{eq:1}
  \frac{dy}{dx}+ay=f(x),
\end{equation}
其中 $a>0$ 为常数,而 $f(x)$ 是以 $2\pi$ 为周期的连续函数.试求方程
\eqref{eq:1} 的 $2\pi$ 周期解.
\end{example}
\begin{proof}[解]
  方程 \eqref{eq:1} 化为
  \begin{equation}
    \label{eq:2}
    dy+(ay-f(x))dx=0.
  \end{equation}
两边同时乘以非零函数 $u(x)$,得到
\begin{equation}
  \label{eq:3}
  u(x)dy+u(x)(ay-f(x))dx=0.
\end{equation}
假设 \eqref{eq:3} 是恰当方程,则
\begin{equation}
  \label{eq:4}
  \frac{du(x)}{dx}=au(x)\ri u(x)=be^{ax},\mbox{不妨设}b=1.
\end{equation}
于是我们得到恰当方程
\begin{equation}
  \label{eq:5}
  e^{ax}dy+e^{ax}(ay-f(x))dx=0.
\end{equation}
设对于二元函数 $\phi(x,y)$ 来说,
\begin{equation}
  \label{eq:6}
  \frac{\pa\phi}{\pa y}=e^{ax}\ri \phi=ye^{ax}+g(x).
\end{equation}
因此
\begin{equation}
  \label{eq:7}
yae^{ax}+g'(x)=yae^{ax}-f(x)e^{ax}\ri g'(x)=-f(x)e^{ax}.
\end{equation}
因此
\begin{equation}
  \label{eq:8}
  g(x)=-\int f(x)e^{ax}dx+C.
\end{equation}
因此通积分为
\begin{equation}
  \label{eq:9}
 ye^{ax}-\int f(x)e^{ax}dx+C=0. 
\end{equation}
设 $H(x)=\int f(x)e^{ax}$,则$x=t$ 时,
\begin{equation}
  \label{eq:10}
  y_{t}e^{at}-H(t)+C=0.
\end{equation}
当 $x=2\pi+t$ 时,
\begin{equation}
  \label{eq:11}
  y_{2\pi+t}e^{2\pi a+ta}-H(2\pi+t)+C=0.
\end{equation}
式 \eqref{eq:10} 减去式 \eqref{eq:11} 可得
\begin{equation}
  \label{eq:12}
  y_te^{at}-y_{2\pi+t}e^{2\pi a+ta}+\int_t^{2\pi+t}f(x)e^{ax}=0.
\end{equation}
由于是 $2\pi$ 周期解,因此 $y_t=y_{2\pi+t}$,因此
\begin{equation}
  \label{eq:13}
  y_t=\frac{\int_t^{2\pi +t}f(x)e^{ax}}{e^{2\pi
      a+ta}-e^{at}}=\frac{\int_{t}^{2\pi+t}f(x)e^{ax-at}dx}{e^{2\pi a}-1}.
\end{equation}
\end{proof}
% ----------------------------------------------------------------------------------------
%	BIBLIOGRAPHY
% ----------------------------------------------------------------------------------------

\bibliographystyle{unsrt}

\bibliography{sample}

% ----------------------------------------------------------------------------------------
\end{document}