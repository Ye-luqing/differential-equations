\documentclass[a4paper, 12pt]{article} % Font size (can be 10pt, 11pt or 12pt) and paper size (remove a4paper for US letter paper)
\usepackage{amsmath,amsfonts,bm}
\usepackage{hyperref,verbatim}
\usepackage{amsthm,epigraph} 
\usepackage{amssymb}
\usepackage{framed,mdframed}
\usepackage{graphicx,color} 
\usepackage{mathrsfs,xcolor} 
\usepackage[all]{xy}
\usepackage{fancybox} 
\usepackage{xeCJK}
\newtheorem*{adtheorem}{定理}
\setCJKmainfont[BoldFont=FZYaoTi,ItalicFont=FZYaoTi]{FZYaoTi}
\definecolor{shadecolor}{rgb}{1.0,0.9,0.9} %背景色为浅红色
\newenvironment{theorem}
{\bigskip\begin{mdframed}[backgroundcolor=gray!40,rightline=false,leftline=false,topline=false,bottomline=false]\begin{adtheorem}}
    {\end{adtheorem}\end{mdframed}\bigskip}
\newtheorem*{bdtheorem}{定义}
\newenvironment{definition}
{\bigskip\begin{mdframed}[backgroundcolor=gray!40,rightline=false,leftline=false,topline=false,bottomline=false]\begin{bdtheorem}}
    {\end{bdtheorem}\end{mdframed}\bigskip}
\newtheorem*{cdtheorem}{习题}
\newenvironment{exercise}
{\bigskip\begin{mdframed}[backgroundcolor=gray!40,rightline=false,leftline=false,topline=false,bottomline=false]\begin{cdtheorem}}
    {\end{cdtheorem}\end{mdframed}\bigskip}
\newtheorem*{ddtheorem}{注}
\newenvironment{remark}
{\bigskip\begin{mdframed}[backgroundcolor=gray!40,rightline=false,leftline=false,topline=false,bottomline=false]\begin{ddtheorem}}
    {\end{ddtheorem}\end{mdframed}\bigskip}
\newtheorem*{edtheorem}{引理}
\newenvironment{lemma}
{\bigskip\begin{mdframed}[backgroundcolor=gray!40,rightline=false,leftline=false,topline=false,bottomline=false]\begin{edtheorem}}
    {\end{edtheorem}\end{mdframed}\bigskip}
\newtheorem*{pdtheorem}{例}
\newenvironment{example}
{\bigskip\begin{mdframed}[backgroundcolor=gray!40,rightline=false,leftline=false,topline=false,bottomline=false]\begin{pdtheorem}}
    {\end{pdtheorem}\end{mdframed}\bigskip}

\usepackage[protrusion=true,expansion=true]{microtype} % Better typography
\usepackage{wrapfig} % Allows in-line images
\usepackage{mathpazo} % Use the Palatino font
\usepackage[T1]{fontenc} % Required for accented characters
\linespread{1.05} % Change line spacing here, Palatino benefits from a slight increase by default

\makeatletter
\renewcommand\@biblabel[1]{\textbf{#1.}} % Change the square brackets for each bibliography item from '[1]' to '1.'
\renewcommand{\@listI}{\itemsep=0pt} % Reduce the space between items in the itemize and enumerate environments and the bibliography

\renewcommand{\maketitle}{ % Customize the title - do not edit title
  % and author name here, see the TITLE block
  % below
  \renewcommand\refname{参考文献}
  \newcommand{\D}{\displaystyle}\newcommand{\ri}{\Rightarrow}
  \newcommand{\ds}{\displaystyle} \renewcommand{\ni}{\noindent}
  \newcommand{\pa}{\partial} \newcommand{\Om}{\Omega}
  \newcommand{\om}{\omega} \newcommand{\sik}{\sum_{i=1}^k}
  \newcommand{\vov}{\Vert\omega\Vert} \newcommand{\Umy}{U_{\mu_i,y^i}}
  \newcommand{\lamns}{\lambda_n^{^{\scriptstyle\sigma}}}
  \newcommand{\chiomn}{\chi_{_{\Omega_n}}}
  \newcommand{\ullim}{\underline{\lim}} \newcommand{\bsy}{\boldsymbol}
  \newcommand{\mvb}{\mathversion{bold}} \newcommand{\la}{\lambda}
  \newcommand{\La}{\Lambda} \newcommand{\va}{\varepsilon}
  \newcommand{\be}{\beta} \newcommand{\al}{\alpha}
  \newcommand{\dis}{\displaystyle} \newcommand{\R}{{\mathbb R}}
  \newcommand{\N}{{\mathbb N}} \newcommand{\cF}{{\mathcal F}}
  \newcommand{\gB}{{\mathfrak B}} \newcommand{\eps}{\epsilon}
  \begin{flushright} % Right align
    {\LARGE\@title} % Increase the font size of the title
    
    \vspace{50pt} % Some vertical space between the title and author name
    
    {\large\@author} % Author name
    \\\@date % Date
    
    \vspace{40pt} % Some vertical space between the author block and abstract
  \end{flushright}
}

% ----------------------------------------------------------------------------------------
%	TITLE
% ----------------------------------------------------------------------------------------
\begin{document}
\title{\textbf{习题2.5.1.6}} 
% \setlength\epigraphwidth{0.7\linewidth}
\author{\small{叶卢庆}\\{\small{杭州师范大学理学院,学号:1002011005}}\\{\small{Email:h5411167@gmail.com}}} % Institution
\renewcommand{\today}{\number\year. \number\month. \number\day}
\date{\today} % Date

% ----------------------------------------------------------------------------------------


\maketitle % Print the title section

% ----------------------------------------------------------------------------------------
%	ABSTRACT AND KEYWORDS
% ----------------------------------------------------------------------------------------

% \renewcommand{\abstractname}{摘要} % Uncomment to change the name of the abstract to something else

% \begin{abstract}

% \end{abstract}

% \hspace*{3,6mm}\textit{关键词:}  % Keywords

% \vspace{30pt} % Some vertical space between the abstract and first section

% ----------------------------------------------------------------------------------------
%	ESSAY BODY
% ----------------------------------------------------------------------------------------
\begin{exercise}[2.5.1.6]
解常微分方程
\begin{equation}
  \label{eq:0}
  y(1+xy)dx-xdy=0.
\end{equation}
\end{exercise}
\begin{proof}[解]
在微分方程 \eqref{eq:0} 的两边同时乘以非零函数 $u(x,y)$,可得
\begin{equation}
  \label{eq:2}
  u(x,y)y(1+xy)dx-u(x,y)xdy=0.
\end{equation}
我们希望 \eqref{eq:2} 是一个恰当微分方程,即
$$
\frac{\pa u}{\pa y}y(1+xy)+u(2+2xy)=-\frac{\pa u}{\pa x}x.
$$
不妨让 $u$ 只是关于 $y$ 的函数,则我们有
$$
\frac{du}{dy}y(1+xy)+2u(1+xy)=0.
$$
当 $xy\neq -1$ 时,即
$$
\frac{du}{dy}y+2u=0.
$$
当 $y\neq 0$ 时,不妨让 $u=\frac{1}{y^2}$.因此我们得到恰当微分方程
$$
(\frac{1}{y}+x)dx-\frac{x}{y^2}dy=0.
$$
设存在二元函数 $\phi(x,y)$,使得
$$
\frac{\pa\phi}{\pa x}=\frac{1}{y}+x\ri \phi=\frac{x}{y}+\frac{1}{2}x^2+f(y).
$$
因此,
$$
-\frac{x}{y^2}+f'(y)=\frac{-x}{y^2}\ri f(y)=C.
$$
因此我们得到通积分
$$
\frac{x}{y}+\frac{1}{2}x^2+C=0.
$$
而当 $xy=-1$ 时,可得 $xdy+0dx=0$,因此 $y=D$,$x=\frac{-1}{D}$.而当
$y=0$ 时,我们有解 $y=0$,$x$ 任意.
\begin{comment}
我们将微分方程分组,得到
$$
(ydx-xdy)+xydx=0.
$$
对于微分方程
\begin{equation}
  \label{eq:1}
  ydx-xdy=0
\end{equation}
来说,当 $y\neq 0$ 时,其积分因子为 $\frac{1}{y^2}$,将 \eqref{eq:1} 两边乘以积分因子,可
得恰当微分方程
\begin{equation}
  \label{eq:2}
  \frac{1}{y}dx-\frac{x}{y^{2}}dy=0.
l\end{equation}
设二元函数 $\phi(x,y)$ 满足
$$
\frac{\pa\phi}{\pa x}=\frac{1}{y}\ri \phi=\frac{x}{y}+f(y).
$$
因此
$$
-\frac{x}{y^2}+f'(y)=\frac{-x}{y^2}\ri f(y)=C.
$$
可见,通积分为
$$
\frac{x}{y}+C=0.
$$
易得 $\frac{1}{y^2}g(\frac{x}{y})$ 也是 \eqref{eq:1} 的积分因子,其中
$g$ 是可微函数.下面我们来看另一个微分方程
\begin{equation}
  \label{eq:3}
  xydx=0.
\end{equation}
易得积分因子为 $\frac{1}{y}$.将积分因子乘上 \eqref{eq:3} 的时候,我们就
得到
$$
xdx=0,
$$
因此通积分为 $\frac{1}{2}x^2+C=0$.易得 $\frac{1}{y}h(x^2)$ 也是
\eqref{eq:3} 的一个积分因子,其中 $h$ 是可微函数.我们希望
$$
\frac{1}{y^2}g(\frac{x}{y})=\frac{1}{y}h(x^2).
$$
即
$$
\frac{1}{y}g(\frac{x}{y})=h(x^2).
$$
当 $x\neq 0$ 时,令 
$$
g(\frac{x}{y})=\frac{y}{x},h(x^2)=\frac{1}{x}.
$$
因此可得微分方程 \eqref{eq:0} 的积分因子为 $\frac{1}{xy}$.将积分因子乘
以方程 \eqref{eq:0} 的两侧,得到
\begin{equation}
  \label{eq:5}
  (\frac{1}{x}+y)dx-\frac{1}{y}dy=0.
\end{equation}
设存在二元函数 $\Phi(x,y)$,使得
$$
\frac{\pa\Phi}{\pa x}=\frac{1}{x}+y\ri \Phi=\ln|x| +xy+r(y).
$$
因此
$$
x+r'(y)
$$
\end{comment}







\end{proof}
% ----------------------------------------------------------------------------------------
%	BIBLIOGRAPHY
% ----------------------------------------------------------------------------------------

\bibliographystyle{unsrt}

\bibliography{sample}

% ----------------------------------------------------------------------------------------
\end{document}